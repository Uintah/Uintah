\documentclass[twoside,11pt]{article}
\usepackage[pdftex]{graphicx}
%\usepackage{times}
\usepackage{palatino}
\usepackage{epsf}
\usepackage{epsfig}
\usepackage{fleqn}
\usepackage{boxedminipage}
\usepackage{amsfonts, amssymb, amsmath}
\usepackage{stmaryrd}
\usepackage{cancel}
%\usepackage[noplots]{maple2e}
\usepackage[usenames]{color}
\definecolor{red}{rgb}{1,0,0}
\definecolor{green}{rgb}{0,1,0}
\definecolor{blue}{rgb}{0,0,1}
\definecolor{Red}{rgb}{0.8666,0.03137,0.02352}
\definecolor{Blue}{rgb}{0.00784,0.67059,0.91764}
\definecolor{Darkgreen}{rgb}{0,0.68235,0}
\definecolor{Green}{rgb}{0,0.8,0}
\definecolor{Bl}{rgb}{0,0.2,0.91764}
\definecolor{Royalblue}{rgb}{0,0.2,0.91764}
\definecolor{Brickred}{rgb}{0.644541,0.164065,0.164065}
\definecolor{Brown}{rgb}{0.6,0.4,0.4}
\definecolor{Orange}{rgb}{1,0.647059,0}
\definecolor{Indigo}{rgb}{0.746105,0,0.996109}
\definecolor{Violet}{rgb}{0.308598,0.183597,0.308598}
\definecolor{Lightgrey}{rgb}{0.762951,0.762951,0.762951}
\definecolor{Darkgrey}{rgb}{0.503548,0.503548,0.503548}
\definecolor{Pink}{rgb}{1,0.6,0.6}
\definecolor{MyLightMagenta}{cmyk}{0.1,0.8,0,0.1}
\definecolor{MyDarkBlue}{rgb}{0,0.08,0.45}

\raggedright
%\setlength{\parsep}{0pt}
%\setlength{\itemsep}{0pt}
\setlength{\parskip}{8pt}
%\setlength{\parsep}{-8pt}
%\setlength{\itemsep}{-8pt}
%\setlength{\parindent}{0pt}
\thispagestyle{empty}
%\pagestyle{plain}
\pagestyle{myheadings}

% The page layout plan.

\setlength{\textheight}{9true in}
\setlength{\textwidth}{7.0true in}
\setlength{\topmargin}{-0.3true in}
\setlength{\headheight}{0.1true in}
\setlength{\headsep}{0.2true in}
\setlength{\footskip}{0.2true in}
\setlength{\topskip}{12pt}
\setlength{\oddsidemargin}{-0.3true in}
\setlength{\evensidemargin}{-0.3true in}

\newcommand{\handout}[3]{
        \begin{center}
         % \copyright 
          Biswajit Banerjee \hspace{4.2in} University of Utah\\
          \vspace{10pt}
          {\Large\bf Waves in Composites and Metamaterials}\\
          \vspace{6pt}
          (Instructor: Prof. G. W. Milton)
        \end{center}
        \vspace{8pt}\noindent
        %\begin{center}
        {\underline{\makebox[7.0in]{\large\bf\noindent
                \makebox[1.5in][l]{#1~~~} \hfill {~~~#2~~~} \hfill
                \makebox[1.5in][r]{~~~#3}}}}
        %\end{center}
        }

\newcommand{\heading}[1]{
        \begin{center}{\large\bf{#1}}\end{center}}

\newcommand{\subheading}[1]{
        \begin{center}{\normalsize\bf{#1}}\end{center}}

\newcommand{\subsubheading}[1]{
        \begin{center}{\small\bf{#1}}\end{center}}

\newcommand{\Heading}[1]{
        \vspace{12pt}\begin{center}{\Large\bf{#1}}\end{center}}

\newcommand{\Subheading}[1]{
        \vspace{8pt}\begin{center}{\large\bf{#1}}\end{center}}

\newcommand{\Red}{\color{Brickred}}
\newcommand{\Blue}{\color{Royalblue}}
\newcommand{\Green}{\color{Darkgreen}}
\newcommand{\Violet}{\color{Violet}}
\newcommand{\That}{\ensuremath{\widehat{T}}}
\newcommand{\Jump}[1]{\ensuremath{\llbracket#1\rrbracket}}
\newcommand{\Blimitx}[1]{\ensuremath{\left[#1\right]_{x_a}^{x_b}}}
\newcommand{\Deriv}[2]{\ensuremath{\cfrac{d#1}{d#2}}}
\newcommand{\MDeriv}[2]{\ensuremath{\cfrac{D#1}{D#2}}}
\newcommand{\DDeriv}[2]{\ensuremath{\cfrac{d^2#1}{d#2^2}}}
\newcommand{\DDDeriv}[2]{\ensuremath{\cfrac{d^3#1}{d#2^3}}}
\newcommand{\DDDDeriv}[2]{\ensuremath{\cfrac{d^4#1}{d#2^4}}}
\newcommand{\Intx}{\ensuremath{\int_{x_a}^{x_b}}}
\newcommand{\IntX}{\ensuremath{\int_{X_a}^{X_b}}}
\newcommand{\Intiso}{\ensuremath{\int_{-1}^{1}}}
\newcommand{\IntOmegaA}{\ensuremath{\int_{\Omega_0}}}
\newcommand{\IntOmega}{\ensuremath{\int_{\Omega}}}
\newcommand{\IntDOmega}{\ensuremath{\int_{\partial\Omega}}}
\newcommand{\IntOmegap}{\ensuremath{\int_{\Omega'}}}
\newcommand{\Norm}[2]{\ensuremath{\left\lVert#1\right\rVert_{#2}}}
\newcommand{\norm}[1]{\ensuremath{\left\lVert#1\right\rVert}}
\newcommand{\Var}[1]{\ensuremath{\delta #1}}
\newcommand{\DelT}{\ensuremath{\Delta t}}
\newcommand{\CalD}{\ensuremath{\mathcal{D}}}
\newcommand{\BCalD}{\ensuremath{\boldsymbol{\CalD}}}
\newcommand{\CalF}{\ensuremath{\mathcal{F}}}
\newcommand{\CalL}{\ensuremath{\mathcal{L}}}
\newcommand{\CalM}{\ensuremath{\mathcal{M}}}
\newcommand{\BCalM}{\ensuremath{\boldsymbol{\CalM}}}
\newcommand{\CalN}{\ensuremath{\mathcal{N}}}
\newcommand{\CalP}{\ensuremath{\mathcal{P}}}
\newcommand{\CalS}{\ensuremath{\mathcal{S}}}
\newcommand{\BCalS}{\ensuremath{\boldsymbol{\CalS}}}
\newcommand{\CalT}{\ensuremath{\mathcal{T}}}
\newcommand{\CalV}{\ensuremath{\mathcal{V}}}
\newcommand{\CalW}{\ensuremath{\mathcal{W}}}
\newcommand{\CalX}{\ensuremath{\mathcal{X}}}
\newcommand{\Comp}[2]{\ensuremath{#1 \circ #2}}
\newcommand{\Map}[3]{\ensuremath{#1 : #2 \rightarrow #3}}
\newcommand{\MapTo}[3]{\ensuremath{#1 : #2 \mapsto #3}}
\newcommand{\Real}[1]{\ensuremath{\mathbb{R}^{#1}}}
\newcommand{\Ve}{\ensuremath{\varepsilon}}
\newcommand{\BHat}[1]{\ensuremath{\widehat{\boldsymbol{#1}}}}
\newcommand{\BTx}{\ensuremath{\tilde{\boldsymbol{x}}}}
\newcommand{\Beh}{\ensuremath{\hat{\boldsymbol{e}}}}
\newcommand{\BHex}{\ensuremath{\hat{\boldsymbol{e}}_1}}
\newcommand{\BHey}{\ensuremath{\hat{\boldsymbol{e}}_2}}
\newcommand{\BHez}{\ensuremath{\hat{\boldsymbol{e}}_3}}
\newcommand{\BHn}[1]{\ensuremath{\hat{\boldsymbol{n}}_{#1}}}
\newcommand{\BHe}[1]{\ensuremath{\hat{\boldsymbol{e}}_{#1}}}
\newcommand{\BHg}[1]{\ensuremath{\hat{\boldsymbol{g}}_{#1}}}
\newcommand{\BHG}[1]{\ensuremath{\hat{\boldsymbol{G}}_{#1}}}
\newcommand{\Hn}{\ensuremath{\hat{\boldsymbol{n}}}}
\newcommand{\Mba}{\ensuremath{\mathbf{a}}}
\newcommand{\Mbatilde}{\ensuremath{\widetilde{\mathbf{a}}}}
\newcommand{\Mbb}{\ensuremath{\mathbf{b}}}
\newcommand{\Mbd}{\ensuremath{\mathbf{d}}}
\newcommand{\Mbf}{\ensuremath{\mathbf{f}}}
\newcommand{\Mbn}{\ensuremath{\mathbf{n}}}
\newcommand{\Mbntilde}{\ensuremath{\widetilde{\mathbf{n}}}}
\newcommand{\Mbr}{\ensuremath{\mathbf{r}}}
\newcommand{\Mbu}{\ensuremath{\mathbf{u}}}
\newcommand{\Mbv}{\ensuremath{\mathbf{v}}}
\newcommand{\Mbx}{\ensuremath{\mathbf{x}}}
\newcommand{\MbA}{\ensuremath{\mathbf{A}}}
\newcommand{\MbB}{\ensuremath{\mathbf{B}}}
\newcommand{\MbC}{\ensuremath{\mathbf{C}}}
\newcommand{\MbD}{\ensuremath{\mathbf{D}}}
\newcommand{\MbH}{\ensuremath{\mathbf{H}}}
\newcommand{\MbHbar}{\ensuremath{\mathbf{\overline{H}}}}
\newcommand{\MbI}{\ensuremath{\mathbf{I}}}
\newcommand{\MbK}{\ensuremath{\mathbf{K}}}
\newcommand{\MbKbar}{\ensuremath{\overline{\mathbf{K}}}}
\newcommand{\MbKtilde}{\ensuremath{\widetilde{\mathbf{K}}}}
\newcommand{\MbM}{\ensuremath{\mathbf{M}}}
\newcommand{\MbN}{\ensuremath{\mathbf{N}}}
\newcommand{\MbP}{\ensuremath{\mathbf{P}}}
\newcommand{\MbPbar}{\ensuremath{\overline{\mathbf{P}}}}
\newcommand{\MbR}{\ensuremath{\mathbf{R}}}
\newcommand{\MbT}{\ensuremath{\mathbf{T}}}
\newcommand{\MbV}{\ensuremath{\mathbf{V}}}
\newcommand{\MbX}{\ensuremath{\mathbf{X}}}
\newcommand{\MbSig}{\ensuremath{\boldsymbol{\sigma}}}
\newcommand{\Mbone}{\ensuremath{\mathbf{1}}}
\newcommand{\Mbzero}{\ensuremath{\mathbf{0}}}
\newcommand{\Mb}{\ensuremath{\left[\mathsf{b}\right]}}
\newcommand{\Mu}{\ensuremath{\left[\mathsf{u}\right]}}
\newcommand{\Mv}{\ensuremath{\left[\mathsf{v}\right]}}
\newcommand{\Mw}{\ensuremath{\left[\mathsf{w}\right]}}
\newcommand{\Mx}{\ensuremath{\left[\mathsf{x}\right]}}
\newcommand{\MA}{\ensuremath{\left[\mathsf{A}\right]}}
\newcommand{\MC}{\ensuremath{\left[\mathsf{C}\right]}}
\newcommand{\MD}{\ensuremath{\left[\mathsf{D}\right]}}
\newcommand{\MH}{\ensuremath{\left[\mathsf{H}\right]}}
\newcommand{\MI}{\ensuremath{\left[\mathsf{I}\right]}}
\newcommand{\ML}{\ensuremath{\left[\mathsf{L}\right]}}
\newcommand{\MM}{\ensuremath{\left[\mathsf{M}\right]}}
\newcommand{\MN}{\ensuremath{\left[\mathsf{N}\right]}}
\newcommand{\MP}{\ensuremath{\left[\mathsf{P}\right]}}
\newcommand{\MR}{\ensuremath{\left[\mathsf{R}\right]}}
\newcommand{\MT}{\ensuremath{\left[\mathsf{T}\right]}}
\newcommand{\MV}{\ensuremath{\left[\mathsf{V}\right]}}
\newcommand{\Mone}{\ensuremath{\left[\mathsf{1}\right]}}
\newcommand{\Mzero}{\ensuremath{\left[\mathsf{0}\right]}}
\newcommand{\SfA}{\ensuremath{\boldsymbol{\mathsf{A}}}}
\newcommand{\SfB}{\ensuremath{\boldsymbol{\mathsf{B}}}}
\newcommand{\SfC}{\ensuremath{\boldsymbol{\mathsf{C}}}}
\newcommand{\SfD}{\ensuremath{\boldsymbol{\mathsf{D}}}}
\newcommand{\SfI}{\ensuremath{\boldsymbol{\mathsf{I}}}}
\newcommand{\SfL}{\ensuremath{\boldsymbol{\mathsf{L}}}}
\newcommand{\SfS}{\ensuremath{\boldsymbol{\mathsf{S}}}}
\newcommand{\SfT}{\ensuremath{\boldsymbol{\mathsf{T}}}}
\newcommand{\Msig}{\ensuremath{\left[\boldsymbol{\sigma}\right]}}
\newcommand{\Meps}{\ensuremath{\left[\boldsymbol{\varepsilon}\right]}}
\newcommand{\Ex}{\ensuremath{\boldsymbol{e}_1}}
\newcommand{\Ey}{\ensuremath{\boldsymbol{e}_2}}
\newcommand{\Ez}{\ensuremath{\boldsymbol{e}_3}}
\newcommand{\Exp}{\ensuremath{\boldsymbol{e}^{'}_1}}
\newcommand{\Eyp}{\ensuremath{\boldsymbol{e}^{'}_2}}
\newcommand{\Ezp}{\ensuremath{\boldsymbol{e}^{'}_3}}
\newcommand{\Ep}{\ensuremath{\varepsilon_p}}
\newcommand{\Epi}{\ensuremath{\varepsilon_{pi}}}
\newcommand{\Epdot}[1]{\ensuremath{\dot{\varepsilon}_{p#1}}}
\newcommand{\Epsxx}{\ensuremath{\varepsilon_{11}}}
\newcommand{\Epsyy}{\ensuremath{\varepsilon_{22}}}
\newcommand{\Epszz}{\ensuremath{\varepsilon_{33}}}
\newcommand{\Epsyz}{\ensuremath{\varepsilon_{23}}}
\newcommand{\Epszx}{\ensuremath{\varepsilon_{31}}}
\newcommand{\Epsxy}{\ensuremath{\varepsilon_{12}}}
\newcommand{\Sigxx}{\ensuremath{\sigma_{11}}}
\newcommand{\Sigyy}{\ensuremath{\sigma_{22}}}
\newcommand{\Sigzz}{\ensuremath{\sigma_{33}}}
\newcommand{\Sigyz}{\ensuremath{\sigma_{23}}}
\newcommand{\Sigzx}{\ensuremath{\sigma_{31}}}
\newcommand{\Sigxy}{\ensuremath{\sigma_{12}}}
\newcommand{\Eps}[1]{\ensuremath{\varepsilon_{#1}}}
\newcommand{\Sig}[1]{\ensuremath{\sigma_{#1}}}
\newcommand{\X}{\ensuremath{X_1}}
\newcommand{\Y}{\ensuremath{X_2}}
\newcommand{\Z}{\ensuremath{X_3}}
\newcommand{\erf}{\text{erf}}
\newcommand{\Xidot}{\ensuremath{\dot{\xi}}}
\newcommand{\Balpha}{\ensuremath{\boldsymbol{\alpha}}}
\newcommand{\Balphahat}{\ensuremath{\widehat{\boldsymbol{\alpha}}}}
\newcommand{\Bbeta}{\ensuremath{\boldsymbol{\beta}}}
\newcommand{\Bchi}{\ensuremath{\boldsymbol{\chi}}}
\newcommand{\Beta}{\ensuremath{\boldsymbol{\eta}}}
\newcommand{\Bveps}{\ensuremath{\boldsymbol{\varepsilon}}}
\newcommand{\BGamma}{\ensuremath{\boldsymbol{\mathit{\Gamma}}}}
\newcommand{\BGammahat}{\ensuremath{\boldsymbol{\mathit{\widehat{\Gamma}}}}}
%\newcommand{\BGammahat}{\ensuremath{\widehat{\BGamma}}}
\newcommand{\Bkappa}{\ensuremath{\boldsymbol{\kappa}}}
\newcommand{\Bbeps}{\ensuremath{\bar{\boldsymbol{\varepsilon}}}}
\newcommand{\Bnabla}{\ensuremath{\boldsymbol{\nabla}}}
\newcommand{\Bomega}{\ensuremath{\boldsymbol{\omega}}}
\newcommand{\Bsig}{\ensuremath{\boldsymbol{\sigma}}}
\newcommand{\Btau}{\ensuremath{\boldsymbol{\tau}}}
\newcommand{\Bpi}{\ensuremath{\boldsymbol{\pi}}}
\newcommand{\Brho}{\ensuremath{\boldsymbol{\rho}}}
\newcommand{\Bvarphi}{\ensuremath{\boldsymbol{\varphi}}}
\newcommand{\Blambda}{\ensuremath{\boldsymbol{\lambda}}}
\newcommand{\Btheta}{\ensuremath{\boldsymbol{\theta}}}
\newcommand{\Bmu}{\ensuremath{\boldsymbol{\mu}}}
\newcommand{\Bxi}{\ensuremath{\boldsymbol{\xi}}}
\newcommand{\BPi}{\ensuremath{\boldsymbol{\Pi}}}
\newcommand{\Bone}{\ensuremath{\boldsymbol{\mathit{1}}}}
\newcommand{\Bonev}{\ensuremath{\boldsymbol{1}}}
\newcommand{\Bzero}{\ensuremath{\boldsymbol{0}}}
\newcommand{\BzeroT}{\ensuremath{\boldsymbol{\mathit{0}}}}
\newcommand{\Ba}{\ensuremath{\mathbf{a}}}
\newcommand{\Bb}{\ensuremath{\mathbf{b}}}
\newcommand{\BbT}{\ensuremath{\boldsymbol{b}}}
\newcommand{\Bc}{\ensuremath{\mathbf{c}}}
\newcommand{\Bd}{\ensuremath{\boldsymbol{d}}}
\newcommand{\Bdv}{\ensuremath{\mathbf{d}}}
\newcommand{\Be}{\ensuremath{\mathbf{e}}}
\newcommand{\Bf}{\ensuremath{\mathbf{f}}}
\newcommand{\Bg}{\ensuremath{\mathbf{g}}}
\newcommand{\Bh}{\ensuremath{\boldsymbol{h}}}
\newcommand{\Bhv}{\ensuremath{\mathbf{h}}}
\newcommand{\Bk}{\ensuremath{\mathbf{k}}}
\newcommand{\Bl}{\ensuremath{\mathbf{l}}}
\newcommand{\Bm}{\ensuremath{\mathbf{m}}}
\newcommand{\Bn}{\ensuremath{\mathbf{n}}}
\newcommand{\Bo}{\ensuremath{\mathbf{o}}}
\newcommand{\Bp}{\ensuremath{\mathbf{p}}}
\newcommand{\Bq}{\ensuremath{\mathbf{q}}}
\newcommand{\Br}{\ensuremath{\boldsymbol{r}}}
\newcommand{\Brv}{\ensuremath{\mathbf{r}}}
\newcommand{\Bs}{\ensuremath{\boldsymbol{s}}}
\newcommand{\Bsv}{\ensuremath{\mathbf{s}}}
\newcommand{\Bt}{\ensuremath{\mathbf{t}}}
\newcommand{\Bu}{\ensuremath{\mathbf{u}}}
\newcommand{\Bv}{\ensuremath{\mathbf{v}}}
\newcommand{\Bw}{\ensuremath{\mathbf{w}}}
\newcommand{\Bx}{\ensuremath{\mathbf{x}}}
\newcommand{\By}{\ensuremath{\mathbf{y}}}
\newcommand{\BA}{\ensuremath{\boldsymbol{A}}}
\newcommand{\BB}{\ensuremath{\boldsymbol{B}}}
\newcommand{\BC}{\ensuremath{\boldsymbol{C}}}
\newcommand{\BD}{\ensuremath{\boldsymbol{D}}}
\newcommand{\BE}{\ensuremath{\boldsymbol{E}}}
\newcommand{\BF}{\ensuremath{\boldsymbol{F}}}
\newcommand{\BG}{\ensuremath{\boldsymbol{G}}}
\newcommand{\BH}{\ensuremath{\boldsymbol{H}}}
\newcommand{\BI}{\ensuremath{\boldsymbol{I}}}
\newcommand{\BJ}{\ensuremath{\boldsymbol{J}}}
\newcommand{\BK}{\ensuremath{\boldsymbol{K}}}
\newcommand{\BL}{\ensuremath{\boldsymbol{L}}}
\newcommand{\BM}{\ensuremath{\boldsymbol{M}}}
\newcommand{\BNv}{\ensuremath{\mathbf{N}}}
\newcommand{\BN}{\ensuremath{\boldsymbol{N}}}
\newcommand{\BP}{\ensuremath{\boldsymbol{P}}}
\newcommand{\BQ}{\ensuremath{\boldsymbol{Q}}}
\newcommand{\BR}{\ensuremath{\boldsymbol{R}}}
\newcommand{\BS}{\ensuremath{\boldsymbol{S}}}
\newcommand{\BT}{\ensuremath{\boldsymbol{T}}}
\newcommand{\BTv}{\ensuremath{\mathbf{T}}}
\newcommand{\BU}{\ensuremath{\boldsymbol{U}}}
\newcommand{\BW}{\ensuremath{\boldsymbol{W}}}
\newcommand{\BX}{\ensuremath{\mathbf{X}}}
\newcommand{\BXT}{\ensuremath{\boldsymbol{X}}}
\newcommand{\BY}{\ensuremath{\boldsymbol{Y}}}
\newcommand{\Dev}{\ensuremath{\text{dev}}}
\newcommand{\Trial}{\ensuremath{\text{trial}}}
\newcommand{\Tint}{\ensuremath{\text{int}}}
\newcommand{\Text}{\ensuremath{\text{ext}}}
\newcommand{\Tkin}{\ensuremath{\text{kin}}}
\newcommand{\Tbody}{\ensuremath{\text{body}}}
\newcommand{\Tmin}{\ensuremath{\text{min}}}
\newcommand{\Tmax}{\ensuremath{\text{max}}}
\newcommand{\Tor}{\ensuremath{\text{or}}}
\newcommand{\Tr}{\ensuremath{\text{tr}}}
\newcommand{\Half}{\ensuremath{\frac{1}{2}}}
\newcommand{\SThr}{\ensuremath{\sqrt{3}}}
\newcommand{\STT}{\ensuremath{\frac{\sqrt{3}}{2}}}
\newcommand{\Third}{\ensuremath{\frac{1}{3}}}
\newcommand{\Inner}[2]{\ensuremath{\langle#1,~#2\rangle}}
\newcommand{\Bcross}[2]{\ensuremath{#1\boldsymbol{\times}#2}}
\newcommand{\Bdot}[2]{\ensuremath{#1\cdot#2}}
\newcommand{\Dyad}[2]{\ensuremath{#1\boldsymbol{\otimes}#2}}
\newcommand{\Grad}[1]{\ensuremath{\Bnabla #1}}
\newcommand{\Gradp}[1]{\ensuremath{\Bnabla' #1}}
\newcommand{\Grads}[1]{\ensuremath{\Bnabla_s #1}}
\newcommand{\Grady}[1]{\ensuremath{\Bnabla_y #1}}
\newcommand{\Lap}[1]{\ensuremath{\nabla^2 #1}}
\newcommand{\Biharm}[1]{\ensuremath{\nabla^4 #1}}
\newcommand{\Div}[1]{\ensuremath{\Bdot{\Bnabla}{#1}}}
\newcommand{\Divp}[1]{\ensuremath{\Bdot{\Bnabla'}{#1}}}
\newcommand{\Divy}[1]{\ensuremath{\Bdot{\Bnabla_y}{#1}}}
\newcommand{\Curl}[1]{\ensuremath{\Bcross{\Bnabla}{#1}}}
\newcommand{\Curlp}[1]{\ensuremath{\Bcross{\Bnabla'}{#1}}}
\newcommand{\Curls}[1]{\ensuremath{\Bcross{\Bnabla_s}{#1}}}
\newcommand{\Curly}[1]{\ensuremath{\Bcross{\Bnabla_y}{#1}}}
\newcommand{\Gradu}{\ensuremath{\Grad{\Bu}}}
\newcommand{\Divu}{\ensuremath{\Div{\Bu}}}
\newcommand{\Curlu}{\ensuremath{\Curl{\Bu}}}
\newcommand{\Gradv}{\ensuremath{\Grad{\Bv}}}
\newcommand{\Divv}{\ensuremath{\Div{\Bv}}}
\newcommand{\Curlv}{\ensuremath{\Curl{\Bv}}}
\newcommand{\Dualn}{\ensuremath{\Bdual{\Bn}{\Bn}}}
\newcommand{\Over}[1]{\ensuremath{\frac{1}{#1}}}
\newcommand{\Diff}[2]{\ensuremath{\frac{d #1}{d #2}}}
\newcommand{\Partial}[2]{\ensuremath{\frac{\displaystyle\partial #1}{\displaystyle\partial #2}}}
\newcommand{\PPartial}[2]{\ensuremath{\frac{\partial^2 #1}{\partial #2^2}}}
\newcommand{\PPartialA}[3]{\ensuremath{\frac{\partial^2 #1}{\partial #2\partial#3}}}
\newcommand{\FPartial}[2]{\ensuremath{\frac{\partial^4 #1}{\partial #2^4}}}
\newcommand{\FPartialA}[3]{\ensuremath{\frac{\partial^4 #1}{\partial #2^2
         \partial #3^2}}}
\newcommand{\DotMbT}{\ensuremath{\dot{\MbT}}}
\newcommand{\TildeMbT}{\ensuremath{\widetilde{\MbT}}}
\newcommand{\BarT}{\ensuremath{\overline{T}}}
\newcommand{\Barq}{\ensuremath{\overline{q}}}
\newcommand{\Domega}{\ensuremath{\partial{\Omega}}}
\newcommand{\Av}[1]{\ensuremath{\left\langle#1\right\rangle}}
\newcommand{\AvSig}{\ensuremath{\langle\Bsig\rangle}}
\newcommand{\AvTau}{\ensuremath{\langle\Btau\rangle}}
\newcommand{\AvP}{\ensuremath{\langle\BP\rangle}}
\newcommand{\AvEps}{\ensuremath{\langle\Beps\rangle}}
\newcommand{\AvEpsdot}{\ensuremath{\langle\dot{\Beps}\rangle}}
\newcommand{\AvDisp}{\ensuremath{\langle\Bu\rangle}}
\newcommand{\AvF}{\ensuremath{\langle\BF\rangle}}
\newcommand{\AvFdot}{\ensuremath{\langle\dot{\BF}\rangle}}
\newcommand{\Avl}{\ensuremath{\overline{\Bl}}}
\newcommand{\AvSigBar}{\ensuremath{\overline{\Bsig}}}
\newcommand{\AvTauBar}{\ensuremath{\overline{\Btau}}}
\newcommand{\AvOmega}{\ensuremath{\langle\Bomega\rangle}}
\newcommand{\AvGradu}{\ensuremath{\langle\Gradu\rangle}}
\newcommand{\AvGradudot}{\ensuremath{\langle\Grad{\dot{\Bu}}\rangle}}
\newcommand{\AvGradv}{\ensuremath{\langle\Gradv\rangle}}
\newcommand{\AvPower}{\ensuremath{\langle\Bsig:\Gradv\rangle}}
\newcommand{\AvPowerInf}{\ensuremath{\langle\Bsig:\dot{\Beps}\rangle}}
\newcommand{\AvWorkInf}{\ensuremath{\langle\Bsig:\Beps\rangle}}
\newcommand{\AvPowerPF}{\ensuremath{\langle\BP^T:\dot{\BF}\rangle}}
\newcommand{\DA}{\ensuremath{\text{dA}}}
\newcommand{\DAvec}{\ensuremath{\text{d}\mathbf{A}}}
\newcommand{\Da}{\ensuremath{\text{da}}}
\newcommand{\Davec}{\ensuremath{\text{d}\mathbf{a}}}
\newcommand{\DV}{\ensuremath{\text{dV}}}
\newcommand{\BCe}{\ensuremath{\mathcal{E}}}
\newcommand{\GradX}[1]{\ensuremath{\Bnabla_0~#1}}
\newcommand{\DivX}[1]{\ensuremath{\Bdot{\Bnabla_0}{#1}}}
\newcommand{\Bxdot}{\ensuremath{\dot{\Bx}}}
\newcommand{\BFdot}{\ensuremath{\dot{\BF}}}
\newcommand{\BAv}{\ensuremath{\mathbf{A}}}
\newcommand{\BBv}{\ensuremath{\mathbf{B}}}
\newcommand{\BDv}{\ensuremath{\mathbf{D}}}
\newcommand{\BEv}{\ensuremath{\mathbf{E}}}
\newcommand{\BFv}{\ensuremath{\mathbf{F}}}
\newcommand{\BHv}{\ensuremath{\mathbf{H}}}
\newcommand{\BJv}{\ensuremath{\mathbf{J}}}
\newcommand{\BMv}{\ensuremath{\mathbf{M}}}
\newcommand{\BPv}{\ensuremath{\mathbf{P}}}
\newcommand{\BRv}{\ensuremath{\mathbf{R}}}
\newcommand{\BVv}{\ensuremath{\mathbf{V}}}
\newcommand{\Bdelta}{\ensuremath{\boldsymbol{\delta}}}
\newcommand{\Beps}{\ensuremath{\boldsymbol{\epsilon}}}
\newcommand{\BVeps}{\ensuremath{\boldsymbol{\varepsilon}}}
\newcommand{\BDtildev}{\ensuremath{\mathbf{\widetilde{D}}}}
\newcommand{\IntInfT}{\ensuremath{\int_{-\infty}^t}}
\newcommand{\IntInfInf}{\ensuremath{\int_{-\infty}^{\infty}}}
\newcommand{\IntInfZero}{\ensuremath{\int_{-\infty}^{0}}}
\newcommand{\IntZeroInf}{\ensuremath{\int_{0}^{\infty}}}
\newcommand{\IIntInfInf}{\ensuremath{\int_{-\infty}^{\infty}\int_{-\infty}^{\infty}}}
\newcommand{\IIIntInfInf}{\ensuremath{\int_{-\infty}^{\infty}\int_{-\infty}^{\infty}\int_{-\infty}^{\infty}}}
\newcommand{\Dtau}{\ensuremath{\text{d}\tau}}
\newcommand{\domega}{\ensuremath{\text{d}\omega}}
\newcommand{\dOmega}{\ensuremath{\text{d}\Omega}}
\newcommand{\dGamma}{\ensuremath{\text{d}\Gamma}}
\newcommand{\dzeta}{\ensuremath{\text{d}\zeta}}
\newcommand{\Ds}{\ensuremath{\text{d}s}}
\newcommand{\Dt}{\ensuremath{\text{d}t}}
\newcommand{\Dx}{\ensuremath{\text{d}\Bx}}
\newcommand{\dr}{\ensuremath{\text{d}r}}
\newcommand{\dx}{\ensuremath{\text{d}x}}
\newcommand{\dy}{\ensuremath{\text{d}y}}
\newcommand{\dz}{\ensuremath{\text{d}z}}
\newcommand{\dk}{\ensuremath{\text{d}k}}
\newcommand{\dBx}{\ensuremath{\text{d}\Bx}}
\newcommand{\dBk}{\ensuremath{\text{d}\Bk}}
\newcommand{\dBr}{\ensuremath{\text{d}\Br}}
\newcommand{\BKbar}{\ensuremath{\boldsymbol{\bar{K}}}}
\newcommand{\Bahat}{\ensuremath{\widehat{\Ba}}}
\newcommand{\Bbhat}{\ensuremath{\widehat{\Bb}}}
\newcommand{\BAhat}{\ensuremath{\widehat{\BA}}}
\newcommand{\BBhat}{\ensuremath{\widehat{\BB}}}
\newcommand{\BBhatv}{\ensuremath{\mathbf{\widehat{B}}}}
\newcommand{\BDhatv}{\ensuremath{\mathbf{\widehat{D}}}}
\newcommand{\BEhatv}{\ensuremath{\mathbf{\widehat{E}}}}
\newcommand{\BFhatv}{\ensuremath{\mathbf{\widehat{F}}}}
\newcommand{\BHhatv}{\ensuremath{\mathbf{\widehat{H}}}}
\newcommand{\BPhatv}{\ensuremath{\mathbf{\widehat{P}}}}
\newcommand{\BVhatv}{\ensuremath{\mathbf{\widehat{V}}}}
\newcommand{\Rea}{\ensuremath{\text{Re}}}
\newcommand{\Img}{\ensuremath{\text{Im}}}
\newcommand{\Teff}{\ensuremath{\text{eff}}}
\newcommand{\Tand}{\ensuremath{\text{and}}}
\newcommand{\CalE}{\ensuremath{\mathcal{E}}}
\newcommand{\CalH}{\ensuremath{\mathcal{H}}}
\newcommand{\CalJ}{\ensuremath{\mathcal{J}}}
\newcommand{\CalU}{\ensuremath{\mathcal{U}}}
\newcommand{\Dhat}{\ensuremath{\widehat{D}}}
\newcommand{\Ehat}{\ensuremath{\widehat{E}}}
\newcommand{\Fhat}{\ensuremath{\widehat{F}}}
\newcommand{\Phat}{\ensuremath{\widehat{P}}}
\newcommand{\Uhat}{\ensuremath{\widehat{U}}}
\newcommand{\Vhat}{\ensuremath{\widehat{V}}}
\newcommand{\fhat}{\ensuremath{\widehat{f}}}
\newcommand{\ghat}{\ensuremath{\widehat{g}}}
\newcommand{\phat}{\ensuremath{\widehat{p}}}
\newcommand{\uhat}{\ensuremath{\widehat{u}}}
\newcommand{\vhat}{\ensuremath{\widehat{v}}}
\newcommand{\xhat}{\ensuremath{\widehat{x}}}
\newcommand{\yhat}{\ensuremath{\widehat{y}}}
\newcommand{\Beq}{\begin{equation}}
\newcommand{\Eeq}{\end{equation}}
\newcommand{\Bal}{\begin{aligned}}
\newcommand{\Eal}{\end{aligned}}
\newcommand{\Tbar}{\ensuremath{\text{bar}}}
\newcommand{\Tball}{\ensuremath{\text{ball}}}
\newcommand{\Buhat}{\ensuremath{\widehat{\Bu}}}
\newcommand{\BHhat}{\ensuremath{\widehat{\BH}}}
\newcommand{\Bsighat}{\ensuremath{\widehat{\Bsig}}}
\newcommand{\Bepshat}{\ensuremath{\widehat{\Beps}}}
\newcommand{\sighat}{\ensuremath{\widehat{\sigma}}}
\newcommand{\epshat}{\ensuremath{\widehat{\epsilon}}}
\newcommand{\rhohat}{\ensuremath{\widehat{\rho}}}
\newcommand{\phihat}{\ensuremath{\widehat{\varphi}}}
\newcommand{\kappahat}{\ensuremath{\widehat{\kappa}}}
\newcommand{\SfK}{\ensuremath{\boldsymbol{\mathsf{K}}}}
\newcommand{\SfZero}{\ensuremath{\boldsymbol{\mathsf{0}}}}
\newcommand{\Gradbar}[1]{\ensuremath{\overline{\Bnabla} #1}}
\newcommand{\Divbar}[1]{\ensuremath{\Bdot{\overline{\Bnabla}}{#1}}}
\newcommand{\ktilde}{\ensuremath{\widetilde{k}}}
\newcommand{\Etilde}{\ensuremath{\widetilde{E}}}
\newcommand{\Htilde}{\ensuremath{\widetilde{H}}}
\newcommand{\Rtilde}{\ensuremath{\widetilde{R}}}
\newcommand{\Ttilde}{\ensuremath{\widetilde{T}}}
\newcommand{\TAi}{\ensuremath{\text{Ai}}}
\newcommand{\TBi}{\ensuremath{\text{Bi}}}
\newcommand{\sgn}{\ensuremath{\text{sgn}}}
\newcommand{\DelTwo}{\ensuremath{\Delta/2}}
\newcommand{\Bepseff}{\ensuremath{\Beps_\Teff}}
\newcommand{\Bmueff}{\ensuremath{\Bmu_\Teff}}
\newcommand{\epseff}{\ensuremath{\epsilon_\Teff}}
\newcommand{\mueff}{\ensuremath{\mu_\Teff}}
\newcommand{\BAeff}{\ensuremath{\BA_\Teff}}
\newcommand{\Conj}[1]{\ensuremath{\overline{#1}}}

\title{Material Models and the Uintah Computational Framework}

\begin{document}
\maketitle
\tableofcontents

\section{Introduction}
This document deals with some features of Uintah and some material models 
for solids that have been implemented in the Uintah Computational Framework 
(for use with the Material Point Method).  The approach taken has been to 
separate the stress-strain relations from the numerical stress update 
algorithms as far as possible.  A moderate rotation/small strain 
hypoelastic-plastic stress update algorithm is discussed and the manner in 
which plasticity flow rules, damage models and equations of state
fit into the stress update algorithms are shown.

\section{Load Curves}
It is often more convenient to apply a specified load at the MPM particles.
The load may be a function of time.  Such a load versus time curve is called
a {\bf load curve}.

In Uintah, the load curve infrastructure is available for general use
(and not only for particles).  However, it has been implemented only for
a special case of pressure loading.

We invoke the load curve in the \verb|<MPM>| section of the input file
using  \verb|<use_load_curves> true </use_load_curves>|.  The default value 
is \verb|<use_load_curves> false </use_load_curves>|.

In Uintah, a load curve infrastructure is implemented in the file
\verb|.../MPM/PhysicalBC/LoadCurve.h|.  This file is essentially a templated
structure that has the following private data
\begin{verbatim}
  // Load curve information 
  std::vector<double> d_time;
  std::vector<T> d_load;
  int d_id;
\end{verbatim}
The variable \verb|d_id| is the load curve ID, \verb|d_time| is the time,
and \verb|d_load| is the load.  Note that the load can have any form - scalar,
vector, matrix, etc.

In our current implementation, the actual specification of the load curve 
information is in the \verb|<PhysicalBC>| section of the input file.  The
implementation is limited in that it applies only to pressure boundary 
conditions for some special geometries (the implementation is in 
\verb|.../MPM/PhysicalBC/PressureBC.cc|).  However, the load curve template can
be used in other, more general, contexts.

A sample input file specification of a pressure load curve is shown below.
In this case, a pressure is applied to the inside and outside of a cylinder.
The pressure is ramped up from 0 to 1 GPa on the inside and from 0 to 0.1 MPa
on the outside over a time of 10 microsecs. 
\begin{verbatim}
   <PhysicalBC>
     <MPM>
       <pressure>
         <geom_object>
           <cylinder label = "inner cylinder">
             <bottom>           [0.0,0.0,0.0]   </bottom>
             <top>              [0.0,0.0,.02]   </top>
             <radius>           0.5             </radius>
           </cylinder>
         </geom_object>
         <load_curve>
           <id>1</id>
           <time_point>
             <time> 0 </time>
             <load> 0 </load>
           </time_point>
           <time_point>
             <time> 1.0e-5 </time>
             <load> 1.0e9 </load>
           </time_point>
         </load_curve>
       </pressure>
       <pressure>
         <geom_object>
           <cylinder label = "outer cylinder">
             <bottom>           [0.0,0.0,0.0]   </bottom>
             <top>              [0.0,0.0,.02]   </top>
             <radius>           1.0             </radius>
           </cylinder>
         </geom_object>
         <load_curve>
           <id>2</id>
           <time_point>
             <time> 0 </time>
             <load> 0 </load>
           </time_point>
           <time_point>
             <time> 1.0e-5 </time>
             <load> 101325.0 </load>
           </time_point>
         </load_curve>
       </pressure>
     </MPM>
   </PhysicalBC>
\end{verbatim}
The complete input file can be found in \verb|Uintah/StandAlone/inputs/MPM/thickCylinderMPM.ups|.

\section{Stress Update Algorithms}
The hypoelastic-plastic stress update is based on an additive
decomposition of the rate of deformation tensor into elastic
and plastic parts while the hyperelastic-plastic stress update
is based on a multiplicative decomposition of the elastic and 
plastic deformation gradients.  Incompressibility is assumed
for plastic deformations.  The volumetric response is therefore
determined either by a bulk modulus and the trace of the rate
of deformation tensor or using an equation of state.  The 
deviatoric response is determined either by an elastic constitutive 
equation or using a plastic flow rule in combination with a 
yield condition. 
  
The material models that can be varied in these stress update approaches
are (not all are applicable to both hypo- and hyperelastic formulations
nor is the list exhaustive):
\begin{enumerate}
    \item The elasticity model, for example, 
      \begin{itemize}
        \item Isotropic linear elastic model. 
        \item Anisotropic linear elastic models.
        \item Isotropic nonlinear elastic models.
        \item Anisotropic nonlinear elastic models.
      \end{itemize}
    \item Isotropic hardening or Kinematic hardening using a 
          back stress evolution rule, for example,
      \begin{itemize}
        \item Ziegler evolution rule   .
      \end{itemize}
    \item The flow rule and hardening/softening law, for example,
      \begin{itemize} 
        \item Perfect plasticity/power law hardening plasticity.
        \item Johnson-Cook plasticity .
        \item Mechanical Threshold Stress (MTS) plasticity .
        \item Anand plasticity .
      \end{itemize} 
    \item The yield condition, for example,
      \begin{itemize} 
        \item von Mises yield condition.
        \item Drucker-Prager yield condition.
        \item Mohr-Coulomb yield condition.
      \end{itemize} 
    \item A continuum or nonlocal damage model with damage evolution
          given by, for example, 
      \begin{itemize}
        \item Johnson-Cook damage model.
        \item Gurson-Needleman-Tvergaard model.
        \item Sandia damage model.
      \end{itemize}
    \item An equation of state to determine the pressure (or 
          volumetric response), for example, 
      \begin{itemize}
        \item Mie-Gruneisen equation of state.
      \end{itemize}
\end{enumerate}
  
The currently implemented stress update algorithms in the Uintah
Computational Framework do not allow for arbitrary elasticity models,
kinematic hardening, arbitrary yield conditions and continuum or 
nonlocal damage (however the a damage parameter is updated and 
used in the erosion algorithm).  The models that can be varied
are the flow rule models, damage variable evolution models and the
equation of state models.

Note that there are no checks in the Unitah Computational Framework
to prevent users from mixing and matching inappropriate models.

This section describes the current implementation of the hypoelastic-
plastic model.  The stress update algorithm is a slightly modified 
version of the approach taken by 
Nemat-Nasser et al. (1991,1992)~\cite{Nemat91,Nemat92}, Wang (1994)~\cite{Wang94}, 
Maudlin (1996)~\cite{Maudlin96}, and Zocher et al. (2000)~\cite{Zocher00}.  

\subsection{Simplified theory for hypoelastic-plasticity}
A simplified version of the theory behind the stress update algorithm
(in the contex of von Mises plasticity) is given below.

  Following ~\cite{Maudlin96}, the rotated spatial rate of deformation 
  tensor ($\Bd$) is decomposed into an elastic part ($\Bd^e$) and a 
  plastic part ($\Bd^p$)
  \begin{equation}
     \Bd = \Bd^e + \Bd^p
  \end{equation}
  If we assume plastic incompressibility ($\Tr{(\Bd^p)} = 0$), we get
  \begin{equation}
     \Beta = \Beta^e + \Beta^p
  \end{equation}
  where $\Beta$, $\Beta^e$, and $\Beta^p$ are the deviatoric parts of $\Bd$,
  $\Bd^e$, and $\Bd^p$, respectively.  For isotropic materials, the hypoelastic
  constitutive equation for deviatoric stress is
  \begin{equation}
    \dot{\Bs} = 2\mu(\Beta - \Beta^p)
  \end{equation}
  where $\Bs$ is the deviatoric part of the stress tensor and $\mu$ is the
  shear modulus.  We assume that the flow stress obeys the Huber-von Mises
  yield condition
  \begin{equation}
    f := \sqrt{\frac{3}{2}}\norm{\Bs} - \sigma_y \le 0  ~~\text{or,}~~
    F := \frac{3}{2} \Bs:\Bs - \sigma_y^2 \le 0 
  \end{equation}
  where $\sigma_y$ is the flow stress.  Assuming an associated flow rule,
  and noting that $\Bd^p = \Beta^p$, we have
  \begin{equation}
    \Beta^p = \Bd^p = \lambda\Partial{f}{\Bsig} 
                    = \Lambda\Partial{F}{\Bsig} = 3\Lambda\Bs
  \end{equation}
  where $\Bsig$ is the stress.  Let $\Bu$ be a tensor proportional to the 
  plastic straining direction, and define $\gamma$ as
  \begin{equation}
    \Bu = \sqrt{3} \frac{\Bs}{\norm{\Bs}}; \quad
    \gamma := \sqrt{3}\Lambda\norm{\Bs}  \quad \Longrightarrow
    \gamma\Bu = 3\Lambda\Bs
  \end{equation}
  Therefore, we have
  \begin{equation} \label{eq:stresseqn}
    \Beta^p = \gamma\Bu; \quad  
    \dot{\Bs} = 2\mu(\Beta - \gamma\Bu)
  \end{equation}
  From the consistency condition, if we assume that the deviatoric stress
  remains constant over a timestep, we get 
  \begin{equation}
    \gamma = \frac{\Bs:\Beta}{\Bs:\Bu}
  \end{equation}
  which provides an initial estimate of the plastic strain-rate.  To obtain
  a semi-implicit update of the stress using equation (\ref{eq:stresseqn}), we
  define
  \begin{equation}\label{eq:taueqn}
    \tau^2 := \frac{3}{2} \Bs:\Bs = \sigma_y^2
  \end{equation}
  Taking a time derivative of equation (\ref{eq:taueqn}) gives us
  \begin{equation}\label{eq:taudot}
    \sqrt{2} \dot{\tau} = \sqrt{3} \frac{\Bs:\dot{\Bs}}{\norm{\Bs}}
  \end{equation}
  Plugging equation (\ref{eq:taudot}) into equation (\ref{eq:stresseqn})$_2$
  we get
  \begin{equation}\label{eq:tau}
    \dot{\tau} = \sqrt{2}\mu(\Bu:\Beta - \gamma\Bu:\Bu)
               = \sqrt{2} \mu (d - 3\gamma)
  \end{equation} 
  where $d = \Bu:\Beta$.  If the initial estimate of the plastic strain-rate
  is that all of the deviatoric strain-rate is plastic, then we get an 
  approximation to $\gamma$, and the corresponding error 
  ($\gamma_{\text{er}}$) given by
  \begin{equation}\label{eq:gammaer}
    \gamma_{\text{approx}} = \frac{d}{3}; \quad
    \gamma_{\text{er}} = \gamma_{\text{approx}} - \gamma = \frac{d}{3} - \gamma
  \end{equation}
  The incremental form of the above equation is
  \begin{equation}\label{eq:delgamma}
    \Delta\gamma = \frac{d^*\Delta t}{3} - \Delta\gamma_{\text{er}}
  \end{equation}
  Integrating equation (\ref{eq:tau}) from time $t_n$ to time $t_{n+1} = 
  t_n + \Delta t$, and using equation (\ref{eq:delgamma}) we get
  \begin{equation}\label{eq:taun}
    \tau_{n+1} = \tau_n + \sqrt{2}\mu(d^*\Delta t - 3\Delta\gamma)
               = \tau_n + 3\sqrt{2}\mu\Delta\gamma_{\text{er}}
  \end{equation}
  where $d^*$ is the average value of $d$ over the timestep.
  Solving for $\Delta\gamma_{\text{er}}$ gives
  \begin{equation}\label{eq:delgammaer}
    \Delta\gamma_{\text{er}} = \cfrac{\tau_{n+1} - \tau_n}{3\sqrt{2}\mu}
      = \cfrac{\sqrt{2}\sigma_y - \sqrt{3}\norm{\Bs_n}}{6\mu}
  \end{equation}
  The direction of the total strain-rate ($\Bu^{\eta}$) and the
  direction of the plastic strain-rate ($\Bu^s$) are given by 
  \begin{equation}
    \Bu^{\eta} = \frac{\Beta}{\norm{\Beta}} ; \quad
    \Bu^{s} = \frac{\Bs}{\norm{\Bs}} 
  \end{equation}
  Let $\theta$ be the fraction of the time increment that sees elastic
  straining.  Then
  \begin{equation}\label{eq:theta}
    \theta = \frac{d^* - 3\gamma_n}{d^*}
  \end{equation}
  where $\gamma_n = d_n/3$ is the value of $\gamma$ at the beginning of the 
  timestep.  We also assume that 
  \begin{equation}\label{eq:dstar}
    d^* = \sqrt{3}\Beta:[(1-\theta)\Bu^{\eta} + \frac{\theta}{2}
              (\Bu^{\eta}+\Bu^{s})]
  \end{equation}
  Plugging equation (\ref{eq:theta}) into equation (\ref{eq:dstar}) we get
  a quadratic equation that can be solved for $d^*$ as follows
  \begin{equation}
    \frac{2}{\sqrt{3}} (d^*)^2 - (\Beta:\Bu^s + \norm{\Beta}) d^*
       + 3\gamma_n (\Beta:\Bu^s - \norm{\Beta}) = 0
  \end{equation}
  The real positive root of the above quadratic equation is taken as the
  estimate for $d$.  The value of $\Delta\gamma$ can now be calculated
  using equations (\ref{eq:delgamma}) and (\ref{eq:delgammaer}).  A
  semi-implicit estimate of the deviatoric stress can be obtained at this
  stage by integrating equation (\ref{eq:stresseqn})$_2$
  \begin{align}
    \tilde{\Bs}_{n+1} & = \Bs_n + 2\mu\left(\eta\Delta t  - \sqrt{3}\Delta\gamma
         \cfrac{\tilde{\Bs}_{n+1}}{\norm{\Bs_{n+1}}}\right) \\
     & = \Bs_n + 2\mu\left(\eta\Delta t  - \frac{3}{\sqrt{2}}\Delta\gamma
         \cfrac{\tilde{\Bs}_{n+1}}{\sigma_y}\right)
  \end{align}
  Solving for $\tilde{\Bs}_{n+1}$, we get
  \begin{equation}
    \tilde{\Bs}_{n+1} = \cfrac{\Bs_{n+1}^{\text{trial}}}
           {1 + 3\sqrt{2}\mu\cfrac{\Delta\gamma}{\sigma_y}}
  \end{equation}
  where $\Bs_{n+1}^{\text{trial}} = \Bs_n + 2\mu\Delta t\Beta$.
  A final radial return adjustment is used to move the stress to the yield
  surface 
  \begin{equation}
    \Bs_{n+1} = \sqrt{\frac{2}{3}}\sigma_y \cfrac{\tilde{\Bs}_{n+1}}
                {\norm{\tilde{\Bs}_{n+1}}}
  \end{equation}
  A pathological situation arises if $\gamma_n = \Bu_n:\Beta_n$ is less
  than or equal to zero or 
  $\Delta\gamma_{\text{er}} \ge \frac{d^*}{3} \Delta t $.  
  This can occur is the rate of plastic deformation
  is small compared to the rate of elastic deformation or if the timestep
  size is too small (see~\cite{Nemat92}).  In such situations, we use a
  locally implicit stress update that uses Newton iterations (as
  discussed in \cite{Simo98}, page 124) to compute $\tilde{\Bs}$.

\subsection{Models}
  Below are some of the strain-rate, strain, and temperature dependent models 
  for metals that are implemented in Uintah.

\subsection{Equation of State Models}
The elastic-plastic stress update assumes that the volumetric part of the 
Cauchy stress can be calculated using an equation of state.  There are 
three equations of state that are implemented in Uintah.  These are
\begin{enumerate}
    \item A default hypoelastic equation of state.
    \item A neo-Hookean equation of state.    
    \item A Mie-Gruneisen type equation of state.
\end{enumerate}

\paragraph{Default hypoelastic equation of state}
In this case we assume that the stress rate is given by
\Beq
    \dot{\Bsig} = \lambda~\Tr(\Bd^e)~\Bone + 2~\mu~\Bd^e
\Eeq
where $\Bsig$ is the Cauchy stress, $\Bd^e$ is the elastic part of
the rate of deformation, and $\lambda, \mu$ are constants.

If $\Beta^e$ is the deviatoric part of $\Bd^e$ then we can write
\Beq
    \dot{\Bsig} = \left(\lambda + \frac{2}{3}~\mu\right)~\Tr(\Bd^e)~\Bone + 
        2~\mu~\Beta^e = \kappa~\Tr(\Bd^e)~\Bone + 2~\mu~\Beta^e ~.
\Eeq
If we split $\Bsig$ into a volumetric and a deviatoric part, i.e.,
$\Bsig = p~\Bone + \Bs$ and take the time derivative to get
$\dot{\Bsig} = \dot{p}~\Bone + \dot{\Bs}$ then
\Beq
    \dot{p} = \kappa~\Tr(\Bd^e) ~.
\Eeq
In addition we assume that $\Bd = \Bd^e + \Bd^p$.  If we also assume that 
the plastic volume change is negligible, we can then write that
\Beq
    \dot{p} = \kappa~\Tr(\Bd) ~.
\Eeq
This is the equation that is used to calculate the pressure $p$ in the 
default hypoelastic equation of state, i.e.,
\Beq
    \boxed{
    p_{n+1} = p_n + \kappa~\Tr(\Bd_{n+1})~\Delta t ~.
    }
\Eeq
To get the derivative of $p$ with respect to $J$, where $J = \det(\BF)$,
we note that
\Beq
    \dot{p} = \Partial{p}{J}~\dot{J} = \Partial{p}{J}~J~\Tr(\Bd) ~.
\Eeq
Therefore,
\Beq
    \boxed{
    \Partial{p}{J} = \cfrac{\kappa}{J} ~.
    }
\Eeq

This model is invoked in Uintah using
\begin{verbatim}
  <equation_of_state type="default_hypo">
  </equation_of_state>
\end{verbatim}
The code is in \verb|.../MPM/ConstitutiveModel/PlasticityModels/DefaultHypoElasticEOS.cc|.

\paragraph{Default hyperelastic equation of state}
In this model the pressure is computed using the relation
\begin{equation}
  p = \Half~\kappa~\left(J^e - \cfrac{1}{J^e}\right)
\end{equation}
where $\kappa$ is the bulk modulus and $J^e$ is determinant of the elastic 
part of the deformation gradient.

We can also compute
\begin{equation}
  \Deriv{p}{J} = \Half~\kappa~\left(1 + \cfrac{1}{(J^e)^2}\right) ~.
\end{equation}

This model is invoked in Uintah using
\begin{verbatim}
  <equation_of_state type="default_hyper">
  </equation_of_state>
\end{verbatim}
The code is in \verb|.../MPM/ConstitutiveModel/PlasticityModels/HyperElasticEOS.cc|.  If an EOS is not specified then this model is the {\bf default}.

\paragraph{Mie-Gruneisen equation of state}
The pressure ($p$) is calculated using a Mie-Gr{\"u}neisen equation of state 
of the form (\cite{Wilkins99,Zocher00})
\begin{equation} \label{eq:EOSMG_upd}
  \boxed{
  p_{n+1} =  - \frac{\rho_0~C_0^2~(1 - J^e_{n+1})
           [1 - \Gamma_0 (1 - J^e_{n+1})/2]}
           {[1 - S_{\alpha}(1 - J^e_{n+1})]^2} - \Gamma_0~e_{n+1} 
  ~;~~~ J^e := \det{\BF^e} 
  }
\end{equation}
where $C_0$ is the bulk speed of sound, $\rho_0$ is the initial mass density,
$\Gamma_0$ is the Gr{\"u}neisen's gamma at the reference state,
$S_{\alpha} = dU_s/dU_p$ is a linear Hugoniot slope coefficient,
$U_s$ is the shock wave velocity, $U_p$ is the particle velocity, and
$e$ is the internal energy density (per unit reference volume), $\BF^e$ is
the elastic part of the deformation gradient.  For isochoric plasticity,
\begin{equation*}
  J^e = J = \det(\BF) = \cfrac{\rho_0}{\rho} ~.
\end{equation*}
  The internal energy is computed using
  \begin{equation}
    E = \frac{1}{V_0} \int C_v dT \approx \frac{C_v (T-T_0)}{V_0}
  \end{equation}
  where $V_0 = 1/\rho_0$ is the reference specific volume at temperature 
  $T = T_0$, and $C_v$ is the specific heat at constant volume.

Also,
\Beq
  \boxed{
  \Partial{p}{J^e} = 
  \cfrac{\rho_0~C_0^2~[1 + (S_{\alpha}-\Gamma_0)~(1-J^e)]}
        {[1-S_{\alpha}~(1-J^e)]^3} - \Gamma_0~\Partial{e}{J^e}.
  }
\Eeq
We neglect the $\Partial{e}{J^e}$ term in our calculations.

This model is invoked in Uintah using
\begin{verbatim}
  <equation_of_state type="mie_gruneisen">
    <C_0>5386</C_0>
    <Gamma_0>1.99</Gamma_0>
    <S_alpha>1.339</S_alpha>
  </equation_of_state>
\end{verbatim}
The code is in \verb|.../MPM/ConstitutiveModel/PlasticityModels/MieGruneisenEOS.cc|.

\subsubsection{Melting Temperature}
  \paragraph{Default model}
  The default model is to use a constant melting temperature.  This model
  is invoked using
  \begin{verbatim}  
      <melting_temp_model type="constant_Tm">
      </melting_temp_model>
  \end{verbatim}  
  
  \paragraph{SCG melt model}
  We use a pressure dependent relation to determine the melting 
  temperature ($T_m$).  The Steinberg-Cochran-Guinan (SCG) melt model 
  (\cite{Steinberg80}) has been used for our simulations of copper.  
  This model is based on a modified Lindemann law and has the form
  \begin{equation} \label{eq:TmSCG}
    T_m(\rho) = T_{m0} \exp\left[2a\left(1-\frac{1}{\eta}\right)\right]
              \eta^{2(\Gamma_0-a-1/3)}; \quad
    \eta = \frac{\rho}{\rho_0}
  \end{equation}
  where $T_{m0}$ is the melt temperature at $\eta = 1$, 
  $a$ is the coefficient of the first order volume correction to 
  Gr{\"u}neisen's gamma ($\Gamma_0$).  

  This model is invoked with
  \begin{verbatim}
    <melting_temp_model type="scg_Tm">
      <T_m0> 2310.0 </T_m0>
      <Gamma_0> 3.0 </Gamma_0>
      <a> 1.67 </a>
    </melting_temp_model>
  \end{verbatim}

  \paragraph{BPS melt model}
  An alternative melting relation that is based on dislocation-mediated
  phase transitions - the Burakovsky-Preston-Silbar (BPS) model
  (\cite{Burakovsky00}) can also be used.  This model has been used to 
  determine the melt temperature for 4340 steel.  The BPS model has the form
  \begin{align}
    T_m(p) & = T_m(0)
      \left[\cfrac{1}{\eta} + 
            \cfrac{1}{\eta^{4/3}}~\cfrac{\mu_0^{'}}{\mu_0}~p\right]~; 
    \quad
    \eta = \left(1 + \cfrac{K_0^{'}}{K_0}~p\right)^{1/K_0^{'}} 
    \label{eq:TmBPS}\\
    T_m(0) & = \cfrac{\kappa\lambda\mu_0~v_{WS}}{8\pi\ln(z-1)~k_b}
               \ln\left(\cfrac{\alpha^2}{4~b^2\rho_c(T_m)}\right)
  \end{align}
  where $p$ is the pressure, $\eta = \rho/\rho_0$ is the compression, 
  $\mu_0$ is the shear modulus at room temperature and zero pressure, 
  $\mu_0^{'} = \partial\mu/\partial p$ is the derivative of the shear modulus 
  at zero pressure, $K_0$ is the bulk modulus at room temperature and
  zero pressure, $K_0^{'} = \partial K/\partial p$ is the derivative of the 
  bulk modulus at zero pressure, $\kappa$ is a constant, $\lambda = b^3/v_{WS}$
  where $b$ is the magnitude of the Burgers' vector, $v_{WS}$ is the 
  Wigner-Seitz volume, $z$ is the coordination number, $\alpha$ is a 
  constant, $\rho_c(T_m)$ is the critical density of dislocations, and
  $k_b$ is the Boltzmann constant.

  This model is invoked with
  \begin{verbatim}
    <melting_temp_model type="bps_Tm">
      <B0> 137e9 </B0>
      <dB_dp0> 5.48 <dB_dp0>
      <G0> 47.7e9 <G0>
      <dG_dp0> 1.4 <dG_dp0>
      <kappa> 1.25 <kappa>
      <z> 12 <z>
      <b2rhoTm> 0.64 <b2rhoTm>
      <alpha> 2.9 <alpha>
      <lambda> 1.41 <lambda>
      <a> 3.6147e-9<a>
      <v_ws_a3_factor> 1/4 <v_ws_a3_factor>
      <Boltzmann_Constant> <Boltzmann_Constant>
    </melting_temp_model>
  \end{verbatim}

  \subsubsection{Shear Modulus} \label{sec:ModelShear}
  Three models for the shear modulus ($\mu$) have been tested in our 
  simulations.  The first has been associated with the Mechanical 
  Threshold Stress (MTS) model and we call it the MTS shear model.
  The second is the model used by Steinberg-Cochran-Guinan and we call
  it the SCG shear model while the third is a model developed by 
  Nadal and Le Poac that we call the NP shear model.

  \paragraph{Default model}
  The default model gives a constant shear modulus.  The model is 
  invoked using
  \begin{verbatim}
    <shear_modulus_model type="constant_shear">
    </shear_modulus_model>
  \end{verbatim}

  \paragraph{MTS Shear Modulus Model}
  The simplest model is of the form suggested by \cite{Varshni70} 
  (\cite{Chen96})
  \begin{equation} \label{eq:MTSShear}
    \mu(T) = \mu_0 - \frac{D}{exp(T_0/T) - 1}
  \end{equation}
  where $\mu_0$ is the shear modulus at 0K, and $D, T_0$ are material
  constants. 

  The model is invoked using
  \begin{verbatim}
    <shear_modulus_model type="mts_shear">
      <mu_0>28.0e9</mu_0>
      <D>4.50e9</D>
      <T_0>294</T_0>
    </shear_modulus_model>
  \end{verbatim}

  \paragraph{SCG Shear Modulus Model}
  The Steinberg-Cochran-Guinan (SCG) shear modulus 
  model (\cite{Steinberg80,Zocher00}) is pressure dependent and
  has the form
  \begin{equation} \label{eq:SCGShear}
    \mu(p,T) = \mu_0 + \Partial{\mu}{p} \frac{p}{\eta^{1/3}} +
         \Partial{\mu}{T}(T - 300) ; \quad
    \eta = \rho/\rho_0
  \end{equation}
  where, $\mu_0$ is the shear modulus at the reference state($T$ = 300 K, 
  $p$ = 0, $\eta$ = 1), $p$ is the pressure, and $T$ is the temperature.
  When the temperature is above $T_m$, the shear modulus is instantaneously
  set to zero in this model.

  The model is invoked using
  \begin{verbatim}
  <shear_modulus_model type="scg_shear">
    <mu_0> 81.8e9 </mu_0>
    <A> 20.6e-12 </A>
    <B> 0.16e-3 </B>
  </shear_modulus_model>
  \end{verbatim}

  \paragraph{NP Shear Modulus Model}
  A modified version of the SCG model has been developed by 
  \cite{Nadal03} that attempts to capture the sudden drop in the
  shear modulus close to the melting temperature in a smooth manner.
  The Nadal-LePoac (NP) shear modulus model has the form
  \begin{equation} \label{eq:NPShear}
    \mu(p,T) = \frac{1}{\mathcal{J}(\That)}
      \left[
        \left(\mu_0 + \Partial{\mu}{p} \cfrac{p}{\eta^{1/3}} \right)
        (1 - \That) + \frac{\rho}{Cm}~k_b~T\right]; \quad
    C := \cfrac{(6\pi^2)^{2/3}}{3} f^2
  \end{equation}
  where
  \begin{equation}
    \mathcal{J}(\That) := 1 + \exp\left[-\cfrac{1+1/\zeta}
        {1+\zeta/(1-\That)}\right] \quad
       \text{for} \quad \That:=\frac{T}{T_m}\in[0,1+\zeta],
  \end{equation}
  $\mu_0$ is the shear modulus at 0 K and ambient pressure, $\zeta$ is
  a material parameter, $k_b$ is the Boltzmann constant, $m$ is the atomic
  mass, and $f$ is the Lindemann constant.

  The model is invoked using
  \begin{verbatim}
    <shear_modulus_model type="np_shear">
      <mu_0>26.5e9</mu_0>
      <zeta>0.04</zeta>
      <slope_mu_p_over_mu0>65.0e-12</slope_mu_p_over_mu0>
      <C> 0.047 </C>
      <m> 26.98 </m>
    </shear_modulus_model>
  \end{verbatim}
  
  \paragraph{PTW Shear model}
  The PTW shear model is a simplified version of the SCG shear model.
  The inputs can be found in \verb|.../MPM/ConstitutiveModel/PlasticityModel/PTWShear.h|.

  \subsubsection{Flow Stress}
  We have explored five temperature and strain rate dependent 
  models that can be used to compute the flow stress:
  \begin{enumerate}
    \item the Johnson-Cook (JC) model 
    \item the Steinberg-Cochran-Guinan-Lund (SCG) model. 
    \item the Zerilli-Armstrong (ZA) model.
    \item the Mechanical Threshold Stress (MTS) model. 
    \item the Preston-Tonks-Wallace (PTW) model.
  \end{enumerate}

  \paragraph{JC Flow Stress Model}
  The Johnson-Cook (JC) model (\cite{Johnson83}) is purely empirical and gives 
  the following relation for the flow stress ($\sigma_y$) 
  \begin{equation}
    \sigma_y(\Ep,\Epdot{},T) = 
    \left[A + B (\Ep)^n\right]\left[1 + C \ln(\Epdot{}^{*})\right]
    \left[1 - (T^*)^m\right]
  \end{equation}
  where $\Ep$ is the equivalent plastic strain, $\Epdot{}$ is the 
  plastic strain rate, A, B, C, n, m are material constants, 
  \begin{equation}
    \Epdot{}^{*} = \cfrac{\Epdot{}}{\Epdot{0}}; \quad
    T^* = \cfrac{(T-T_0)}{(T_m-T_0)}~,
  \end{equation}
  $\Epdot{0}$ is a user defined plastic strain rate, 
  $T_0$ is a reference temperature, and $T_m$ is the melt temperature.  
  For conditions where $T^* < 0$, we assume that $m = 1$.

  The inputs for this model are
  \begin{verbatim}
  <plasticity_model type="johnson_cook">
    <A>792.0e6</A>
    <B>510.0e6</B>
    <C>0.014</C>
    <n>0.26</n>
    <m>1.03</m>
    <T_r>298.0</T_r>
    <T_m>1793.0</T_m>
    <epdot_0>1.0</epdot_0>
  </plasticity_model>
  \end{verbatim}

  \paragraph{SCG Flow Stress Model}
  The Steinberg-Cochran-Guinan-Lund (SCG) model is a semi-empirical model
  that was developed by \cite{Steinberg80} for high strain rate 
  situations and extended to low strain rates and bcc materials by
  \cite{Steinberg89}.  The flow stress in this model is given by
  \begin{equation}\label{eq:SCGL}
    \sigma_y(\Ep,\Epdot{},T) = 
     \left[\sigma_a f(\Ep) + \sigma_t (\Epdot{}, T)\right]
     \frac{\mu(p,T)}{\mu_0} 
  \end{equation}
  where $\sigma_a$ is the athermal component of the flow stress,
  $f(\Ep)$ is a function that represents strain hardening,
  $\sigma_t$ is the thermally activated component of the flow stress,
  $\mu(p,T)$ is the shear modulus, and $\mu_0$ is the shear modulus 
  at standard temperature and pressure.  The strain hardening function
  has the form
  \begin{equation}
    f(\Ep) = [1 + \beta(\Ep + \Epi)]^n ; \quad
    \sigma_a f(\Ep) \le \sigma_{\text{max}}
  \end{equation}
  where $\beta, n$ are work hardening parameters, and $\Epi$ is the 
  initial equivalent plastic strain.  The thermal component $\sigma_t$
  is computed using a bisection algorithm from the following equation (based 
  on the work of \cite{Hoge77})
  \begin{equation}
    \Epdot{} = \left[\frac{1}{C_1}\exp\left[\frac{2U_k}{k_b~T}
      \left(1 - \frac{\sigma_t}{\sigma_p}\right)^2\right] + 
      \frac{C_2}{\sigma_t}\right]^{-1}; \quad
    \sigma_t \le \sigma_p
  \end{equation}
  where $2 U_k$ is the energy to form a kink-pair in a dislocation segment
  of length $L_d$, $k_b$ is the Boltzmann constant, $\sigma_p$ is the Peierls
  stress. The constants $C_1, C_2$ are given by the relations
  \begin{equation}
    C_1 := \frac{\rho_d L_d a b^2 \nu}{2 w^2}; \quad
    C_2 := \frac{D}{\rho_d b^2}
  \end{equation}
  where $\rho_d$ is the dislocation density, $L_d$ is the length of a 
  dislocation segment, $a$ is the distance between Peierls valleys, 
  $b$ is the magnitude of the Burgers' vector, $\nu$ is the Debye frequency,
  $w$ is the width of a kink loop, and $D$ is the drag coefficient.

  The inputs for this model are of the form
  \begin{verbatim}
    <plasticity_model type="steinberg_cochran_guinan">
      <mu_0> 81.8e9 </mu_0>
      <sigma_0> 1.15e9 </sigma_0>
      <Y_max> 0.25e9 </Y_max>
      <beta> 2.0 </beta>
      <n> 0.50 </n>
      <A> 20.6e-12 </A>
      <B> 0.16e-3 </B>
      <T_m0> 2310.0 </T_m0>
      <Gamma_0> 3.0 </Gamma_0>
      <a> 1.67 </a>
      <epsilon_p0> 0.0 </epsilon_p0>
    </plasticity_model>
  \end{verbatim}

  \paragraph{ZA Flow Stress Model}
  The Zerilli-Armstrong (ZA) model (\cite{Zerilli87,Zerilli93,Zerilli04}) 
  is based on simplified dislocation mechanics.  The general form of the
  equation for the flow stress is
  \begin{equation}
    \sigma_y(\Ep,\Epdot{},T) = 
      \sigma_a + B\exp(-\beta(\Epdot{}) T) + 
                           B_0\sqrt{\Ep}\exp(-\alpha(\Epdot{}) T)
  \end{equation}
  where $\sigma_a$ is the athermal component of the flow stress given by
  \begin{equation}
    \sigma_a := \sigma_g + \frac{k_h}{\sqrt{l}} + K\Ep^n,
  \end{equation}
  $\sigma_g$ is the contribution due to solutes and initial dislocation
  density, $k_h$ is the microstructural stress intensity, $l$ is the 
  average grain diameter, $K$ is zero for fcc materials, 
  $B, B_0$ are material constants.  The functional forms of the exponents 
  $\alpha$ and $\beta$ are 
  \begin{equation}
    \alpha = \alpha_0 - \alpha_1 \ln(\Epdot{}); \quad
    \beta = \beta_0 - \beta_1 \ln(\Epdot{}); 
  \end{equation}
  where $\alpha_0, \alpha_1, \beta_0, \beta_1$ are material parameters that
  depend on the type of material (fcc, bcc, hcp, alloys).  The Zerilli-Armstrong
  model has been modified by \cite{Abed05} for better performance at high 
  temperatures.  However, we have not used the modified equations in our
  computations.

  The inputs for this model are of the form
  \begin{verbatim}
    <bcc_or_fcc> fcc </bcc_or_fcc>
    <c2> </c2>
    <c3> </c3>
    <c4> </c4>
    <n> </n>
  \end{verbatim}

  \paragraph{MTS Flow Stress Model}
  The Mechanical Threshold Stress (MTS) model 
  (\cite{Follans88,Goto00a,Kocks01})  
  gives the following form for the flow stress
  \begin{equation}
    \sigma_y(\Ep,\Epdot{},T) = 
      \sigma_a + (S_i \sigma_i + S_e \sigma_e)\frac{\mu(p,T)}{\mu_0} 
  \end{equation}
  where $\sigma_a$ is the athermal component of mechanical threshold stress,
  $\mu_0$ is the shear modulus at 0 K and ambient pressure, 
  $\sigma_i$ is the component of the flow stress due to intrinsic barriers 
  to thermally activated dislocation motion and dislocation-dislocation 
  interactions, $\sigma_e$ is the component of the flow stress due to 
  microstructural evolution with increasing deformation (strain hardening), 
  ($S_i, S_e$) are temperature and strain rate dependent scaling factors.  The
  scaling factors take the Arrhenius form
  \begin{align}
    S_i & = \left[1 - \left(\frac{k_b~T}{g_{0i}b^3\mu(p,T)}
    \ln\frac{\Epdot{0i}}{\Epdot{}}\right)^{1/q_i}
    \right]^{1/p_i} \\
    S_e & = \left[1 - \left(\frac{k_b~T}{g_{0e}b^3\mu(p,T)}
    \ln\frac{\Epdot{0e}}{\Epdot{}}\right)^{1/q_e}
    \right]^{1/p_e}
  \end{align}
  where $k_b$ is the Boltzmann constant, $b$ is the magnitude of the Burgers' 
  vector, ($g_{0i}, g_{0e}$) are normalized activation energies, 
  ($\Epdot{0i}, \Epdot{0e}$) are constant reference strain rates, and
  ($q_i, p_i, q_e, p_e$) are constants.  The strain hardening component
  of the mechanical threshold stress ($\sigma_e$) is given by a
  modified Voce law
  \begin{equation}\label{eq:MTSsige}
    \frac{d\sigma_e}{d\Ep} = \theta(\sigma_e)
  \end{equation}
  where
  \begin{align}
    \theta(\sigma_e) & = 
       \theta_0 [ 1 - F(\sigma_e)] + \theta_{IV} F(\sigma_e) \\
    \theta_0 & = a_0 + a_1 \ln \Epdot{} + a_2 \sqrt{\Epdot{}} - a_3 T \\
    F(\sigma_e) & = 
      \cfrac{\tanh\left(\alpha \cfrac{\sigma_e}{\sigma_{es}}\right)}
      {\tanh(\alpha)}\\
    \ln(\cfrac{\sigma_{es}}{\sigma_{0es}}) & =
    \left(\frac{kT}{g_{0es} b^3 \mu(p,T)}\right)
    \ln\left(\cfrac{\Epdot{}}{\Epdot{0es}}\right)
  \end{align}
  and $\theta_0$ is the hardening due to dislocation accumulation, 
  $\theta_{IV}$ is the contribution due to stage-IV hardening,
  ($a_0, a_1, a_2, a_3, \alpha$) are constants,
  $\sigma_{es}$ is the stress at zero strain hardening rate, 
  $\sigma_{0es}$ is the saturation threshold stress for deformation at 0 K,
  $g_{0es}$ is a constant, and $\Epdot{0es}$ is the maximum strain rate.  Note
  that the maximum strain rate is usually limited to about $10^7$/s.

  The inputs for this model are of the form
  \begin{verbatim}
    <plasticity_model type="mts_model">
      <sigma_a>363.7e6</sigma_a>
      <mu_0>28.0e9</mu_0>
      <D>4.50e9</D>
      <T_0>294</T_0>
      <koverbcubed>0.823e6</koverbcubed>
      <g_0i>0.0</g_0i>
      <g_0e>0.71</g_0e>
      <edot_0i>0.0</edot_0i>
      <edot_0e>2.79e9</edot_0e>
      <p_i>0.0</p_i>
      <q_i>0.0</q_i>
      <p_e>1.0</p_e>
      <q_e>2.0</q_e>
      <sigma_i>0.0</sigma_i>
      <a_0>211.8e6</a_0>
      <a_1>0.0</a_1>
      <a_2>0.0</a_2>
      <a_3>0.0</a_3>
      <theta_IV>0.0</theta_IV>
      <alpha>2</alpha>
      <edot_es0>3.42e8</edot_es0>
      <g_0es>0.15</g_0es>
      <sigma_es0>1679.3e6</sigma_es0>
    </plasticity_model>
  \end{verbatim}

  \paragraph{PTW Flow Stress Model}
  The Preston-Tonks-Wallace (PTW) model (\cite{Preston03}) attempts to 
  provide a model for the flow stress for extreme strain rates 
  (up to $10^{11}$/s) and temperatures up to melt.  The flow stress is
  given by
  \begin{equation}
    \sigma_y(\Ep,\Epdot{},T) = 
       \begin{cases}
         2\left[\tau_s + \alpha\ln\left[1 - \varphi
          \exp\left(-\beta-\cfrac{\theta\Ep}{\alpha\varphi}\right)\right]\right]
         \mu(p,T) & \text{thermal regime} \\
         2\tau_s\mu(p,T) & \text{shock regime}
       \end{cases}
  \end{equation}
  with 
  \begin{equation}
    \alpha := \frac{s_0 - \tau_y}{d}; \quad
    \beta := \frac{\tau_s - \tau_y}{\alpha}; \quad
    \varphi := \exp(\beta) - 1
  \end{equation}
  where $\tau_s$ is a normalized work-hardening saturation stress,
  $s_0$ is the value of $\tau_s$ at 0K,
  $\tau_y$ is a normalized yield stress, $\theta$ is the hardening constant
  in the Voce hardening law, and $d$ is a dimensionless material
  parameter that modifies the Voce hardening law.  The saturation stress
  and the yield stress are given by
  \begin{align}
    \tau_s & = \max\left\{s_0 - (s_0 - s_{\infty})
       \erf\left[\kappa
         \That\ln\left(\cfrac{\gamma\Xidot}{\Epdot{}}\right)\right],
       s_0\left(\cfrac{\Epdot{}}{\gamma\Xidot}\right)^{s_1}\right\} \\
    \tau_y & = \max\left\{y_0 - (y_0 - y_{\infty})
       \erf\left[\kappa
         \That\ln\left(\cfrac{\gamma\Xidot}{\Epdot{}}\right)\right],
       \min\left\{
         y_1\left(\cfrac{\Epdot{}}{\gamma\Xidot}\right)^{y_2}, 
         s_0\left(\cfrac{\Epdot{}}{\gamma\Xidot}\right)^{s_1}\right\}\right\} 
  \end{align}
  where $s_{\infty}$ is the value of $\tau_s$ close to the melt temperature,
  ($y_0, y_{\infty}$) are the values of $\tau_y$ at 0K and close to melt,
  respectively, $(\kappa, \gamma)$ are material constants, $\That = T/T_m$,
  ($s_1, y_1, y_2$) are material parameters for the high strain rate
  regime, and
  \begin{equation}
    \Xidot = \frac{1}{2}\left(\cfrac{4\pi\rho}{3M}\right)^{1/3}
             \left(\cfrac{\mu(p,T)}{\rho}\right)^{1/2}
  \end{equation}
  where $\rho$ is the density, and $M$ is the atomic mass.

  The inputs for this model are of the form
  \begin{verbatim}
    <plasticity_model type="preston_tonks_wallace">
      <theta> 0.025 </theta>
      <p> 2.0 </p>
      <s0> 0.0085 </s0>
      <sinf> 0.00055 </sinf>
      <kappa> 0.11 </kappa>
      <gamma> 0.00001 </gamma>
      <y0> 0.0001 </y0>
      <yinf> 0.0001 </yinf>
      <y1> 0.094 </y1>
      <y2> 0.575 </y2>
      <beta> 0.25 </beta>
      <M> 63.54 </M>
      <G0> 518e8 </G0>
      <alpha> 0.20 </alpha>
      <alphap> 0.20 </alphap>
    </plasticity_model>
  \end{verbatim}

  \subsubsection{Adiabatic Heating and Specific Heat}
  A part of the plastic work done is converted into heat and used to update the 
  temperature of a particle.  The increase in temperature ($\Delta T$) due to 
  an increment in plastic strain ($\Delta\epsilon_p$) is given by the equation
  \begin{equation}
    \Delta T = \cfrac{\chi\sigma_y}{\rho C_p} \Delta \epsilon_p
  \end{equation}
  where $\chi$ is the Taylor-Quinney coefficient, and $C_p$ is the specific
  heat.  The value of the Taylor-Quinney coefficient is taken to be 0.9
  in all our simulations (see \cite{Ravi01} for more details on the 
  variation of $\chi$ with strain and strain rate).

  The Taylor-Quinney coefficient is taken as input in the ElasticPlastic model
  using the tags
  \begin{verbatim}
    <taylor_quinney_coeff> 0.9 </taylor_quinney_coeff>
  \end{verbatim}

  \paragraph{Default specific heat model}
  The default model returns a constant specific heat and is invoked using
  \begin{verbatim}
    <specific_heat_model type="constant_Cp">
    </specific_heat_model>
  \end{verbatim}

  \paragraph{Specific heat model for copper}
  The specific heat model for copper is of the form
  \Beq
    Cp = 
    \begin{cases}
      A_0~T^3 - B_0~T^2 + C_0~T - D_0 & \text{if} ~~T < T_0 \\
      A_1~T + B_1 & \text{if} ~~T \ge T_0 ~.
    \end{cases}
  \Eeq
  The model is invoked using
  \begin{verbatim}
  <specific_heat_model type = "copper_Cp"> </specific_heat_model>
  \end{verbatim}

  \paragraph{Specific heat model for steel}
  A relation for the dependence of $C_p$ upon temperature is
  used for the steel (\cite{Lederman74}).
  \begin{align}
    C_p & = \begin{cases}
            A_1 + B_1~t + C_1~|t|^{-\alpha} & \text{if}~~ T < T_c \\
            A_2 + B_2~t + C_2~t^{-\alpha^{'}} & \text{if}~~ T > T_c 
          \end{cases} \label{eq:CpSteel}\\
    t & = \cfrac{T}{T_c} - 1 
  \end{align}
  where $T_c$ is the critical temperature at which the phase transformation
  from the $\alpha$ to the $\gamma$ phase takes place, and $A_1, A_2, B_1, B_2,
  \alpha, \alpha^{'}$ are constants.

  The model is invoked using
  \begin{verbatim}
  <specific_heat_model type = "steel_Cp"> </specific_heat_model>
  \end{verbatim}

  The heat generated at a material point is conducted away at the end of a 
  time step using the transient heat equation.  The effect of conduction on
  material point temperature is negligible (but non-zero) for the high 
  strain-rate problems simulated using Uintah.

\subsection{Adding new models}
  In the parallel implementation of the stress update algorithm, sockets have 
  been added to allow for the incorporation of a variety of plasticity, damage, 
  yield, and bifurcation models without requiring any change in the stress 
  update code.  The algorithm is shown in Algorithm~\ref{algo1}.  The
  equation of state, plasticity model, yield condition, damage model, and
  the stability criterion are all polymorphic objects created using a 
  factory idiom in C++~(\cite{Coplien92}).
  \begin{table}[p]
    \caption{Stress Update Algorithm} \label{algo1}
    \vspace{12pt}
    \begin{tabbing}
    \quad \=\quad \=\quad \=\quad \=\quad \kill
    {\bf Persistent}:Initial moduli, temperature, porosity, \\
      \>\>        scalar damage, equation of state, plasticity model, \\
      \>\>        yield condition, stability criterion, damage model\\
    {\bf Temporary}:Particle state at time $t$ \\
    {\bf Output:} Particle state at time $t+\Delta t$\\ \\

    {\bf For} {\it all the patches in the domain}\\
      \> Read the particle data and initialize updated data storage\\
      \> {\bf For} {\it all the particles in the patch}\\ 
      \>\>   Compute the velocity gradient and the rate of deformation tensor\\ 
      \>\>   Compute the deformation gradient and the rotation tensor\\
      \>\>   Rotate the Cauchy stress and the rate of deformation tensor \\
      \>\>\> to the material configuration\\
      \>\>   Compute the current shear modulus and melting temperature\\
      \>\>   Compute the pressure using the equation of state,  \\
      \>\>\>  update the hydrostatic stress, and  \\
      \>\>\>  compute the trial deviatoric stress\\
      \>\>   Compute the flow stress using the plasticity model\\
      \>\>   Evaluate the yield function\\
      \>\>   {\bf If} {\it particle is elastic} \\
      \>\>\>     Update the elastic deviatoric stress from the trial stress\\
      \>\>\>     Rotate the stress back to laboratory coordinates\\
      \>\>\>     Update the particle state\\
      \>\>   {\bf Else} \\
      \>\>\>     Compute the elastic-plastic deviatoric stress\\
      \>\>\>     Compute updated porosity, scalar damage, and \\
      \>\>\>\>       temperature increase due to plastic work\\
      \>\>\>     Compute elastic-plastic tangent modulus and 
                     evaluate stability condition\\
      \>\>\>     Rotate the stress back to laboratory coordinates\\
      \>\>\>     Update the particle state\\
      \>\>   {\bf End If} \\
      \>\>  {\bf If} 
             {\it Temperature $>$ Melt Temperature} or 
             {\it Porosity $>$ Critical Porosity} or
             {\it Unstable}\\
      \>\>\>       Tag particle as failed\\
      \>\>  {\bf End If} \\
      \>\> Convert failed particles into a material with a different 
           velocity field \\
      \> {\bf End For} \\
    {\bf End For}
    \end{tabbing}
  \end{table}
      
Addition of a new model requires the following steps (the example below is only
for the flow stress model but the same idea applies to other models) :
\begin{enumerate}
    \item Creation of a new class that encapsulates the plasticity 
    model.  The template for this class can be copied from the
    existing plasticity models.  The data that is unique to 
    the new model are specified in the form of 
    \begin{itemize}
      \item A structure containing the constants for the plasticity
            model.
      \item Particle variables that specify the variables that 
            evolve in the plasticity model.
    \end{itemize}
    \item The implementation of the plasticity model involves the
    following steps.
    \begin{itemize}
      \item Reading the input file for the model constants in the
            constructor.
      \item Adding the variables that evolve in the plasticity model
            appropriately to the task graph.
      \item Adding the appropriate flow stress calculation method.
    \end{itemize}
    \item The \verb+PlasticityModelFactory+ is then modified so that
          it recognizes the added plasticity model.
\end{enumerate}

\subsection{Damage Models and Failure}
Only the Johnson-Cook damage evolution rule has been added to the 
DamageModelFactory so far.  The damage model framework is designed 
to be similar to the plasticity model framework.  New models can
be added using the approach described in the previous section.

  A particle is tagged as ``failed'' when its temperature is greater than the
  melting point of the material at the applied pressure.  An additional
  condition for failure is when the porosity of a particle increases beyond a
  critical limit and the strain exceeds the fracture strain of the material.  
  Another condition for failure is when a material bifurcation 
  condition such as the Drucker stability postulate is satisfied.  Upon failure,
  a particle is either removed from the computation by setting the stress to
  zero or is converted into a material with a different velocity field 
  which interacts with the remaining particles via contact.  Either approach
  leads to the simulation of a newly created surface.  More details of the
  approach can be found in ~\cite{Banerjee04a,Banerjee04c,Banerjee05}.

  \subsubsection{Yield conditions}
  When failure is to be simulated we can use the Gurson-Tvergaard-Needleman
  yield condition instead of the von Mises condition.

  \paragraph{The von Mises yield condition}
  The von Mises yield condition is the default and is invoked using the tags
  \begin{verbatim}
    <yield_condition type="vonMises">
    </yield_condition>
  \end{verbatim}

  \paragraph{The Gurson-Tvergaard-Needleman (GTN) yield condition}
  The Gurson-Tvergaard-Needleman (GTN) yield 
  condition~\cite{Gurson77,Tver84} depends on porosity.  An associated 
  flow rule is used to determine the plastic rate parameter in either case.
  The GTN yield condition can be written as
  \begin{equation}
    \Phi = \left(\frac{\sigma_{eq}}{\sigma_f}\right)^2 +
    2 q_1 f_* \cosh \left(q_2 \frac{Tr(\sigma)}{2\sigma_f}\right) -
    (1+q_3 f_*^2) = 0
  \end{equation}
  where $q_1,q_2,q_3$ are material constants and $f_*$ is the porosity 
  (damage) function given by
  \begin{equation}
    f* = 
    \begin{cases}
      f & \text{for}~~ f \le f_c,\\ 
      f_c + k (f - f_c) & \text{for}~~ f > f_c 
    \end{cases}
  \end{equation}
  where $k$ is a constant and $f$ is the porosity (void volume fraction).  The 
  flow stress in the matrix material is computed using either of the two 
  plasticity models discussed earlier.  Note that the flow stress in the matrix 
  material also remains on the undamaged matrix yield surface and uses an 
  associated flow rule.

  This yield condition is invoked using
  \begin{verbatim}
    <yield_condition type="gurson">
      <q1> 1.5 </q1>
      <q2> 1.0 </q2>
      <q3> 2.25 </q3>
      <k> 4.0 </k>
      <f_c> 0.05 </f_c>
    </yield_condition>
  \end{verbatim}

  \subsubsection{Porosity model}
  The evolution of porosity is calculated as the sum of the rate of growth 
  and the rate of nucleation~\cite{Ramaswamy98a}.  The rate of growth of
  porosity and the void nucleation rate are given by the following equations
  ~\cite{Chu80}
  \begin{align}
    \dot{f} &= \dot{f}_{\text{nucl}} + \dot{f}_{\text{grow}} \\
    \dot{f}_{\text{grow}} & = (1-f) \text{Tr}(\BD_p) \\
    \dot{f}_{\text{nucl}} & = \cfrac{f_n}{(s_n \sqrt{2\pi})}
            \exp\left[-\Half \cfrac{(\epsilon_p - \epsilon_n)^2}{s_n^2}\right]
            \dot{\epsilon}_p
  \end{align}
  where $\BD_p$ is the rate of plastic deformation tensor, $f_n$ is the volume 
  fraction of void nucleating particles , $\epsilon_n$ is the mean of the 
  distribution of nucleation strains, and $s_n$ is the standard 
  deviation of the distribution.

  The inputs tags for porosity are of the form
  \begin{verbatim}
    <evolve_porosity> true </evolve_porosity>
    <initial_mean_porosity>         0.005 </initial_mean_porosity>
    <initial_std_porosity>          0.001 </initial_std_porosity>
    <critical_porosity>             0.3   </critical_porosity>
    <frac_nucleation>               0.1   </frac_nucleation>
    <meanstrain_nucleation>         0.3   </meanstrain_nucleation>
    <stddevstrain_nucleation>       0.1   </stddevstrain_nucleation>
    <initial_porosity_distrib>      gauss </initial_porosity_distrib>
  \end{verbatim}

  \subsubsection{Damage model}
  After the stress state has been determined on the basis of the yield condition
  and the associated flow rule, a scalar damage state in each material point can
  be calculated using the Johnson-Cook model ~\cite{Johnson85}.  
  The Johnson-Cook model has an explicit dependence on temperature, plastic
  strain, ans strain rate.

  The damage evolution rule for the Johnson-Cook damage model can be written as
  \begin{equation}
    \dot{D} = \cfrac{\dot{\epsilon_p}}{\epsilon_p^f} ~;~~
    \epsilon_p^f = 
      \left[D_1 + D_2 \exp \left(\cfrac{D_3}{3} \sigma^*\right)\right]
      \left[1+ D_4 \ln(\dot{\epsilon_p}^*)\right]
      \left[1+D_5 T^*\right]~;~~
    \sigma^*= \cfrac{\text{Tr}(\Bsig)}{\sigma_{eq}}~;~~
  \end{equation}
  where $D$ is the damage variable which has a value of 0 for virgin material
  and a value of 1 at fracture, $\epsilon_p^f$ is the fracture strain, 
  $D_1, D_2, D_3, D_4, D_5$ are constants, $\Bsig$ is the Cauchy stress, and
  $T^*$ is the scaled temperature as in the Johnson-Cook plasticity model.

  The input tags for the damage model are :
  \begin{verbatim}
    <damage_model type="johnson_cook">
      <D1>0.05</D1>
      <D2>3.44</D2>
      <D3>-2.12</D3>
      <D4>0.002</D4>
      <D5>0.61</D5>
    </damage_model>
  \end{verbatim}

  An initial damage distribution can be created using the following tags
  \begin{verbatim}
    <evolve_damage>                 true  </evolve_damage>
    <initial_mean_scalar_damage>    0.005  </initial_mean_scalar_damage>
    <initial_std_scalar_damage>     0.001 </initial_std_scalar_damage>
    <critical_scalar_damage>        1.0   </critical_scalar_damage>
    <initial_scalar_damage_distrib> gauss </initial_scalar_damage_distrib>
  \end{verbatim}

  \subsection{Erosion algorithm}
  Under normal conditions, the heat generated at a material point is conducted 
  away at the end of a time step using the heat equation.  If special adiabatic 
  conditions apply (such as in impact problems), the heat is accumulated at a 
  material point and is not conducted to the surrounding particles.  This 
  localized heating can be used to determine whether a material point has
  melted.

  The determination of whether a particle has failed can be made on the 
  basis of either or all of the following conditions:
  \begin{itemize}
    \item The particle temperature exceeds the melting temperature.
    \item The TEPLA-F fracture condition~\cite{Johnson88} is satisfied.
       This condition can be written as
       \begin{equation}
         (f/f_c)^2 + (\epsilon_p/\epsilon_p^f)^2 = 1
       \end{equation}
       where $f$ is the current porosity, $f_c$ is the maximum 
       allowable porosity, $\epsilon_p$ is the current plastic strain, and
       $\epsilon_p^f$ is the plastic strain at fracture.
    \item An alternative to ad-hoc damage criteria is to use the concept of 
       bifurcation to determine whether a particle has failed or not.  Two
       stability criteria have been explored in this paper - the Drucker
       stability postulate~\cite{Drucker59} and the loss of hyperbolicity
       criterion (using the determinant of the acoustic tensor)
       \cite{Rudnicki75,Perzyna98}.  
  \end{itemize}

  The simplest criterion that can be used is the Drucker stability postulate 
  \cite{Drucker59} which states that time rate of change of the rate of 
  work done by a material cannot be negative.  Therefore, the material is 
  assumed to become unstable (and a particle fails) when
  \begin{equation}
    \dot\Bsig:\BD^p \le 0
  \end{equation}

  Another stability criterion that is less restrictive is the acoustic
  tensor criterion which states that the material loses stability if the 
  determinant of the acoustic tensor changes sign~\cite{Rudnicki75,Perzyna98}.  
  Determination of the acoustic tensor requires a search for a normal vector 
  around the material point and is therefore computationally expensive.  A 
  simplification of this criterion is a check which assumes that the direction 
  of instability lies in the plane of the maximum and minimum principal 
  stress~\cite{Becker02}.  In this approach, we assume that the strain is 
  localized in a band with normal $\Bn$, and the magnitude of the velocity 
  difference across the band is $\Bg$.  Then the bifurcation condition 
  leads to the relation 
  \begin{equation} 
    R_{ij} g_{j} = 0 ~;~~~
    R_{ij} = M_{ikjl} n_k n_l + M_{ilkj} n_k n_l - \sigma_{ik} n_j n_k
  \end{equation} 
  where $M_{ijkl}$ are the components of the co-rotational tangent
  modulus tensor and $\sigma_{ij}$ are the components of the co-rotational 
  stress tensor.  If $\det(R_{ij}) \le 0 $, then $g_j$ can be arbitrary and 
  there is a possibility of strain localization.  If this condition for 
  loss of hyperbolicity is met,  then a particle deforms in an unstable 
  manner and failure can be assumed to have occurred at that particle.  
  We use a combination of these criteria to simulate failure.

  Since the material in the container may unload locally after fracture, the 
  hypoelastic-plastic stress update may not work accurately under certain 
  circumstances.  An improvement would be to use a hyperelastic-plastic stress 
  update algorithm.  Also, the plasticity models are temperature dependent.
  Hence there is the issue of severe mesh dependence due to change of the
  governing equations from hyperbolic to elliptic in the softening regime
  ~\cite{Hill75,Bazant85,Tver90}.  Viscoplastic stress update models or 
  nonlocal/gradient plasticity models~\cite{Ramaswamy98,Hao00} can be used 
  to eliminate some of these effects and are currently under investigation. 

  The tags used to control the erosion algorithm are in two places.  
  In the \verb|<MPM> </MPM>| section the following flags can be set
  \begin{verbatim}
    <erosion algorithm = "ZeroStress"/>
    <create_new_particles>           false      </create_new_particles>
    <manual_new_material>            false      </manual_new_material>
  \end{verbatim}
  If the erosion algorithm is \verb|"none"| then no particle failure is done.

  In the \verb|<constitutive_model type="elastic_plastic">| section, the 
  following flags can be set
  \begin{verbatim}
    <evolve_porosity>               true  </evolve_porosity>
    <evolve_damage>                 true  </evolve_damage>
    <do_melting>                    true  </do_melting>
    <useModifiedEOS>                true  </useModifiedEOS>
    <check_TEPLA_failure_criterion> true  </check_TEPLA_failure_criterion>
    <check_max_stress_failure>      false </check_max_stress_failure>
    <critical_stress>              12.0e9 </critical_stress>
  \end{verbatim}

\subsection{Implementation}
The elastic response is assumed to be isotropic.  The material
constants that are taken as input for the elastic response are the
bulk and shear modulus.  The flow rule is determined from the input
and the appropriate plasticity model is created using the 
\verb+PlasticityModelFactory+ class.  The damage evolution rule
is determined from the input and a damage model is created using
the \verb+DamageModelFactory+ class.  The equation of state 
that is used to determine the pressure is also determined from the
input.  The equation of state model is created using the 
\verb+MPMEquationOfStateFactory+ class.

In addition, a damage evolution variable ($D$) is stored at each time 
step (this need not be the case and will be transfered to the 
damage models in the future).  The left stretch and rotation are 
updated incrementally at each
time step (instead of performing a polar decomposition) and the 
rotation tensor is used to rotate the Cauchy stress and rate of deformation
to the material coordinates at each time step (instead of using a 
objective stress rate formulation).  

Any evolution variables for the plasticity model, damage model or the
equation of state are specified in the class that encapsulates the 
particular model.  

The flow stress is calculated from the plasticity model using a 
function call of the form
\begin{verbatim}
    double flowStress = d_plasticity->computeFlowStress(tensorEta, tensorS, 
                                                        pTemperature[idx],
                                                        delT, d_tol, matl, idx);
\end{verbatim}
A number of plasticity models can be evaluated using the inputs in the
\verb+computeFlowStress+ call.  The variable \verb+d_plasticity+ is
polymorphic and can represent any of the plasticity models that can be
created by the plasticity model factory.  The plastic evolution variables
are updated using a polymorphic function along the lines of
\verb+computeFlowStress+.

The equation of state is used to calculate the hydrostatic stress using
a function call of the form
\begin{verbatim}
    Matrix3 tensorHy = d_eos->computePressure(matl, bulk, shear, 
                                              tensorF_new, tensorD, 
                                              tensorP, pTemperature[idx], 
                                              rho_cur, delT);
\end{verbatim}

Similarly, the damage model is called using a function of the type
\begin{verbatim}
    double damage = d_damage->computeScalarDamage(tensorEta, tensorS, 
                                                  pTemperature[idx],
                                                  delT, matl, d_tol, 
                                                  pDamage[idx]);
\end{verbatim}

Therefore, the plasticity, damage and equation of state models are 
easily be inserted into any other type of stress update algorithm 
without any change being needed in them as can be seen in the 
hyperelastic-plastic stress update algorithm discussed below.

  \section{Example Input Files}
  \subsection{Hypoelastic-plastic model}
  An example of the portion of an input file that specifies a copper body
  with a hypoelastic stress update, Johnson-Cook plasticity model,
  Johnson-Cook Damage Model and Mie-Gruneisen Equation of State is shown 
  below.
  \begin{verbatim}
  <material>

    <include href="inputs/MPM/MaterialData/MaterialConstAnnCopper.xml"/>
    <constitutive_model type="hypoelastic_plastic">
      <tolerance>5.0e-10</tolerance>
      <include href="inputs/MPM/MaterialData/IsotropicElasticAnnCopper.xml"/>
      <include href="inputs/MPM/MaterialData/JohnsonCookPlasticAnnCopper.xml"/>
      <include href="inputs/MPM/MaterialData/JohnsonCookDamageAnnCopper.xml"/>
      <include href="inputs/MPM/MaterialData/MieGruneisenEOSAnnCopper.xml"/>
    </constitutive_model>

    <burn type = "null" />
    <velocity_field>1</velocity_field>

    <geom_object>
      <cylinder label = "Cylinder">
        <bottom>[0.0,0.0,0.0]</bottom>
        <top>[0.0,2.54e-2,0.0]</top>
        <radius>0.762e-2</radius>
      </cylinder>
      <res>[3,3,3]</res>
      <velocity>[0.0,-208.0,0.0]</velocity>
      <temperature>294</temperature>
    </geom_object>

  </material>
  \end{verbatim}

  The general material constants for copper are in the file 
  \verb+MaterialConstAnnCopper.xml+.  The contents are shown below
  \begin{verbatim}
  <?xml version='1.0' encoding='ISO-8859-1' ?>
  <Uintah_Include>
    <density>8930.0</density>
    <toughness>10.e6</toughness>
    <thermal_conductivity>1.0</thermal_conductivity>
    <specific_heat>383</specific_heat>
    <room_temp>294.0</room_temp>
    <melt_temp>1356.0</melt_temp>
  </Uintah_Include>
  \end{verbatim}

  The elastic properties are in the file \verb+IsotropicElasticAnnCopper.xml+.
  The contents of this file are shown below.
  \begin{verbatim}
  <?xml version='1.0' encoding='ISO-8859-1' ?>
  <Uintah_Include>
    <shear_modulus>45.45e9</shear_modulus>
    <bulk_modulus>136.35e9</bulk_modulus>
  </Uintah_Include>
  \end{verbatim}
  
  The constants for the Johnson-Cook plasticity model are in the file
  \verb+JohnsonCookPlasticAnnCopper.xml+.  The contents of this file are
  shown below.
  \begin{verbatim}
  <?xml version='1.0' encoding='ISO-8859-1' ?>
  <Uintah_Include>
    <plasticity_model type="johnson_cook">
      <A>89.6e6</A>
      <B>292.0e6</B>
      <C>0.025</C>
      <n>0.31</n>
      <m>1.09</m>
    </plasticity_model>
  </Uintah_Include>
  \end{verbatim}

  The constants for the Johnson-Cook damage model are in the file
  \verb+JohnsonCookDamageAnnCopper.xml+.  The contents of this file are
  shown below.
  \begin{verbatim}
  <?xml version='1.0' encoding='ISO-8859-1' ?>
  <Uintah_Include>
    <damage_model type="johnson_cook">
      <D1>0.54</D1>
      <D2>4.89</D2>
      <D3>-3.03</D3>
      <D4>0.014</D4>
      <D5>1.12</D5>
    </damage_model>
  </Uintah_Include>
  \end{verbatim}

  The constants for the Mie-Gruneisen model (as implemented in the 
  Uintah Computational Framework) are in the file
  \verb+MieGruneisenEOSAnnCopper.xml+.  The contents of this file are
  shown below.
  \begin{verbatim}
  <?xml version='1.0' encoding='ISO-8859-1' ?>
  <Uintah_Include>
    <equation_of_state type="mie_gruneisen">
      <C_0>3940</C_0>
      <Gamma_0>2.02</Gamma_0>
      <S_alpha>1.489</S_alpha>
    </equation_of_state>
  </Uintah_Include>
  \end{verbatim}

  As can be seen from the input file, any other plasticity model, damage
  model and equation of state can be used to replace the Johnson-Cook
  and Mie-Gruneisen models without any extra effort (provided the models
  have been implemented and the data exist).

  The material data can easily be taken from a material database or specified
  for a new material in an input file kept at a centralized location.  At this
  stage material data for a range of materials is kept in the directory
  \verb|.../Uintah/StandAlone/inputs/MPM/MaterialData|.

  \subsection{Elastic-plastic model}
  The \verb|<constitutive_model type="elastic_plastic">| model is more stable
  (and also more general)
  than the \verb|<constitutive_model type="hypoelastic_plastic">| model.  A
  sample input file for this model is shown below.
  \begin{verbatim}
  <MPM>
    <do_grid_reset> false </do_grid_reset>
    <time_integrator>explicit</time_integrator>
    <boundary_traction_faces>[zminus,zplus]</boundary_traction_faces> 
    <dynamic>true</dynamic>
    <solver>simple</solver>
    <convergence_criteria_disp>1.e-10</convergence_criteria_disp>
    <convergence_criteria_energy>4.e-10</convergence_criteria_energy>
    <DoImplicitHeatConduction>true</DoImplicitHeatConduction>
    <interpolator>linear</interpolator>
    <minimum_particle_mass> 1.0e-8</minimum_particle_mass>
    <maximum_particle_velocity> 1.0e8</maximum_particle_velocity>
    <artificial_damping_coeff> 0.0 </artificial_damping_coeff>
    <artificial_viscosity> true </artificial_viscosity>
    <accumulate_strain_energy> true </accumulate_strain_energy>
    <use_load_curves> false </use_load_curves>
    <turn_on_adiabatic_heating>   false    </turn_on_adiabatic_heating>
    <do_contact_friction_heating> false    </do_contact_friction_heating>
    <create_new_particles>        false    </create_new_particles>
    <erosion algorithm = "none"/>
  </MPM>

  <MaterialProperties>
    <MPM>
      <material name = "OFHCCu">
        <density>              8930.0 </density>
        <thermal_conductivity> 386.0  </thermal_conductivity>
        <specific_heat>        414.0  </specific_heat>
        <room_temp>            294.0  </room_temp>
        <melt_temp>            1356.0 </melt_temp>
        <constitutive_model type="elastic_plastic">
          <isothermal>                    false   </isothermal>
          <tolerance>                     1.0e-12 </tolerance>
          <do_melting>                    false   </do_melting>
          <evolve_porosity>               false   </evolve_porosity>
          <evolve_damage>                 false   </evolve_damage>
          <check_TEPLA_failure_criterion> false   </check_TEPLA_failure_criterion>
          <check_max_stress_failure>      false   </check_max_stress_failure>
          <initial_material_temperature>  696.0   </initial_material_temperature>
          
          <shear_modulus>           46.0e9  </shear_modulus>
          <bulk_modulus>            129.0e9 </bulk_modulus>
          <coeff_thermal_expansion> 1.76e-5 </coeff_thermal_expansion>
          <taylor_quinney_coeff>    0.9     </taylor_quinney_coeff>
          <critical_stress>         129.0e9 </critical_stress>

          <equation_of_state type = "mie_gruneisen">
            <C_0>     3940  </C_0>
            <Gamma_0> 2.02  </Gamma_0>
            <S_alpha> 1.489 </S_alpha>
          </equation_of_state>
          
          <plasticity_model type="mts_model">
            <sigma_a>40.0e6</sigma_a>
            <mu_0>47.7e9</mu_0>
            <D>3.0e9</D>
            <T_0>180</T_0>
            <koverbcubed>0.823e6</koverbcubed>
            <g_0i>0.0</g_0i>
            <g_0e>1.6</g_0e>
            <edot_0i>0.0</edot_0i>
            <edot_0e>1.0e7</edot_0e>
            <p_i>0.0</p_i>
            <q_i>0.0</q_i>
            <p_e>0.666667</p_e>
            <q_e>1.0</q_e>
            <sigma_i>0.0</sigma_i>
            <a_0>2390.0e6</a_0>
            <a_1>12.0e6</a_1>
            <a_2>1.696e6</a_2>
            <a_3>0.0</a_3>
            <theta_IV>0.0</theta_IV>
            <alpha>2</alpha>
            <edot_es0>1.0e7</edot_es0>
            <g_0es>0.2625</g_0es>
            <sigma_es0>770.0e6</sigma_es0>
          </plasticity_model>

          <shear_modulus_model type="mts_shear">
            <mu_0>47.7e9</mu_0>
            <D>3.0e9</D>
            <T_0>180</T_0>
          </shear_modulus_model>

          <melting_temp_model type = "constant_Tm">
          </melting_temp_model>

          <yield_condition type = "vonMises">
          </yield_condition>

          <stability_check type = "none">
          </stability_check>

          <damage_model type = "hancock_mackenzie">
            <D0> 0.0001 </D0>
            <Dc> 0.7    </Dc>
          </damage_model>
          
          <compute_specfic_heat> false </compute_specfic_heat>
          <specific_heat_model type="constant_Cp">
          </specific_heat_model>
          
        </constitutive_model>
        <geom_object>
          <box label = "box">
            <min>[0.0,    0.0,    0.0]</min>
            <max>[1.0e-2, 1.0e-2, 1.0e-2]</max>
          </box>
          <res>[1,1,1]</res>
          <velocity>[0.0, 0.0, 0.0]</velocity>
          <temperature>696</temperature>
        </geom_object>
      </material>

    </MPM>
  </MaterialProperties>
  \end{verbatim}

  \subsubsection{An exploding ring experiment}
  The follwing shows the complete input file for an expanding ring test.
  \begin{verbatim}

<?xml version='1.0' encoding='ISO-8859-1' ?>
<Uintah_specification> 
<!--Please use a consistent set of units, (mks, cgs,...)-->
    <!-- First crack at the tuna can problem -->

   <Meta>
       <title>Pressurization of a container via burning w/o fracture</title>
   </Meta>&gt;    
   <SimulationComponent>
     <type> mpmice </type>
   </SimulationComponent>
    <!--____________________________________________________________________-->
    <!--   T  I  M  E     V  A  R  I  A  B  L  E  S                         -->
    <!--____________________________________________________________________-->
   <Time>
       <max_Timesteps>   99999          </max_Timesteps>>
       <maxTime>             2.00e-2    </maxTime>
       <initTime>            0.0        </initTime>
       <delt_min>            1.0e-12    </delt_min>
       <delt_max>            1.0        </delt_max>
       <delt_init>           2.1e-8     </delt_init>
       <timestep_multiplier> 0.5        </timestep_multiplier>
   </Time>    
    <!--____________________________________________________________________-->
    <!--   G  R  I  D     V  A  R  I  A  B  L  E  S                         -->
    <!--____________________________________________________________________-->
    <Grid>
    <BoundaryConditions>
      <Face side = "x-">
        <BCType id = "all" label = "Symmetric" var = "symmetry">
        </BCType>
      </Face>
      <Face side = "x+">
        <BCType id = "0" label = "Pressure" var = "Neumann">
                              <value>  0.0 </value>
        </BCType>
        <BCType id = "all" label = "Velocity" var = "Dirichlet">
                              <value> [0.,0.,0.] </value>
        </BCType>
        <BCType id = "all" label = "Temperature" var = "Neumann">
                              <value> 0.0  </value>
        </BCType>
        <BCType id = "all" label = "Density" var = "Neumann">
                              <value> 0.0  </value>
        </BCType>
      </Face>
      <Face side = "y-">
        <BCType id = "all" label = "Symmetric" var = "symmetry">
        </BCType>
      </Face>                  
      <Face side = "y+">
        <BCType id = "0" label = "Pressure" var = "Neumann">
                              <value> 0.0   </value>
        </BCType>
        <BCType id = "all" label = "Velocity" var = "Dirichlet">
                              <value> [0.,0.,0.] </value>
        </BCType>
        <BCType id = "all" label = "Temperature" var = "Neumann">
                              <value> 0.0  </value>
        </BCType>
        <BCType id = "all" label = "Density" var = "Neumann">
                              <value> 0.0  </value>
        </BCType>
      </Face>
      <Face side = "z-">
        <BCType id = "all" label = "Symmetric" var = "symmetry">
        </BCType>
      </Face>                  
      <Face side = "z+">
        <BCType id = "all" label = "Symmetric" var = "symmetry">
        </BCType>
      </Face>
    </BoundaryConditions>
       <Level>
           <Box label = "1">
              <lower>        [ -0.08636, -0.08636, -0.0016933] </lower>
              <upper>        [ 0.08636, 0.08636, 0.0016933]    </upper>
              <extraCells>   [1,1,1]    </extraCells>
              <patches>      [2,2,1]    </patches>
              <resolution>   [102, 102, 1]                 </resolution>
           </Box>
       </Level>
    </Grid>
   
    <!--____________________________________________________________________-->
    <!--   O  U  P  U  T     V  A  R  I  A  B  L  E  S                      -->
    <!--____________________________________________________________________-->
   <DataArchiver>
      <filebase>explodeRFull.uda</filebase>
      <outputTimestepInterval> 20 </outputTimestepInterval>
      <save label = "rho_CC"/>
      <save label = "press_CC"/>
      <save label = "temp_CC"/>
      <save label = "vol_frac_CC"/>
      <save label = "vel_CC"/>
      <save label = "g.mass"/>
      <save label = "p.x"/>
      <save label = "p.mass"/>
      <save label = "p.temperature"/>
      <save label = "p.porosity"/>
      <save label = "p.particleID"/>
      <save label = "p.velocity"/>
      <save label = "p.stress"/>
      <save label = "p.damage" material = "0"/>
      <save label = "p.plasticStrain" material = "0"/>
      <save label = "p.strainRate" material = "0"/>
      <save label = "g.stressFS"/>
      <save label = "delP_Dilatate"/>
      <save label = "delP_MassX"/>
      <save label = "p.localized"/>
      <checkpoint cycle = "2" timestepInterval = "20"/>
   </DataArchiver>

   <Debug>
   </Debug>
    <!--____________________________________________________________________-->
    <!--    I  C  E     P  A  R  A  M  E  T  E  R  S                        -->
    <!--____________________________________________________________________-->
    <CFD>
       <cfl>0.5</cfl>
       <CanAddICEMaterial>true</CanAddICEMaterial>
       <ICE>
        <advection type = "SecondOrder"/>
        <ClampSpecificVolume>true</ClampSpecificVolume>
      </ICE>        
    </CFD>

    <!--____________________________________________________________________-->
    <!--     P  H  Y  S  I  C  A  L     C  O  N  S  T  A  N  T  S           -->
    <!--____________________________________________________________________-->   
    <PhysicalConstants>
       <gravity>            [0,0,0]   </gravity>
       <reference_pressure> 101325.0  </reference_pressure>
    </PhysicalConstants>

    <MPM>
      <time_integrator>                explicit   </time_integrator>
      <nodes8or27>                     27         </nodes8or27>
      <minimum_particle_mass>          3.e-12     </minimum_particle_mass>
      <maximum_particle_velocity>      1.e3       </maximum_particle_velocity>
      <artificial_damping_coeff>       0.0        </artificial_damping_coeff>
      <artificial_viscosity>           true       </artificial_viscosity>
      <artificial_viscosity_coeff1>    0.07       </artificial_viscosity_coeff1>
      <artificial_viscosity_coeff2>    1.6        </artificial_viscosity_coeff2>
      <turn_on_adiabatic_heating>      false      </turn_on_adiabatic_heating>
      <accumulate_strain_energy>       false      </accumulate_strain_energy>
      <use_load_curves>                false      </use_load_curves>
      <create_new_particles>           false      </create_new_particles>
      <manual_new_material>            false      </manual_new_material>
      <DoThermalExpansion>             false      </DoThermalExpansion>
      <testForNegTemps_mpm>            false      </testForNegTemps_mpm>
      <erosion algorithm = "ZeroStress"/>
    </MPM>

    <!--____________________________________________________________________-->
    <!--    MATERIAL PROPERTIES INITIAL CONDITIONS                          -->
    <!--____________________________________________________________________-->
    <MaterialProperties>
       <MPM>
         <material name = "Steel Ring">
           <include href="inputs/MPM/MaterialData/MatConst4340St.xml"/>
           <constitutive_model type="elastic_plastic">
             <isothermal>                    false </isothermal>
             <tolerance>                  1.0e-10  </tolerance>
             <evolve_porosity>               true  </evolve_porosity>
             <evolve_damage>                 true  </evolve_damage>
             <compute_specific_heat>         true  </compute_specific_heat>
             <do_melting>                    true  </do_melting>
             <useModifiedEOS>                true  </useModifiedEOS>
             <check_TEPLA_failure_criterion> true  </check_TEPLA_failure_criterion>
             <initial_material_temperature>  600.0 </initial_material_temperature>
             <taylor_quinney_coeff>          0.9   </taylor_quinney_coeff>
             <check_max_stress_failure>      false </check_max_stress_failure>
             <critical_stress>              12.0e9 </critical_stress>

             <!-- Warning: you must copy link this input file into your -->
             <!-- sus directory or these paths won't work.              -->

             <include href="inputs/MPM/MaterialData/IsoElastic4340St.xml"/>
             <include href="inputs/MPM/MaterialData/MieGrunEOS4340St.xml"/>
             <include href="inputs/MPM/MaterialData/ConstantShear.xml"/>
             <include href="inputs/MPM/MaterialData/ConstantTm.xml"/>
             <include href="inputs/MPM/MaterialData/JCPlastic4340St.xml"/>
             <include href="inputs/MPM/MaterialData/VonMisesYield.xml"/>
             <include href="inputs/MPM/MaterialData/DruckerBeckerStabilityCheck.xml"/>
             <include href="inputs/MPM/MaterialData/JCDamage4340St.xml"/>
             <specific_heat_model type="steel_Cp"> </specific_heat_model>

             <initial_mean_porosity>         0.005 </initial_mean_porosity>
             <initial_std_porosity>          0.001 </initial_std_porosity>
             <critical_porosity>             0.3   </critical_porosity>
             <frac_nucleation>               0.1   </frac_nucleation>
             <meanstrain_nucleation>         0.3   </meanstrain_nucleation>
             <stddevstrain_nucleation>       0.1   </stddevstrain_nucleation>
             <initial_porosity_distrib>      gauss </initial_porosity_distrib>

             <initial_mean_scalar_damage>    0.005  </initial_mean_scalar_damage>
             <initial_std_scalar_damage>     0.001 </initial_std_scalar_damage>
             <critical_scalar_damage>        1.0   </critical_scalar_damage>
             <initial_scalar_damage_distrib> gauss </initial_scalar_damage_distrib>
           </constitutive_model>                                      
       
                <geom_object>
                   <difference>
                    <cylinder label = "outer cylinder">
                     <bottom>           [0.0,0.0,-.05715]   </bottom>
                     <top>              [0.0,0.0, .05715]   </top>
                     <radius>           0.05715            </radius>
                   </cylinder>
                   <cylinder label = "inner cylinder">
                     <bottom>           [0.0,0.0,-.0508]   </bottom>
                     <top>              [0.0,0.0, .0508]   </top>
                     <radius>           0.0508             </radius>
                   </cylinder>
                   </difference>
                 <res>                 [2,2,2]         </res>
                 <velocity>            [0.0,0.0,0.0]   </velocity>
                 <temperature>         600             </temperature>
                </geom_object>
         </material>
         <material name = "reactant">
             <include href="inputs/MPM/MaterialData/MatConstPBX9501.xml"/>
             <constitutive_model type = "visco_scram">
               <include href="inputs/MPM/MaterialData/ViscoSCRAMPBX9501.xml"/>
               <include href="inputs/MPM/MaterialData/TimeTempPBX9501.xml"/>
               <randomize_parameters>          false </randomize_parameters>
               <use_time_temperature_equation> true  </use_time_temperature_equation>
               <useObjectiveRate>              true  </useObjectiveRate>
               <useModifiedEOS>                true  </useModifiedEOS>
             </constitutive_model>
                <geom_object>
                   <difference>
                     <cylinder label = "inner cylinder"> </cylinder>
                     <cylinder label = "inner hole">
                       <bottom>           [0.0,0.0,-.0508]   </bottom>
                       <top>              [0.0,0.0, .0508]   </top>
                       <radius>           0.01             </radius>
                     </cylinder>
                  </difference>
                  <res>                 [2,2,2]         </res>
                  <velocity>            [0.0,0.0,0.0]   </velocity>
                  <temperature>         440.0           </temperature>
                </geom_object>
         </material>

            <contact>
              <type>approach</type>
              <materials>              [0,1]         </materials>
              <mu> 0.0 </mu>
            </contact>
            <thermal_contact>
            </thermal_contact>
      </MPM>

       <ICE>
         <material>
           <EOS type = "ideal_gas">
           </EOS>
           <dynamic_viscosity>          0.0            </dynamic_viscosity>
           <thermal_conductivity>       0.0            </thermal_conductivity>
           <specific_heat>              716.0          </specific_heat>
           <gamma>                      1.4            </gamma>
           <geom_object>
             <difference>
                  <box>
                    <min>           [-0.254,-0.254,-0.254] </min>
                    <max>           [ 0.254, 0.254, 0.254] </max>
                  </box>
                 <cylinder label = "outer cylinder"> </cylinder>
               </difference>
             <cylinder label="inner hole"> </cylinder>
             <res>                      [2,2,2]        </res>
             <velocity>           [0.0,0.0,0.0]       </velocity>
             <!--
             <temperature>        300.0               </temperature>
             <density>    1.1792946927374306000e+00   </density>
             -->
             <temperature>        400.0               </temperature>
             <density>            0.884471019553073   </density>
             <pressure>           101325.0             </pressure>
           </geom_object>
         </material>
      </ICE>       

      <exchange_properties>  
         <exchange_coefficients>
              <momentum>  [0, 1e15, 1e15]     </momentum>
              <heat>      [0, 1e10, 1e10]     </heat>
        </exchange_coefficients>
      </exchange_properties> 
    </MaterialProperties>

    <AddMaterialProperties>  
      <ICE>
         <material name = "product">
           <EOS type = "ideal_gas">
           </EOS>
           <dynamic_viscosity>          0.0            </dynamic_viscosity>
           <thermal_conductivity>       0.0            </thermal_conductivity>
           <specific_heat>              716.0          </specific_heat>
           <gamma>                      1.4            </gamma>
           <geom_object>
                  <box>
                    <min>           [ 1.0, 1.0, 1.0] </min>
                    <max>           [ 2.0, 2.0, 2.0] </max>
                  </box>
             <res>                      [2,2,2]        </res>
             <velocity>           [0.0,0.0,0.0]       </velocity>
             <temperature>        300.0               </temperature>
             <density>    1.1792946927374306000e+00   </density>
             <pressure>           101325.0             </pressure>
           </geom_object>
         </material>
      </ICE>

      <exchange_properties>
         <exchange_coefficients>
              <momentum>  [0, 1e15, 1e15, 1e15, 1e15, 1e15]     </momentum>
              <heat>      [0, 1e10, 1e10, 1e10, 1e10, 1e10]     </heat>
              <!--
              <heat>      [0, 1, 1, 1, 1, 1]     </heat>
              -->
        </exchange_coefficients>
      </exchange_properties>
    </AddMaterialProperties>


    <Models>
      <Model type="Simple_Burn">
        <Active>       false        </Active>
        <fromMaterial> reactant     </fromMaterial>
        <toMaterial>   product      </toMaterial>
        <ThresholdTemp>       450.0 </ThresholdTemp>
        <ThresholdPressure> 50000.0 </ThresholdPressure>
        <Enthalpy>        2000000.0 </Enthalpy>
        <BurnCoeff>            75.3 </BurnCoeff>
        <refPressure>      101325.0 </refPressure>
      </Model>
    </Models>


</Uintah_specification>
  \end{verbatim}

The PBS script used to run this test  is
\begin{verbatim}

#
# ASK PBS TO SEND YOU AN EMAIL ON CERTAIN EVENTS: (a)bort (b)egin (e)nd (n)ever
#
# (User May Change)

#PBS -m abe

#
# SET THE NAME OF THE JOB:
#
# (User May Change)

#PBS -N ExplodeRing

#
# SET THE QUEUE IN WHICH TO RUN THE JOB.
#      (Note, there is currently only one queue, so you should never change this field.)

#PBS -q defaultq

#
# SET THE RESOURCES (# NODES, TIME) REQESTED FROM THE BATCH SCHEDULER:
#   - select: <# nodes>,ncpus=2,walltime=<time>
#     - walltime: walltime before PBS kills our job.
#               [[hours:]minutes:]seconds[.milliseconds]
#               Examples:
#                 walltime=60      (60 seconds)
#                 walltime=10:00   (10 minutes)
#                 walltime=5:00:00 (5 hours)
#
# (User May Change)

#PBS -l select=2:ncpus=2,walltime=24:00

#
# START UP LAM

cd $PBS_O_WORKDIR
lamboot

# [place your command here] >& ${PBS_O_WORKDIR}/output.${PBS_JOBID}
mpirun -np 4 ../sus_opt explodeRFull.ups >& output.${PBS_JOBID}

#
# REMEMBER, IF YOU ARE RUNNING TWO SERIAL JOBS, YOU NEED A:
# wait

#
# STOP LAM

lamhalt -v
exit
\end{verbatim}

  \bibliographystyle{unsrt}
  \bibliography{Bibliography}

\end{document}
