%
% Material models in Uintah
% Biswajit Banerjee
% 03/25/2003
%

\documentclass[10pt]{article}
\usepackage[pdftex]{graphicx}
\usepackage{times}
\usepackage{amsfonts}
\usepackage{amsmath}
\usepackage{amssymb}
\usepackage{amstext}

\topmargin 0.5in
\headsep 0pt
\headheight 0pt
\oddsidemargin 0pt
\evensidemargin 0pt
\textheight 8.6in
\textwidth 6.5in
\columnsep 20pt
\columnseprule 0.5pt
\setlength{\parskip}{0.5em}
\raggedright

\title{Material Models and the Uintah Computational Framework}
\author{
\\
Biswajit Banerjee \\
Department of Mechanical Engineering \\
University of Utah \\
Salt Lake City, Utah 84112 \\
\\
}
\date{}

\begin{document}
  \maketitle
  \tableofcontents

  \section{Introduction}
  This document deals with some material models for solids that
  have been implemented in the Uintah Computational Framework 
  for use with the Material Point Method.  The approach taken
  has been to separate the stress-strain relations from the 
  numerical stress update algorithms as far as possible.  Two
  stress update algorithms (hypoelastic-plastic and 
  hyperelastic-plastic) are discussed and the manner in which 
  plasticity flow rules, damage models and equations of state
  fit into the stress update algorithms are shown.
  
  \section{Stress Update Algorithms}
  The hypoelastic-plastic stress update is based on an additive
  decomposition of the rate of deformation tensor into elastic
  and plastic parts while the hyperelastic-plastic stress update
  is based on a multiplicative decomposition of the elastic and 
  plastic deformation gradients.  Incompressibility is assumed
  for plastic deformations.  The volumetric response is therefore
  determined either by a bulk modulus and the trace of the rate
  of deformation tensor or using an equation of state.  The 
  deviatoric response is determined either by an elastic constitutive 
  equation or using a plastic flow rule in combination with a 
  yield condition. 
  
  The material models that can be varied in these stress update approaches
  are (not all are applicable to both hypo- and hyperelastic formulations
  nor is the list exhaustive):
  \begin{enumerate}
    \item The elasticity model, for example, 
      \begin{itemize}
        \item Isotropic linear elastic model. 
        \item Anisotropic linear elastic models.
        \item Isotropic nonlinear elastic models.
        \item Anisotropic nonlinear elastic models.
      \end{itemize}
    \item Isotropic hardening or Kinematic hardening using a 
          back stress evolution rule, for example,
      \begin{itemize}
        \item Ziegler evolution rule   .
      \end{itemize}
    \item The flow rule and hardening/softening law, for example,
      \begin{itemize} 
        \item Perfect plasticity/power law hardening plasticity.
        \item Johnson-Cook plasticity .
        \item Mechanical Threshold Stress (MTS) plasticity .
        \item Anand plasticity .
      \end{itemize} 
    \item The yield condition, for example,
      \begin{itemize} 
        \item von Mises yield condition.
        \item Drucker-Prager yield condition.
        \item Mohr-Coulomb yield condition.
      \end{itemize} 
    \item A continuum or nonlocal damage model with damage evolution
          given by, for example, 
      \begin{itemize}
        \item Johnson-Cook damage model.
        \item Gurson-Needleman-Tvergaard model.
        \item Sandia damage model.
      \end{itemize}
    \item An equation of state to determine the pressure (or 
          volumetric response), for example, 
      \begin{itemize}
        \item Mie-Gruneisen equation of state.
      \end{itemize}
  \end{enumerate}
  
  The currently implemented stress update algorithms in the Uintah
  Computational Framework do not allow for arbitrary elasticity models,
  kinematic hardening, arbitrary yield conditions and continuum or 
  nonlocal damage (however the a damage parameter is updated and 
  used in the erosion algorithm).  The models that can be varied
  are the flow rule models, damage variable evolution models and the
  equation of state models.

  Note that there are no checks in the Unitah Computational Framework
  to prevent users from mixing and matching inappropriate models.

  \subsection{Hypoelastic-plastic Stress Update}
  This section describes the current implementation of the hypoelastic-
  plastic model.

  The elastic response is assumed to be isotropic.  The material
  constants that are taken as input for the elastic response are the
  bulk and shear modulus.  The flow rule is determined from the input
  and the appropriate plasticity model is created using the 
  \verb+PlasticityModelFactory+ class.  The damage evolution rule
  is determined from the input and a damage model is created using
  the \verb+DamageModelFactory+ class.  The equation of state 
  that is used to determine the pressure is also determined from the
  input.  The equation of state model is created using the 
  \verb+MPMEquationOfStateFactory+ class.

  The evolution variables specific to the hypoelastic-plastic model
  are the left stretch ($\mathbf{V}$) and the rotation ($\mathbf{R}$).
  In addition, a damage evolution variable ($D$) is stored at each time 
  step (this need not be the case and will be transfered to the 
  damage models in the future).  The left stretch and rotation are 
  updated incrementally at each
  time step (instead of performing a polar decomposition) and the 
  rotation tensor is used to rotate the Cauchy stress and rate of deformation
  to the material coordinates at each time step (instead of using a 
  objective stress rate formulation).  The stress update formula implemented
  is one based on the approach taken by  Zocher et al. (2000)~\cite{Zocher2000}.  

  Any evolution variables for the plasticity model, damage model or the
  equation of state are specified in the class that encapsulates the 
  particular model.  

  The flow stress is calculated from the plasticity model using a 
  function call of the form
  \begin{verbatim}
    double flowStress = d_plasticity->computeFlowStress(tensorEta, tensorS, 
                                                        pTemperature[idx],
                                                        delT, d_tol, matl, idx);
  \end{verbatim}
  A number of plasticity models can be evaluated using the inputs in the
  \verb+computeFlowStress+ call.  The variable \verb+d_plasticity+ is
  polymorphic and can represent any of the plasticity models that can be
  created by the plasticity model factory.  The plastic evolution variables
  are updated using a polymorphic function along the lines of
   \verb+computeFlowStress+.

  The equation of state is used to calculate the hydrostatic stress using
  a function call of the form
  \begin{verbatim}
    Matrix3 tensorHy = d_eos->computePressure(matl, bulk, shear, 
                                              tensorF_new, tensorD, 
                                              tensorP, pTemperature[idx], 
                                              rho_cur, delT);
  \end{verbatim}

  Similarly, the damage model is called using a function of the type
  \begin{verbatim}
    double damage = d_damage->computeScalarDamage(tensorEta, tensorS, 
                                                  pTemperature[idx],
                                                  delT, matl, d_tol, 
                                                  pDamage[idx]);
  \end{verbatim}

  Therefore, the plasticity, damage and equation of state models are 
  easily be inserted into any other type of stress update algorithm 
  without any change being needed in them as can be seen in the 
  hyperelastic-plastic stress update algorithm discussed below.

  \subsection{Hyperelastic-plastic Stress Update}
  The stress update used for the hyperelastic-plastic material is
  a return mapping plasticity algorithm based on a multiplicative 
  decomposition of the deformation gradient and the intermediate
  configuration concept.  The algorithms has been taken from
  Simo and Hughes (1998)~\cite{Simo1998}.

  The variables that are evolved locally in the Hyperelastic-plastic 
  stress update algorithm are the deviatoric part of the left Cauchy-Green
  tensor and a scalar damage variable.  

  The plasticity, damage and equation of state models are initialized
  and called in exactly the same way as in the hypoelastic-plastic 
  model.  Therefore, the stress update algorithm and the plasticity,
  damage and equation of state models are encapsulated from each other
  and no modifications of the latter are required.

  \section{Plasticity Model Framework}
  Currently implemented plasticity models are isotropic hardening, 
  Johnson-Cook and MTS.  Addition of a new model requires the 
  following steps :

  \begin{enumerate}
    \item Creation of a new class that encapsulates the plasticity 
    model.  The template for this class can be copied from the
    existing plasticity models.  The data that is unique to 
    the new model are specified in the form of 
    \begin{itemize}
      \item A structure containing the constants for the plasticity
            model.
      \item Particle variables that specify the variables that 
            evolve in the plasticity model.
    \end{itemize}
    \item The implementation of the plasticity model involves the
    following steps.
    \begin{itemize}
      \item Reading the input file for the model constants in the
            constructor.
      \item Adding the variables that evolve in the plasticity model
            appropriately to the task graph.
      \item Adding the appropriate flow stress calculation method.
    \end{itemize}
    \item The \verb+PlasticityModelFactory+ is then modified so that
          it recognizes the added plasticity model.
  \end{enumerate}

  \section{Damage and Equation of State Model Frameworks}
  Only the Johnson-Cook damage evolution rule has been added to the 
  DamageModelFactory so far.  The damage model framework is designed 
  to be similar to the plasticity model framework.  New models can
  be added using the approach described in the previous section.

  The equation of state models include an isotropic hypoelastic 
  model, an isotropic neo-Hookean model and a Mie-Gruneisen type
  equation of state.  Once again the framework is designed to be
  similar to the plasticity model framework.
  
  \section{Example Input Files}
  \subsection{Hypoelastic-plastic Stress Update}
  An example of the portion of an input file that specifies a copper body
  with a hypoelastic stress update, Johnson-Cook plasticity model,
  Johnson-Cook Damage Model and Mie-Gruneisen Equation of State is shown 
  below.
  \begin{verbatim}
  <material>

    <include href="inputs/MPM/MaterialData/MaterialConstAnnCopper.xml"/>
    <constitutive_model type="hypoelastic_plastic">
      <tolerance>5.0e-10</tolerance>
      <include href="inputs/MPM/MaterialData/IsotropicElasticAnnCopper.xml"/>
      <include href="inputs/MPM/MaterialData/JohnsonCookPlasticAnnCopper.xml"/>
      <include href="inputs/MPM/MaterialData/JohnsonCookDamageAnnCopper.xml"/>
      <include href="inputs/MPM/MaterialData/MieGruneisenEOSAnnCopper.xml"/>
    </constitutive_model>

    <burn type = "null" />
    <velocity_field>1</velocity_field>

    <geom_object>
      <cylinder label = "Cylinder">
        <bottom>[0.0,0.0,0.0]</bottom>
        <top>[0.0,2.54e-2,0.0]</top>
        <radius>0.762e-2</radius>
      </cylinder>
      <res>[3,3,3]</res>
      <velocity>[0.0,-208.0,0.0]</velocity>
      <temperature>294</temperature>
    </geom_object>

  </material>
  \end{verbatim}

  The general material constants for copper are in the file 
  \verb+MaterialConstAnnCopper.xml+.  The contents are shown below
  \begin{verbatim}
  <?xml version='1.0' encoding='ISO-8859-1' ?>
  <Uintah_Include>
    <density>8930.0</density>
    <toughness>10.e6</toughness>
    <thermal_conductivity>1.0</thermal_conductivity>
    <specific_heat>383</specific_heat>
    <room_temp>294.0</room_temp>
    <melt_temp>1356.0</melt_temp>
  </Uintah_Include>
  \end{verbatim}

  The elastic properties are in the file \verb+IsotropicElasticAnnCopper.xml+.
  The contents of this file are shown below.
  \begin{verbatim}
  <?xml version='1.0' encoding='ISO-8859-1' ?>
  <Uintah_Include>
    <shear_modulus>45.45e9</shear_modulus>
    <bulk_modulus>136.35e9</bulk_modulus>
  </Uintah_Include>
  \end{verbatim}
  
  The constants for the Johnson-Cook plasticity model are in the file
  \verb+JohnsonCookPlasticAnnCopper.xml+.  The contents of this file are
  shown below.
  \begin{verbatim}
  <?xml version='1.0' encoding='ISO-8859-1' ?>
  <Uintah_Include>
    <plasticity_model type="johnson_cook">
      <A>89.6e6</A>
      <B>292.0e6</B>
      <C>0.025</C>
      <n>0.31</n>
      <m>1.09</m>
    </plasticity_model>
  </Uintah_Include>
  \end{verbatim}

  The constants for the Johnson-Cook damage model are in the file
  \verb+JohnsonCookDamageAnnCopper.xml+.  The contents of this file are
  shown below.
  \begin{verbatim}
  <?xml version='1.0' encoding='ISO-8859-1' ?>
  <Uintah_Include>
    <damage_model type="johnson_cook">
      <D1>0.54</D1>
      <D2>4.89</D2>
      <D3>-3.03</D3>
      <D4>0.014</D4>
      <D5>1.12</D5>
    </damage_model>
  </Uintah_Include>
  \end{verbatim}

  The constants for the Mie-Gruneisen model (as implemented in the 
  Uintah Computational Framework) are in the file
  \verb+MieGruneisenEOSAnnCopper.xml+.  The contents of this file are
  shown below.
  \begin{verbatim}
  <?xml version='1.0' encoding='ISO-8859-1' ?>
  <Uintah_Include>
    <equation_of_state type="mie_gruneisen">
      <C_0>3940</C_0>
      <Gamma_0>2.02</Gamma_0>
      <S_alpha>1.489</S_alpha>
    </equation_of_state>
  </Uintah_Include>
  \end{verbatim}

  As can be seen from the input file, any other plasticity model, damage
  model and equation of state can be used to replace the Johnson-Cook
  and Mie-Gruneisen models without any extra effort (provided the models
  have been implemented and the data exists).

  \subsection{Hyperelastic-plastic Stress Update}
  An example of the portion of an input file that specifies a copper body
  with a hyperelastic stress update, Mechanical Threshold Stress plasticity model,
  Johnson-Cook Damage Model and Mie-Gruneisen Equation of State is shown 
  below.
  \begin{verbatim}
  <material>

    <include href="inputs/MPM/MaterialData/MaterialConstAnnCopper.xml"/>
    <constitutive_model type="hyperelastic_plastic">
      <tolerance>5.0e-10</tolerance>
      <include href="inputs/MPM/MaterialData/IsotropicElasticAnnCopper.xml"/>
      <include href="inputs/MPM/MaterialData/MTSPlasticAnnCopper.xml"/>
      <include href="inputs/MPM/MaterialData/JohnsonCookDamageAnnCopper.xml"/>
      <include href="inputs/MPM/MaterialData/MieGruneisenEOSAnnCopper.xml"/>
    </constitutive_model>

    <burn type = "null" />
    <velocity_field>1</velocity_field>

    <geom_object>
      <cylinder label = "Cylinder">
        <bottom>[0.0,0.0,0.0]</bottom>
        <top>[0.0,2.54e-2,0.0]</top>
        <radius>0.762e-2</radius>
      </cylinder>
      <res>[3,3,3]</res>
      <velocity>[0.0,-208.0,0.0]</velocity>
      <temperature>294</temperature>
    </geom_object>

  </material>
  \end{verbatim}

  The data files are the same as those for the hypoelastic-plastic case, 
  except for the plasticity model.  The constants for the MTS plasticity model
  are specified in the file \verb+MTSPlasticAnnCopper.xml+.  The contents
  of this file are show below.
  \begin{verbatim}
  <?xml version='1.0' encoding='ISO-8859-1' ?>
  <Uintah_Include>
    <plasticity_model type="mts_model">
      <s_a>40.0e6</s_a>
      <koverbcubed>0.823e6</koverbcubed>
      <edot0>1.0e7</edot0>
      <g0>1.6</g0>
      <q>1.0</q>
      <p>0.666667</p>
      <alpha>2</alpha>
      <edot_s0>1.0e7</edot_s0>
      <A>0.2625</A>
      <s_s0>770.0e6</s_s0>
      <a0>2390.0e6</a0>
      <a1>12.0e6</a1>
      <a2>1.696e6</a2>
      <b1>45.78e9</b1>
      <b2>3.0e9</b2>
      <b3>180</b3>
      <mu_0>47.7e9</mu_0>
    </plasticity_model>
  </Uintah_Include>
  \end{verbatim}

  The material data can easily be taken from a material database or specified
  for a new material in an input file kept at a centralized location.

  \bibliographystyle{unsrt}
  \bibliography{Bibliography}

\end{document}
