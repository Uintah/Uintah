% running.tex
%
% This is the `Running SCIRun' main section.

%begin{latexonly}
  \newcommand{\srwindow}%
  {\centerline{\epsfig{file=figures/srwindow.eps,width=\columnwidth}}}
%end{latexonly}
\begin{htmlonly}
  \newcommand{\srwindow}{%
  \htmladdimg[align=top,width=6,alt="SCIRun Window"]
  {../figures/srwindow.gif}}
\end{htmlonly}


\section{Starting Up}
\label{sec:startingup}

\subsection{The \command{scirun} Command}
\label{sec:sciruncmd}

Start \sr\ by typing \keyboard{scirun} in a terminal (\eg \command{xterm})
window.  Don't start \sr\ in the background, \ie don't type
\keyboard{scirun \&}.

The \command{scirun} command is located in the \directory{src} directory of
the \directory{\sr} install directory.  The person who installed \sr\ can
locate this command for you.

Typing \keyboard{scirun} with no arguments starts up \sr\ with a blank \sr
window as shown in Figure~\ref{fig:srwindow}.  The main features of this
window are discussed in \secref{Anatomy of the Main
  Window}{sec:windowanatomy}.

The \command{scirun} command may take 1 argument which is the name of a
\sr\ \dfn{network} \index{network} file (these files have a \filename{.net}
extension).  These files hold previously defined \sr\ networks.  \sr\ will
load the specified network.  Network files will be discussed in a later
section.

\sr\ may encounter errors during start up.  These will be displayed in \sr{}'s
error message pane (see Figure~\ref{fig:srwindow}).  \AuthorNote{Need to
  add refs to bug reporting here.}

\subsubsection{Anatomy of the Main Window}
\label{sec:windowanatomy}

The \sr\ main window consists of 4 main components (see Figure~\ref{fig:srwindow}):

\begin{figure}[htb]
  \begin{makeimage}
  \end{makeimage}
  \srwindow
  \caption{\label{fig:srwindow} \sr\ Main Window}
\end{figure}

\begin{description}
\item[Menu Bar] The menu bar is used to load networks, save networks, quit
  \sr, create network modules, and perform other tasks.  The menu bar
  consists of the following menu items:

  \begin{description}
  \item[\menu{File}] 
    \begin{description}
    \item[Save] Saves the current network to a file.
    \item[Load] Loads a network from a file.
    \item[New] This submenu contains items of interest to developers only.
    \item[Add Info] Use this item to add network specific notes to
      the current network.  Notes should be used to document the purpose of
      the network.
    \item[Quit] Quits \sr.
    \end{description}
  \end{description}
  
  \begin{description}
  \item[\menu{SCIRun}] The \menu{SCIRun} menu is used to create modules
    (from the \sr\ package) for use in the network pane.  This menu is
    composed of submenus. Each submenu corresponds to a \dfn{category}
    \index{category} within the \sr\ package. 
    
    An overview of the contents of the \sr\ package is given in \secref{The
      SCIRun Package}{sec:srpackage}.

    Each menu item in a category submenu creates a specific module and
    places it in the network pane.  A category is group of related modules.
    
    The network pane's popup menu (activated by clicking the right most
    mouse button when the mouse pointer is in the network pane) also
    provides access to the \menu{SCIRun} (and possibly other) package
    menus.
  \end{description}

  \begin{description}
  \item [\textit{Possibly Other Package Menus}] There may be other package
    menus if other packages have been installed.
  \end{description}

\item[Navigator Pane] The Navigator Pane is used to navigate complex
  networks.  The Network Pane's scroll bars can also be used for network
  navigation.  The use of the Navigator Pane will be described in
  \secref{Building Networks}{sec:bldnetworks}.
  
\item[Error Pane] Errors during program startup are displayed in the Error
  Pane.  Errors on startup may mean that \sr\ has been installed
  incorrectly or has been installed from a buggy distribution.
  
\item[Network Pane] The Network Pane is used to build and execute networks.
  \secref{Building Networks}{sec:workwithnets} discusses the use of this
  pane.

\end{description}

\subsection{The Terminal Window App}
\label{sec:termwinapp}

After starting, \sr\ will also run a shell-like application in the terminal
window.  It will display the prompt \screen{uintah\ra}.  This program is
actually a modified \dfna{Tool Command Language}{TCL} shell program and it
is possible to type in \acronym{TCL}'ish \sr\ commands at the prompt. The
use of this program will be discribed in a later section.


\subsection{\envvar{SCIRUN\_DATA}}
\label{sec:scirundata}

The environment variable \envvar{SCIRUN\_DATA} affects the behavior of \sr\ 
It specifies the default location of \sr\ data files.  It mainly affects
the behavior of file browsing dialogs -- they will prompt for a file within
the \envvar{SCIRUN\_DATA} directory (of course you may have the dialog look
elsewhere).


\subsection{Stopping}
\label{sec:stopping}

Quit \sr\ by selecting the \menuitem{Quit} item from the \menu{File} menu

Don't press \keyboard{Ctrl-c} to exit \sr.  Doing this will drop you into
a debugger which is probably not what you want to do.

% -*-latex-*-
%
%  For more information, please see: http://software.sci.utah.edu
% 
%  The MIT License
% 
%  Copyright (c) 2004 Scientific Computing and Imaging Institute,
%  University of Utah.
% 
%  License for the specific language governing rights and limitations under
%  Permission is hereby granted, free of charge, to any person obtaining a
%  copy of this software and associated documentation files (the "Software"),
%  to deal in the Software without restriction, including without limitation
%  the rights to use, copy, modify, merge, publish, distribute, sublicense,
%  and/or sell copies of the Software, and to permit persons to whom the
%  Software is furnished to do so, subject to the following conditions:
% 
%  The above copyright notice and this permission notice shall be included
%  in all copies or substantial portions of the Software.
% 
%  THE SOFTWARE IS PROVIDED "AS IS", WITHOUT WARRANTY OF ANY KIND, EXPRESS
%  OR IMPLIED, INCLUDING BUT NOT LIMITED TO THE WARRANTIES OF MERCHANTABILITY,
%  FITNESS FOR A PARTICULAR PURPOSE AND NONINFRINGEMENT. IN NO EVENT SHALL
%  THE AUTHORS OR COPYRIGHT HOLDERS BE LIABLE FOR ANY CLAIM, DAMAGES OR OTHER
%  LIABILITY, WHETHER IN AN ACTION OF CONTRACT, TORT OR OTHERWISE, ARISING
%  FROM, OUT OF OR IN CONNECTION WITH THE SOFTWARE OR THE USE OR OTHER
%  DEALINGS IN THE SOFTWARE.
%


\chapter{Working with Networks}
\label{ch:workwithnets}

This section describes how to create, save, load, execute, and edit
networks.

The following conventions are used when describing mouse and keyboard
actions performed by the user:

\begin{description}
\descitem{Button1} Left mouse button

\descitem{Button2} Middle mouse button

\descitem{Button3} Right mouse button

\descitem{Press} ``Press'' means press and hold a key or mouse button.

\descitem{Click} ``Click'' means press and release a mouse button.

\descitem{Type} ``Type''  means press and release a key.

\descitem{Button\replaceable{N}-Drag} Press Button\replaceable{N},
then drag the mouse.

\descitem{C-Button\replaceable{N}} Press the \keyboard{Control} key,
then press or click Button\replaceable{N}

\descitem{C-\replaceable{key}} Press and hold the control key, then
type key \replaceable{key}
\end{description}


\section{Anatomy of a Module}
\label{sec:modanatomy}

%begin{latexonly}
\newcommand{\modgraphic}{%
  \centerline{\includegraphics[bb=0 0 325 157,width=4in]{Figures/modgraphic-1.eps.gz}}
}
%end{latexonly}
\begin{htmlonly}
  \newcommand{\modgraphic}{%
  \htmladdimg[alt="SCIRun Module Graphic"]{../Figures/modgraphic-1.gif}
}
\end{htmlonly}

All modules \index{module} are similarly represented by a graphic within the NetEdit frame
(see Figure~\ref{fig:modgraphic}). The graphical ``front end'' is the same
for all modules and consists of the following elements:

\begin{figure}[htb]
  \begin{makeimage}
  \end{makeimage}
  \modgraphic
  \caption{\label{fig:modgraphic} Module Graphic (\module{Show
      Field} Module)}
\end{figure}

\begin{description}
  \descitem{Module Name} The module's name.
  
  \descitem{Input Ports} Zero or more input ports located on the top
  of the module.  Each port corresponds to a data type and each data
  type has a unique color.  Table~\ref{tab:portcolors} maps port
  colors to data types.  Input ports connect to other modules' output
  ports.  Connections can only be made between ports of the same type.
  Some modules have a \dfn{dynamic} input port.  When a connection is
  made to a dynamic input port a new instance of the port will be
  created.

  \begin{table}[htbp]
    \begin{center}
      \begin{tabular}{|l|l|}
        \hline
        \textbf{Data Type} & \textbf{Port Color} \\
        \hline
        Field & Yellow \\
        Field Set & Green \\
        Matrices & Blue \\
        Geometric Objects & Pink \\
        Color Maps & Purple \\
        Camera Path & Brown \\
        \hline
      \end{tabular}
      \caption{Data Types and their Port Colors}
      \label{tab:portcolors}
    \end{center}
  \end{table}
  
  \descitem{Output ports} Zero or more output ports located on the
  bottom of the module.  Output ports connect to other modules' input
  ports.  Every module has  at least one input or one output
  port.
  
  \descitem{UI button} Pressing the \guibutton{UI} button displays the
  module's control dialog. Some modules have no dialog, some have
  simple dialogs, and some have complex dialogs that allow elaborate
  control over the module.  Figure~\ref{fig:moddialog} shows the
  control dialog for \module{Show Field} module.  Most module control
  dialogs (except for read/writer modules) contain the following
  buttons:

  \begin{description}
    \descitem{Close}  Closes the control dialog.

    \descitem{Execute} Executes the module.  This may cause other
    modules in the network to execute or ``fire'' (see
    \secref{Executing a Network}{sec:executenet}).
    
    \descitem{Find} Highlights the module in the NetEdit frame
    that owns the control dialog being edited.

    \descitem{?} Displays a module's help information.

  \end{description}

  Reader/writer module control dialogs contain the following button
  set:

  \begin{description}
    \descitem{Set} Sets the read/writer module's filename value to the
    chosen filename then closes the module's control dialog.  In a
    network with multiple reader/writer modules, \guibutton{Set}
    (rather than \guibutton{Execute}) should be used to establish each
    reader/writer module's filename parameter before executing the
    network.
    
    \descitem{Execute}  Sets the read/writer module's filename value
    to the chosen filename, executes the module (reads file contents
    and sends data downstream), and closes the module's
    control dialog.
    
    \descitem{Cancel} Closes the control dialog without setting the
    module's filename value.
    
    \descitem{Find} Highlights the module in the NetEdit frame
    that owns the control dialog being edited.
    
  \end{description}
  
  \descitem{Progress bar} Shows the module's progress.  As the module
  works toward completion of its task, the progress bar is filled
  with red, then yellow, then green.  When the Progress bar
  is green, the module is done.
  
  \descitem{Timer} Displays the amount of CPU time (seconds) the module has
  consumed.  Located to the left of the progress bar.
  
  \descitem{Message Indicator} Indicates the presence of messages in a
  module's log.  Colors represent message types.  Blue represents
  ``remarks'' (informational messages) and yellow represents
  ``warnings'' (attention is needed).  Click \keyboard{Button1} on the
  indicator to display the module's log.
  
  \descitem{Pop-up Menu} Pressing \keyboard{Button3} while the mouse pointer is
  over a module activates a module's pop-up menu.  The content
  of the pop-up menu changes depending on the state of the NetEdit
  window. The following items are available in the pop-up menu:

  \begin{description}
    \menuitemdesc{Package\_Category\_Name\_Instance} This item is a
    label (not a selectable item).  It provides the module's name and
    the category and package to which the module belongs.
    ``Instance'' is a unique number that distinguishes multiple
    instances of the same module.
    
    \menuitemdesc{Execute} Tells the module to execute (or
    re-execute).  This may cause other modules in the network to
    execute or``fire'' (see \secref{Executing a
      Network}{sec:executenet}).

    \menuitemdesc{Help} Displays the module's help window.
    
    \menuitemdesc{Notes} Displays a module's note pad.  The note pad
    is used to document the purpose of a module in a network.  See
    \secref{Displaying Module Notes}{sec:modnotes}.

    \menuitemdesc{Destroy Selected} Destroys the selected modules.
    See \secref{Destroying Module(s)}{sec:destroymod}.
    
    \menuitemdesc{Destroy} Destroys the module See \secref{Destroying
      Module(s)}{sec:destroymod}.

    \menuitemdesc{Duplicate} Duplicates the module, its input connections,
    and its control dialog state.
    
    \menuitemdesc{Replace With} Replaces the module with the chosen
    one.  \menu{Replace With} is a pop-up menu listing modules that
    accept the same connections as the current module.  The chosen
    module replaces the current module.
    
    \menuitemdesc{Show Log} Displays the module's message log.  Most
    modules write messages to their log during the course of
    their execution (see \secref{Viewing a Module's Log}{sec:viewmodslog}).
    
    \menuitemdesc{Make Sub-Network} Creates a sub-network from
    selected modules.  See \secref{Creating a Sub-Network}{sec:crsubnet}.
    
    \menuitemdesc{Expand Sub-Network} Reverses the action of menu item
    \guimenuitem{Make Sub-Network}.  This menu item is available only in
    a sub-network's pop-up menu.  See \secref{Expanding a
      Sub-Network}{sec:expsubnet}.
    
    \menuitemdesc{Disable} Disables a module (or selected modules).
    See \secref{Disabling/Enabling Modules}{sec:disablemod}.
    
    \menuitemdesc{Enable} Enables a module (or selected modules).  See
    \secref{Disabling/Enabling Modules}{sec:disablemod}.

  \end{description}
\end{description}


\section{Anatomy of a Connection}
\label{sec:anatcon}
\index{connections}

A connection lets data flow from one module's output port to (usually) another
module's input port.  Sometimes a connection will connect a module's
output port to its input port.

Connections are colored grey if they are disabled.

Pressing \keyboard{Button3} while the mouse pointer is over a connection
activates a connection's pop-up menu.  The content of the pop-up menu
changes depending on the state of the connection. The following items are
available in the pop-up menu:

\begin{description}
  \menuitemdesc{Delete} Deletes a connection.
  
  \menuitemdesc{Insert Module} Creates and inserts a module ``into'' a
  connection.  \menu{Insert Module} is a pop-up menu listing modules
  that contain at least one input and at least one output port
  matching the connection.  The chosen module is created and then the
  connection is routed into the left-most matching input port and out
  of the left-most matching output port.
  
  \menuitemdesc{Disable (or Enable)} Disables (or enables) a connection.
  
  \menuitemdesc{Notes} Displays a connection's note pad.
\end{description}


\section{Creating a Module}
\label{sec:creatingmodules}

To create a module, select its name from one of the package (\eg{}
\sr) menus' category sub-menus. Package menus are accessed from the
main window's menu bar or from the NetEdit frame's pop-up menu.  The
NetEdit frame's pop-up menu is activated by pressing
\keyboard{Button3} (right mouse button) while the mouse pointer is in
the NetEdit frame (but not over a module or connection).  The pop-up
menu contains a list of category sub-menus from the \sr{} package and
other installed packages.  Each category sub-menu provides access to
modules within the category.

After creating a module, its glyph (or graphic representation---see
Figure~\ref{fig:modgraphic}) is placed in the NetEdit frame.

\section{Setting Module Properties}
\label{sec:setmodprops}

%begin{latexonly}
\newcommand{\moddialog} {%
  \centerline{\includegraphics[bb=0 0 275 483]{Figures/moddialog.eps.gz}}
}
%end{latexonly}
\begin{htmlonly}
  \newcommand{\moddialog}{%
    \htmladdimg[alt="SCIRun Module Dialog"]{../Figures/moddialog.gif}
  }
\end{htmlonly}

To change a module's properties, click its \guibutton{UI} button.
This displays the module's control dialog.  Use the dialog to change
the module's properties.  Each \htmladdnormallinkfoot{module's
  reference
  documentation}{\latexhtml{http://software.sci.utah.edu/doc/}{../../../}Developer/Modules/index.html}
explains the use of its control dialog.  A module's reference
documentation can be displayed by clicking a module's \guibutton{?}
button.  Figure~\ref{fig:moddialog} shows the control dialog for the
\module{Show Field} module.

\begin{figure}[htb]
  \begin{makeimage}
  \end{makeimage}
  \moddialog
  \caption{\label{fig:moddialog} Module \module{Show Field}'s Control Dialog
    (User Interface).}
\end{figure}

\section{Selecting Modules}
\label{sec:selectmods}

Some operations (moving, destroying, disabling, and enabling) act on a group of
selected modules (see \secref{Moving Module(s)}{sec:movemod},
\secref{Destroying Module(s)}{sec:destroymod}, and \secref{Disabling  
Modules}{sec:disablemod}).  Groups of selected modules can be created two ways:

\begin{enumerate}
\item Perform \keyboard{Button1-Drag} while the pointer is in the
  NetEdit frame, but not over a module or a connection. This starts a
  selection box, allowing the user to select multiple modules. 
  Modules overlapping the selection box are selected.  Release \keyboard{Button1}
  to complete the selection.  Additional groups of selected modules
  are created by performing \keyboard{C-Button1-Drag}
  
\item Select the first module by clicking \keyboard{Button1} while the
  pointer is over a module.  Then add modules to the group by clicking
  \keyboard{C-Button1} on additional modules.  All previously selected
  modules remain selected.
\end{enumerate}

Selected modules are colored differently from unselected modules.

\section{Destroying Modules}
\label{sec:destroymod}

To delete a module, choose menu item \guimenuitem{Destroy} from a module's
pop-up menu.

Multiple modules can be deleted at one time. Select one or more
modules, then choose menu item \guimenuitem{Destroy Selected} from a
module's pop-up menu.

\section{Moving Modules}
\label{sec:movemod}

Modules can be moved in the NetEdit frame.  To move a module, perform
\keyboard{Button1-Drag} while the pointer is over a module, and move the module
to its new location.

Multiple modules can be moved at one time.  Select one or more
modules, then perform \keyboard{Button1-Drag} while the pointer is over any one
of the modules in the selected group.

Two modules may be moved by dragging one of the connections between
them.  To move two modules, perform \keyboard{Button1-Drag} when the
pointer is over a connection between two modules.

The NetEdit frame will scroll when a module is moved to the frame's edge.

\section{Disabling and Enabling Modules}
\label{sec:disablemod}

Modules may be temporarily disabled.  Disabled modules do not execute
and data does not flow through them.

To disable modules, select one or more modules, then
choose menu item \guimenuitem{Disable} from a module's pop-up menu.

Disabling a module disables all incoming and outgoing connections to
the module and prevents the module from executing during execution of
the network. Disabled modules are drawn in a darker shade of grey.

To enable disabled modules, choose \guimenuitem{Enable} from a disabled
module's pop-up menu.  Enabling one module enables all selected modules.

See also \secref{Disabling/Enabling Connections}{sec:disableconnect}.

\section{Creating a Sub-Network}
\label{sec:crsubnet}
\index{sub-network}

%begin{latexonly}
\newcommand{\subnetgraphicfig}{%
  \centerline{\includegraphics[bb=0 0 252 144]{Figures/subnetgraphic.eps.gz}}
}
%end{latexonly}
\begin{htmlonly}
  \newcommand{\subnetgraphicfig}{%
    \htmladdimg[alt="Graphic of a Sub-network"]{../Figures/subnetgraphic.gif}
  }
\end{htmlonly}


A sub-network is a group of modules that are treated as a single
module.  In a network, you may use a sub-network as you would a
module.

To create a sub-network, select one or more modules, then choose item
\guimenuitem{Create Sub-Network} from any selected module's pop-up menu.
Selected modules will be replaced by a sub-network graphic and the
sub-network editor will activate.  Sub-networks may be created in
sub-networks.  The next section (\secref{Editing a
  Sub-Network}{sec:editsubnet}) explains the use of the sub-network
editor.

A sub-network editor button replaces the CPU time and Progress
Bar in the sub-network graphic.  See Figure~\ref{fig:subnetgraphic}

\begin{figure}[htb]
  \centering
  \begin{makeimage} \end{makeimage}
  \subnetgraphicfig
  \caption{\label{fig:subnetgraphic} A Sub-Network in the NetEdit frame}
\end{figure}

A sub-network's editor is activated by pressing the
sub-network editor button.

Pressing button \guibutton{UI} activates control dialogs of all
modules in a sub-network.  Control dialogs can be activated
individually in the sub-network editor.

In a network, a sub-network behaves as does a module.  A sub-network
has input and/or output ports (orginating from modules within the
sub-network) that can be connected to modules outside of the
sub-network.  Each sub-network has a pop-up menu and a set of editable
notes (see \secref{Editing and Displaying Module
  Notes}{sec:modnotes}).  A sub-network does not have a log. Logs of
individual modules in the sub-network can be viewed in the sub-network
editor window (see \secref{Viewing a Module's Log}{sec:viewmodslog}).

\section{Expanding a Sub-Network}
\label{sec:expsubnet}

Expanding a network reverses the action of menu item \guimenuitem{Create
  Sub-Network}---all modules in a sub-network are returned to the
network containing the sub-network and the sub-network is destroyed.
This action cannot be undone.

To expand a sub-network choose menu item \guimenuitem{Expand Sub-Network}
from the sub-network's pop-up menu.

\section{Editing a Sub-Network}
\label{sec:editsubnet}

%begin{latexonly}
\newcommand{\subneteditorfig}{%
  \centerline{\includegraphics[bb=0 0 304 307]{Figures/subneteditor.eps.gz}}
}
%end{latexonly}
\begin{htmlonly}
  \newcommand{\subneteditorfig}{%
    \htmladdimg[alt="Sub-net editor"]{../Figures/subneteditor.gif}
  }
\end{htmlonly}

The sub-network editor is activated by pressing a sub-network's editor
button.  The sub-network editor works similarly to the NetEdit window.
All network editing features of the NetEdit window are available in
the sub-network editor---modules can be created, connections can be
made, \etc{} See Figure~\ref{fig:subneteditor}.

\begin{figure}[htb]
  \centering
  \begin{makeimage} \end{makeimage}
  \subneteditorfig
  \caption{\label{fig:subneteditor} Sub-network editor.}
\end{figure}

Sub-networks have input and/or output ports.  These are created in the
sub-network editor.  Sub-network input ports are created by making a
connection between the input port of a module (in the sub-network) and
the top edge of the sub-network editor window.  Sub-network output
ports are created by making a connection between the output port of a
module (in the sub-network) and the bottom edge of the sub-network
editor window.

A sub-network can be renamed by entering its new name into the
\guilabel{Name} text entry widget.

\section{Saving a Sub-network}
\label{sec:savesubnet}

Sub-networks can be saved for reuse in networks and sub-networks.

Sub-networks are saved in the \directory{Subnets} directory of \sr{}'s
build directory, if it exists and is writable).  Otherwise, sub-networks
are saved in the \directory{SCIRun/Subnets} directory of a user's home
directory.

To save a sub-network, press the sub-network's editor
button, which activates the sub-network's editor, then press the
editor's \guibutton{Save} button.

Saved sub-networks are listed as menu items in the \menu{Sub-network}
package menu (which is available in the NetEdit window's \menu{File}
menu and popup menu).

\secref{Loading a Sub-Network}{sec:loadsubnet} gives instructions for
loading sub-networks into networks and sub-networks.

%% Note:  I don't like the terminology used here.  ``creating''
%% subnetworks and ``loading'' subnetworks ought to be ``making'' or
%% ``building'' and ``loading'' ought to be ``instantiating'' or
%% ``creating''.
\section{Loading a Sub-Network}
\label{sec:loadsubnet}

Saved sub-networks are loaded into networks and sub-networks by
selecting their names from the \menu{Sub-network} package menu.
Package menus are accessed from the main window's menu bar or from the
NetEdit frame's pop-up menu.  Unlike other package menus, the
\menu{Sub-network} menu has no category sub-menus.

A sub-network may be loaded one or more times in a network.  Each time
a sub-network is loaded a new instance of it is created in the
network.

\section{Editing and Displaying Module Notes}
\label{sec:modnotes}
\index{annotation of networks}

Module notes allow the user to document the purpose of each module by
attaching notes to each module in the NetEdit window.  Notes can be
displayed or hidden.

To create or edit module notes, choose menu item \guimenuitem{Notes}
from a module's pop-up menu. The note editor dialog can also be
activated by clicking \keyboard{Button1} on notes displayed in the
NetEdit window.  Enter new notes or edit existing notes in the note
editor dialog.

Notes may be hidden, displayed in \dfn{tooltips}, or displayed in the
NetEdit frame.  The note editor provides the following note display
options: \guibutton{Default}, \guibutton{Tooltip}, \guibutton{Top},
\guibutton{Left}, \guibutton{Right}, or \guibutton{Bottom}.  Choose a
display mode by clicking \keyboard{Button1} on the appropriate button.

Choose \guibutton{Tooltip} to display notes as tooltips.  Notes are
displayed only when the mouse pointer hovers over a module.

Choose one of \guibutton{Default}, \guibutton{Top}, \guibutton{Left},
\guibutton{Right}, or \guibutton{Bottom} to display notes in the
NetEdit to the right, top, left, right, or bottom of the module
respectively. 

Choose \guibutton{None} to hide module notes.  Displayed notes can be
hidden by clicking \keyboard{Button2} when the pointer is on the
notes.  Hiding notes does not delete notes.

Click button \guibutton{Text Color} to change the color of notes
displayed in the NetEdit frame.

Click button \guibutton{Cancel} to abort note editing.

Click button \guibutton{Clear} to erase notes.

Click button \guibutton{Done} to accept edited notes.


\section{Viewing a Module's Log}
\label{sec:viewmodslog}

Each module supports a message log. The module writes error messages
or other types of messages to its log.

To display a module's log, choose \guimenuitem{Show Log} from a module's
pop-up menu or click \keyboard{Button1} on a module's message indicator.


\section{Creating a Connection}
\label{sec:connectmods}
\index{connections}

To connect the output (input) port of one module to the input (output)
port of another module, use \keyboard{Button2}.

To make a connection, position the mouse pointer over a module's input
(output) port.  Then perform \keyboard{Button2-Drag} and drag the
mouse pointer toward another module's output (input) port.

When \keyboard{Button2} is pressed, the program shows all valid connections as black
lines.  It also draws one red colored connection, which is the
connection made if the drag is stopped by releasing \keyboard{Button2}.

Make the connection by releasing \keyboard{Button2} when the pointer is over the
desired destination port, or when the red colored connection is the
desired connection.  The connection is drawn using the color
corresponding to the connection's data type.

Users can connect a module's output port to the input ports of one or more
modules by repeating the procedure just described.

\section{Editing and Displaying Connection Notes}
\label{sec:displaynotes}
\index{annotation of networks}

Connection notes allow the user to document the purpose of
each connection by attaching notes to each connection in the NetEdit window.
Notes can be displayed or hidden.

To create or edit connection notes, choose menu item
\guimenuitem{Notes} from a connection's pop-up menu. Enter new notes
or edit existing notes in the note editor dialog.  The note editor
dialog can also be activated by clicking \keyboard{Button1} on notes
displayed in the NetEdit window.

Notes may be hidden, displayed in \dfn{tooltips}, or displayed in the
NetEdit window.  The note editor provides the following note display
options: \guibutton{Default}, \guibutton{Tooltip} and \guibutton{Top}.
Choose a display mode by clicking \keyboard{Button1} on the
appropriate button.

Choose \guibutton{ToolTip} to display notes as tooltips---notes are
displayed only when the mouse pointer hovers over a connection.

Choose \guibutton{Default} or \guibutton{Top} to display notes on top
of the connection.

Choose \guibutton{None} to hide connection notes.  Displayed notes can
also be hidden by clicking \keyboard{Button2} when the pointer is over
the notes.  Hiding notes does not delete notes.

Click button \guibutton{Text Color} to change the color of notes
displayed in the NetEdit frame.

Click button \guibutton{Cancel} to abort note editing.

Click button \guibutton{Clear} to erase notes.

Click button \guibutton{Done} to accept edited notes.


\section{Highlighting Related Connections}
\label{sec:highlightconnect}

To highlight the tree of connections affecting a connection, perform
\keyboard{C-Button1} anywhere on a connection.

To highlight the tree of connections downstream an output port,
perform \keyboard{C-Button1} on the output port.

To highlight the tree of connections upstream an input port,
perform \keyboard{C-Button1} on the input port.

\section{Disabling a Connection}
\label{sec:disableconnect}

To disable a connection, click \keyboard{Button3} on the connection to
bring up the connection menu and select \guimenuitem{Disable}. The
connection appears grey.

Disabling a connection prevents data from flowing through, as if it were
not connected.

\section{Undoing/Redoing a Connection}
\label{sec:undomod}

Type \keyboard{C-z} to undo the last connection creation or deletion.
Undo can be repeated.

Type \keyboard{C-y} to redo the last undone connection creation or
deletion.
 
\section{Deleting a Connection}
\label{sec:deleteconnections}

To delete a connection, choose menu item \guimenuitem{Delete} from a
connection's pop-up menu or click \keyboard{C-Button2} while the
pointer is on a connection.

\section{Executing a Network}
\label{sec:executenet}

``Network Execution'' means one or more modules must be executed in a
coordinated fashion. 
\sr{}'s \dfn{scheduler} manages the coordinated execution of modules.

Note that some modules need to be compiled before they are
executed (see \secref{Dynamic Compilation}{sec:dyncomp}).  Compilation
delays network execution.  Usually, occurs only one time
per module.  After a module is compiled, it does not need to be
compiled again.  Modules change color during compilation.

\subsection{The Basics}

The scheduler is invoked when an event \dfn{triggers} a
module's execution.  The scheduler creates a list of all modules that
must execute in coordination with the triggered module. Modules
\dfn{upstream} (directly or indirectly) from the triggered module are 
put on the execution list if they have not previously executed.
All modules \dfn{downstream} from the triggered module are put
on the execution list.  Once the scheduler determines which modules must be
executed, it executes them (in parallel where possible).

Network execution is mostly transparent.  That is, events that trigger
module execution usually generate automatically. Sometimes,
however, the user must manually
generate a triggering event by choosing the \guimenuitem{Execute} item from a
module's pop-up menu.

\subsection{Details}

Each module executes in its own thread and blocks (waits) until its upstream
modules can supply it with data.  After a module completes its computation,
it sends the results to its downstream modules.  This completes a module's
execution cycle.  The module will not have another chance to receive data from its input ports
and send data to its output ports until some event
puts the  module back on the scheduler's execution list.  

This behavior prevents modules from computing in an iterative fashion 
(sending intermediate results to their downstream modules), because
downstream modules cannot receive results until they are in their
execution cycle. Downstream modules would need to be executed each time the
upstream module posts an intermediate result.


\subsection{Intermediate Results}

Some modules are designed to be used in an iterative fashion. They use
a method called \icode{send\_intermediate} to send the results of each
iteration.  When this method is used, the scheduler (re)executes
downstream modules each time the upstream module posts its next
result.  Downstream modules are able to receive the results of each
iteration as soon as the upstream module sends them.

Modules \module{SolveMatrix} and \module{MatrixSelectVector} (from the
\package{\sr} package and \category{Math} category) are examples of modules
that compute iteratively using the \icode{send\_intermediate} method.

\subsection{Feedback Loops}

Some modules are designed to be used exclusively in feedback
loops. Their output ports can be connected
directly or indirectly to their input ports.  These modules also use the
\icode{send\_intermediate} method.

Examples of feedback modules are \module{DipoleSearch} and
\module{ConductivitySearch} from the \category{Inverse} category of the
\package{BioPSE} package and \module{BuildElemLeadField} from the
\category{LeadField} category of the \package{BioPSE} package.

\section{Documenting a Network}
\label{sec:docnetwork}
\index{annotation of networks}

It is useful to document the function of a network.  A network's note
pad is used for this purpose.  To edit the network's note pad,
select the \guimenuitem{Add Info} item from the main window's \menu{File}
menu.  This displays the network's note pad editor.  The editor
allows the user to write notes on the purpose and use of the network.

\section{Saving a Network}
\label{sec:savenet}

\sr{} can save networks to files.  Network files have an extension of
\filename{.net} (in the past they also had .sr and .uin
extensions).  

To save a network, select item \guimenuitem{Save} from the main window's
\menu{File} menu.  A file browser dialog prompts for the
name and location of the network file.

If changes are made to an existing network, \guimenuitem{Save} saves
changes made to an existing net file.

Save an existing network under a new name using the \guimenuitem{Save
  As...} menu item.  A file browser dialog prompts for the new name of
the network file.  Subsequent uses of \guimenuitem{Save} saves changes
to the newly created file.

The position and size of open Viewer windows are saved to
the network file.

Network files are  \dfna{Tool Command Language}{TCL} scripts.
These files can be edited, however, reasons for doing so are beyond
the scope of this guide.

\section{Loading a Network}
\label{sec:opennet}

To load a network file, select the \guimenuitem{Load} item from the main
window's \menu{File} menu.   A file browser dialog prompts for the
name and location of the network file.  The loaded network replaces an
existing network in the NetEdit frame.

\section{Inserting a Network}
\label{sec:insertnetwork}

To avoid merging networks, select the
\guimenuitem{Insert} item from the main window's \menu{File} menu. This
option allows the user to place one \sr{} network next to another,
avoiding overlap.  A file browser dialog prompts the user for the name and
location of the network file.

The new network is inserted into the upper left corner of the
NetEdit frame.  If a network of modules already exists in the NetEdit
frame, the \guimenuitem{Insert} command places a new network to the
immediate right of the existing network.

\section{Clearing a Network}
\label{sec:clearnetwork}

To remove all modules and connections from the 
NetEdit frame, select the \guimenuitem{Clear} item from the main window's
\menu{File} menu.  A text box appears, confirming whether the
user wants to proceed with or cancel the clearing operation.

\section{Navigating a Network}
\label{sec:navnetwork}

A complex network may not be entirely visible in the NetEdit frame.
Use the NetEdit frame's scroll bars or the network view tool to view
complex networks.

The Global View Frame \index{global view frame}shows the 
entire ``network world.''  The small
rectangular region (outlined in black) in the Global View Frame is the
network view tool and a window on the network world. The position of
the view tool determines the part of the network visible in the
NetEdit frame.  To view other parts of the network, press
\keyboard{Button1} while the pointer is anywhere in the Global View
Frame---this moves the tool to the location of the pointer -- then
drag the tool to the new location.


\section{The \sr{} Shell}
\label{sec:termapp}

After starting, \sr{} runs a shell-like application in the terminal
window. This shell displays the prompt
\screen{scirun\ra} in the terminal window.This program is a
\dfna{Tool Command Language}{TCL} shell program extended with
\sr{} specific commands.

It is possible to type \tcl{} \sr{} commands at the prompt.  For
instance, to load a network type \keyboard{source 
\ptext{network file name}}.  This has the same effect as the \menu{File}
menu's \guimenuitem{Load} command.


% -*-latex-*-
% Filename: viewer.tex
% Author: Rob MacLeod and Dave Weinstein
%
% Last update:Tue Mar 13 19:08:17 MST 2001 by Dave Weinstein
%    - created
%    - DMW: proof-reading and minor corrections
% Last update: Tue Mar 20 23:12:33 2001 by Rob MacLeod
%    - fixing some image size problems and getting it ready for show and
%    tell at doc group meeting
% Last update: Fri Mar 23 00:51:14 2001 by Rob MacLeod
%    - got a working version to publish in HTML too
%
%%%%%%%%%%%%%%%%%%%%%%%%%%%%%%%%%%%%%%%%%%%%%%%%%%%%%%%%%%%%%%%%%%%%%%
%%%%%%%%%%  Figures used in this file %%%%%%%%%%%%%%%%%%%%%%%%%%%%%%%%
%% The basic viewer window
%begin{latexonly}
  \newcommand{\viewerwindow}%
  {\centerline{\epsfig{file=figures/viewwindow.eps.gz,height=4in,
  bbllx=16, bblly=88, bburx=597, bbury=704}}}
%end{latexonly}
\begin{htmlonly}
  \newcommand{\viewerwindow}{%
  \htmladdimg[align=top,width=645,alt="module"]
  {../figures/viewwindow.jpg}}
\end{htmlonly}

%% View of the extended viewer window
%begin{latexonly}
  \newcommand{\extendedwindow}%
  {\centerline{\epsfig{file=figures/viewwindow-ext.eps.gz,height=4in,
  bbllx=16, bblly=310, bburx=600, bbury=703}}}
%end{latexonly}
\begin{htmlonly}
  \newcommand{\extendedwindow}{%
  \htmladdimg[align=top,width=649,alt="extended view window"]
  {../figures/viewwindow-ext.jpg}}
\end{htmlonly}

%% Gauge widget image
%begin{latexonly}
  \newcommand{\gaugewidget}%
  {\centerline{\epsfig{file=figures/widget-gauge.eps.gz,height=2in,
  bbllx=0, bblly=0, bburx=457, bbury=340}}}
%end{latexonly}
\begin{htmlonly}
  \newcommand{\gaugewidget}{%
  \htmladdimg[align=top,width=458,alt="gaugewidget"]
  {../figures/widget-gauge.gif}}
\end{htmlonly}

%% Frame widget image
%begin{latexonly}
  \newcommand{\framewidget}%
  {\centerline{\epsfig{file=figures/widget-frame.eps.gz,height=2in,
  bbllx=0, bblly=0, bburx=328, bbury=268}}}
%end{latexonly}
\begin{htmlonly}
  \newcommand{\framewidget}{%
  \htmladdimg[align=top,width=329,alt="framewidget"]
  {../figures/widget-frame.gif}}
\end{htmlonly}

%% Box widget image
%begin{latexonly}
  \newcommand{\boxwidget}%
  {\centerline{\epsfig{file=figures/widget-box.eps.gz,height=2in,
  bbllx=0, bblly=0, bburx=458, bbury=342}}}
%end{latexonly}
\begin{htmlonly}
  \newcommand{\boxwidget}{%
  \htmladdimg[align=top,width=459,alt="boxwidget"]
  {../figures/widget-box.gif}}
\end{htmlonly}

%% Ring widget image
%begin{latexonly}
  \newcommand{\ringwidget}%
  {\centerline{\epsfig{file=figures/widget-ring.eps.gz,height=2in,
  bbllx=0, bblly=0, bburx=507, bbury=467}}}
%end{latexonly}
\begin{htmlonly}
  \newcommand{\ringwidget}{%
  \htmladdimg[align=top,width=508,alt="ringwidget"]
  {../figures/widget-ring.gif}}
\end{htmlonly}

%%%%%%%%%%%%%%%%%%%%%%%%%%%%%%%%%%%%%%%%%%%%%%%%%%%%%%%%%%%%%%%%%%%%%%
\newcommand{\graphics}{\emph{Graphics}}

\section{Visualization with the \viewer{}}
\label{sec:viewer}
\index{Viewer@\viewer{}}

This section describes perhaps the most frequently used module of \SR{},
the \viewer{}, which has the task of displaying interactive graphical
output to the computer screen.  You will use the \viewer{} any time you
wish to see a geometry, or spatial data.  More important for the
computational steering (described in \secref{Computational
Steering}{sec:con-steering}), is that the \viewer{} provides access to
many simulation parameters and controls and thus indirectly initiates new
iterations of the simulation steps.

We begin with an overview of the \viewer{} window and its controls, then
describe in detail all the options and variations.

\subsection{Anatomy of the \viewer{} window}
\label{sec:viewer-anatomy} 
\index{Viewer@\viewer{}!anatomy}

The \viewer{} window contains two main areas, the upper portion, called the
\graphics{} window, which displays the graphics, and the lower portion,
where most of the control buttons are.  Figure~\ref{fig:viewwindow}
contains an example of a \viewer{} window. In the \graphics{}
window, control is mostly by means of the mouse, mouse buttons, and various
modifier keys (shift/control/alt).  In the lower window are a lot of
buttons and sliders, the function of which will become clear when you read
this manual.

\begin{figure}[htb]
  \begin{makeimage}
  \end{makeimage}
  \viewerwindow
%    \framebox{\parbox[3in]{\columnwidth}{The\dotfill Figure\\
%    \vspace{2in}\\
%    With some \dotfill dummy text}}
  \caption{\label{fig:viewwindow} The default \viewer{} window in \SR{}}
\end{figure}


First, try out the controls for the \graphics{} window by moving the mouse
to the center of the viewer window and clicking and holding the left button
and then dragging the mouse.  The objects should translate along with the
mouse.  Do the same operation with the middle mouse button and the objects
will rotate around a point in the center of the display.  The right mouse
button controls the scale of the display, zooming in when the mouse moves
downward or to the right.  See \secref{Mouse Control in the
Viewer}{sec:view-mouse} for all the gory details on mouse control.

The controls visible along the bottom of the \viewer{} window set some basic
configurations as follows:
%
\begin{description}
  \item [\fbox{Autoview:} ] restores the display back to a default
        condition, very useful when some combination of settings results in
        the objects disappearing from the view window.
  \item [\fbox{Set Home View:} ] captures the setting of the current view
        so you 
        can return to it later by clicking the ``Go home'' button.
  \item [\fbox{Go home:} ] restore the current home view.
  \item [\fbox{Views:} ] lists a number of standard viewing angles and
        orientations.  The view directions align with the Cartesian axes
        of the objects and the ``Up vector'' choice sets the orientation of
        the objects when viewed along the selected axis.
\end{description}

In the left corner of the control panel of the \viewer{} window are
performance indicators that document the current rendering speed for the
display.  The better the graphics performance of the workstation you
have, the higher the drawing rate.

In the lower right corner of the \viewer{} window is a small plus sign
(``+'').  Clicking on this reveals the extended control panel with controls
that we will describe in detail in \secref{Extended control window}
{sec:view-control}.



\subsubsection{Menus}
\index{Viewer@\viewer{}!menus}

At the top of the \viewer{} window are two pull-down menus.
\begin{description}
  \item [\fbox{Edit:} ] provides access to controls for the background
        color for the window, as well as the clipping planes (requires the
        ``Use Clip'' control to be selected in the extended controls
        described in \secref{Extended control window}{sec:view-control}).
  \item [\fbox{Visual:} ] allows you to select between different graphics
        hardware settings that are available on your workstation.  The list
        is ordered heuristically from most to least useful.
\end{description}

\subsection{Mouse control in the \viewer{} window}
\label{sec:view-mouse} 
\index{Viewer@\viewer{}!mouse controls}

The mouse controls within \SR{} are extensive and flexible.  The resulting
action depends on the choice of mouse button, any simultaneous control
keys, and the way the mouse moves.  The description in
Tables~\ref{tab:view-mouse} and~\ref{tab:view-unicam} below may seem overly
complicated at first, but with a little playing, it becomes intuitive
(another way of saying you will learn it if you use it enough).

\begin{table}[htb]
\begin{center}
  \begin{tabular}{|l|l|p{5in}|} \hline
    \multicolumn{3}{|c|}{\large Mouse Controls}\\ \hline \hline 
    \multicolumn{1}{|c|}{Control Key} & 
    \multicolumn{1}{|c|}{Button} & 
    \multicolumn{1}{|c|}{Action}\\ \hline
None & Left & Translate scene \\
     & Middle & \begin{raggedleft} Rotate scene about its center on an arc
    ball that surrounds it; rotation direction is a function of the
    initial mouse location so try different sites and note the
    response. \end{raggedleft}\\  
     & Right & Zoom or scale scene (downwards and to the right increases
     size, upwards or to the left decreases size) \\ \hline
Shift & Left & Select and move a widget in the display \\
      & Middle & Toggle through the modes for a widget \\
      & Right & Pop up a widget information window \\ \hline
Control & Left & Translate in the Z-direction, \ie{} zoom in and out of the
    screen (down moves closer, up further away).  Moving left and
    right increases the ``throttle'' of the Z-direction motion.  If
    the cursor is over a point on an object when clicked, this point
    becomes the center of the screen for translation.\\ 
      & Middle & Rotate the camera view about the eye point (using arcball
    motion). \\ 
      & Right & Unicam movement (see next table)\\ \hline
\end{tabular}
\caption{\label{tab:view-mouse} Mouse controls for the \viewer{}}
\end{center}
\end{table}

\bigskip

\begin{table}
\begin{center}
\begin{tabular}{|l|l|p{3in}|} \hline
    \multicolumn{3}{|c|}{\large Unicam movement (Control key and right mouse
    button} \\ \hline \hline
    \multicolumn{1}{|c|}{Initial mouse location} & 
    \multicolumn{1}{|c|}{Action} & \\ 
    \hline
    Near edge of display & Rotate objects on the arc ball & \\
    Near the objects & Following behavior: & \\
    \hline
    & \multicolumn{1}{|c|}{Initial mouse movement} & 
    \multicolumn{1}{|c|}{Action}\\ \hline
    & Horizontal & Pan objects \\ 
    & Vertical & Zoom and pan: down = zoom in, up = zoom
    out, left and right= pan left and right) \\
    & None & Set rotation point for subsequent arc ball rotation.\\
    \hline
\end{tabular}
\caption{\label{tab:view-unicam} Autocam mouse controls in the \viewer{}}
\end{center}
\end{table}



\subsection{Extended control window}
\label{sec:view-control} 
\index{Viewer@\viewer{}!extended controls}

Click on the ``+'' sign in the lower right corner of the default
\viewer{} window, and the window expands to reveal an extended panel of
control buttons, as shown in Figure~\ref{fig:extviewwindow}.  Click on the
``-'' sign that now replaces the ``+'' and this extended panel disappears
again.  Here we describe the control options available in the extended
control window.

\begin{figure}[htb]
  \begin{makeimage}
  \end{makeimage}
  \extendedwindow
%    \framebox{\parbox[3in]{\columnwidth}{The\dotfill Figure\\
%    \vspace{2in}\\
%    With some \dotfill dummy text}}
  \caption{\label{fig:extviewwindow} The lower portion of extended
    \viewer{} window in \SR{}} 
\end{figure}


\subsubsection{Object selector}

The lower portion of the extended \viewer{} window is divided into three
columns and the middle of these contains a list of all the objects in the
display.  If the list becomes long enough, a scroll bar on the left
hand side controls which are visible.  For each entry in the list, we have
the following controls, reading from left to right:

\begin{itemize}
  \item At the left end of each of the
        objects in the list is box that displays red when that object is
        selected.  The \viewer{} window only displays those objects that
        are selected.
  \item Next comes the name of the object.
  \item The \fbox{Shading} control box that comes next determines the
        shading options that will be used for rendering the object.
        Options include: Lighting, BBox, Fog, Use Clip, Back Cull, and
        Display List (for descriptions, see
        \secref{Rendering controls}{sec:view-rendering}).
  \item As the right end of each entry is the lighting control, initially
        marked \fbox{Default}.  In the Default setting, the common
        rendering controls described in \secref{Rendering
        controls}{sec:view-rendering} below apply.  Clicking this box
        reveals a set of options that will apply only to this object that
        include Wire, Flat and Gouraud.
\end{itemize}


\subsubsection{Rendering controls}
\label{sec:view-rendering} 
\index{Viewer@\viewer{}!rendering}

The left column of the extended \viewer{} window contains controls that
apply to all of the selected objects with ``Default'' lighting selected.
Those without the Default setting will use their own, object specific
settings, as described in the previous section.  The lighting and shading
options available are:
%
\begin{description}
  \item [Lighting: ] toggles whether or not the \viewer{} applies lighting
        to the display.  Objects without lighting have a constant
        color.
  \item [Fog: ] draws objects with variable intensity based on their
        distance from the user, also known as ``depth cueing''.  Close
        objects appear brighter while more remote objects fade gradually
        into the background as a function of distance from the front.
  \item [BBox: ] toggles whether the \viewer{} draws the selected objects
        in full detail or as a simple bounding box.
  \item [Use Clip: ] applies up to six clipping planes to the display.
        To control the clipping plane locations, use the
        ``Edit -> Clipping Planes'' menu at the top.
  \item [Back Cull: ] displays only the forward facing facets of any surface
        objects in the display.
  \item [Display List: ] cache the list of objects to be displayed; this
        option accelerates rendering when the content of the display does
        not change. 
  \item [Shading: ] selects the type of shading for objects from the
        following options:
        \begin{description}
          \item [Wire: ] show only the wire mesh of objects.
          \item [Flat: ] draw each facet with a constant color.
          \item [Gouraud: ] linearly interpolate the color across facets. 
        \end{description}
\end{description}

The right hand column of the extended \viewer{} window contains controls
for displaying the axes and creating stereoscopic rendering.  

\paragraph{Stereo viewing: } requires hardware LCD glasses synchronized
with the display so that visibility for each eye coincides with the
display of the appropriate view.  The ``Fusion Scale'' control provides a
means of setting the eye separation and thus setting the view that is most
suited to facial anatomy and distance from the screen.

\subsubsection{Making movies}
\label{sec:view-movies} 
\index{movies}

The \viewer{} window in \SR{} has simple controls for capturing sequences
of images into animations or movies.  Here we describe how this works.

In the left column of the extended \viewer{} window are controls for
selecting movie type and then initiating and stopping the acquisition of
individual frames in the movie.

\SR{} sends a frame to the movie after each ``redraw'' operation, \ie{}
each time anything moves in the display or any visualization parameter
changes.  If the MPEG package is available (See the
\htmladdnormallink{Installation Manual}{\installguideurl} for
details) then an option will be available for saving the animations as MPEG
movies.

There is also a button that forces the size of the graphical window to be
352x240 pixels in size, which is a standard format well suited to MPEG.

\subsection{Control widgets}
\label{sec:view-widgets} 

While the \hyperref{mouse controls}{mouse controls in
Section}{}{sec:view-mouse} describe many ways to interact with the contents
of the \viewer{}, SR{} also supports some powerful display widgets.
Examples of widgets capabilities include managing cutting surfaces colored
according to the local data values, displaying streamlines in vector
fields, or selecting sub-volumes within the display area for further
manipulation. 
 
We have tried to make interacting with these widgets as consistent as
possible so that, for example, controlling parameters is usually by clicking
and dragging on either a cylindrical ``collar'' or a sphere element of the
widget.  The original design of these modules was by James Purcifal
%%\cite{??}. 
Note that a single widget may have more than one purpose
depending on the context in which it exists.

In this section, we describe the widgets available within \SR{} and \PSE{}.
The same widget may, for example, select a clipping or a display plane
through a three-dimensional object but may also set the seed points for a
streamline module.  

\subsubsection{Gauge Widget}
\label{sec:view-gaugewidget} 

\begin{figure}[htb]
  \begin{makeimage}
  \end{makeimage}
  \gaugewidget
%    \framebox{\parbox[3in]{\columnwidth}{The\dotfill Figure\\
%    \vspace{2in}\\
%    With some \dotfill dummy text}}
  \caption{\label{fig:gaugewidget} The gauge widget for setting location and
    density of seed points}
\end{figure}

\paragraph{Appearance: } The Gauge Widget consists of two spheres (A)
connected by a cylinder (B) with a small slider collar (C) on the cylinder.
There are also small resize cylinders extending from the spheres (D).

\paragraph{Purpose:} The primary use of the Gauge Widget is to set the
location and density of streamlines emerging from the long cylinder.  It
may also be used as a more general purpose three-dimensional slider or as a
source for a stream surface. 

\paragraph{Controls: } Clicking and dragging either sphere causes the
entire widget to move in space, rotating about the other sphere and
following along behind the selected sphere.  Dragging either of the resize
cylinders cases the size of the widget to change and dragging any point on
the main cylinder moves the whole widget without any change in orientation.
Dragging the slider collar changes the associated value, typically the
density of seed points for a streamline source.

\subsubsection{Frame Widget}
\label{sec:view-framewidget} 

\begin{figure}[htb]
  \begin{makeimage}
  \end{makeimage}
  \framewidget
%    \framebox{\parbox[3in]{\columnwidth}{The\dotfill Figure\\
%    \vspace{2in}\\
%    With some \dotfill dummy text}}
  \caption{\label{fig:framewidget} The frame widget for selecting
    cutting/projection planes}
\end{figure}


\paragraph{Appearance: } The Frame Widget consists of four cylinders
connected in a rectangle.  In the middle of each of the cylinders there is
a sphere (B), from which a resize cylinder extends (C).

\paragraph{Purpose:} The primary uses of the Frame Widget is for image
plane definition, for defining stream volumes, and as a "tie dye" as with
the Ring Widget described in \secref{Ring Widget}{sec:view-ringwidget}.

\paragraph{Controls: } Clicking and ragging a sphere on the widget will
cause the widget to rotate about it center; dragging on a resize cylinder
will move the associated edge and this extend or contract the rectangle.
Dragging any of the cylinder will drag the entire widget through space.


\subsubsection{Box Widget}
\label{sec:view-boxwidget} 

\begin{figure}[htb]
  \begin{makeimage}
  \end{makeimage}
  \boxwidget
%    \framebox{\parbox[3in]{\columnwidth}{The\dotfill Figure\\
%    \vspace{2in}\\
%    With some \dotfill dummy text}}
  \caption{\label{fig:boxwidget} The boxwidget for selecting sub-volumes}
\end{figure}

\paragraph{Appearance: } The Box Widget consists of twelve cylinders (A)
connected in a hexahedral box (three-dimensional rectangle) with cylinders
indicating on the edges of the box (B).  In the middle of each face of the
box is a sphere with a cylinder protruding from it (C) that provide resize
control.

\paragraph{Purpose:} The primary use of the Box Widget is to select a
subvolume of the workspace for further manipulation (\eg{} volume
rendering, isosurfaces, streamlines, mesh adaption) where the faces of the
widget act as orthogonal clipping planes.

\paragraph{Controls: } Clicking and dragging on one of the spheres rotates
the widget about its center without changing the position of the center.
Clicking on and dragging any resize handle
% What do these look like??
causes the associated face to extend without changing its orientation.
Dragging a cylinder causes the entire widget to move without changing its
orientation.

\subsubsection{Ring Widget}
\label{sec:view-ringwidget} 

\begin{figure}[htb]
  \begin{makeimage}
  \end{makeimage}
  \ringwidget
%    \framebox{\parbox[3in]{\columnwidth}{The\dotfill Figure\\
%    \vspace{2in}\\
%    With some \dotfill dummy text}}
  \caption{\label{fig:ringwidget} The ring widget for selecting
    cutting/projection planes}
\end{figure}


\paragraph{Appearance: } The Ring Widget consists of a ring (A) with four
embedded spheres (B), each with a resize cylindrical attached (D).  Between
two of the spheres is a sliding collar (C).  One of the resize cylinders
has a special material property (typically a different color from the other
cylinders) to indicate that it is the ``halfway point'' for the slider (E).

\paragraph{Purpose:} The primary use of the Ring Widget is to set the
density of streamlines emerging from the ring---the ring serves as a set of
seed points from which will emerge streamlines.  The Ring Widget can also
serve as a three-dimensional angle gauge, as a source for multiple
streamlines throughout its surface, as a source for a stream surface from
the outer ring, and as a source for a stream volume.  Another use is as a
color sheet, or ``tie dye'', in which the surface is colored as a function of
the scalar value of the field at each point.

\paragraph{Controls: } Clicking and dragging the slider collar along the
ring changes the density of the seed points or some other related
parameter.  Dragging the spheres controls the orientation of the Ring
Widget, while moving the resize cylinders changes the radius of the Ring
Widget about its center.  Dragging any other point on the ring moves the
ring in space without changing its radius or orientation.


%%% Local Variables: 
%%% mode: latex
%%% TeX-master: t
%%% End: 


\input{ui}

%%% Local Variables: 
%%% mode: latex
%%% TeX-master: "usersguide"
%%% End: 
