
\section{MPMICE} \label{Sec:MPMICE}

\subsection{Introduction}

\subsection{Theory - Algorithm Description}

\subsection{Uintah Specification}

\subsubsection{ICE}

%\subsubsection{Basic Inputs}
%\subsubsection{Physical Constants}
%\subsubsection{Material Properties}
%\subsubsection{Equation of State}
%\subsubsection{Exchange Properties}
%\subsubsection{BoundaryConditions}
%\subsubsection{Solvers}


\subsection{MPM}

%\subsubsection{Basic Inputs}
%\subsubsection{Physical Constants}
%\subsubsection{Material Properties}
%\subsubsection{Constitutive Models}
%\subsubsection{Contact}
%\subsubsection{BoundaryConditions}
%\subsubsection{Physical Boundary Conditions}

\subsection{Examples}

\section*{\center Mach 2 Wedge}
\subsection*{\underline{Problem Description}}
This is a simulation of a symmetric $20^o$ wedge traveling through initially
quiescent air at Mach 2.0.  A shock forms at the leading edge of the
wedge and an expansion fan over its top.  Consultation of oblique shock
tables, e.g.~\cite{Saad} (pp.308-309) reveals that the angle of the leading
shock compares quite well with the expected value.  In addition, this
simulation demonstrates a few other useful features of the fluid-structure
interaction capability.  In this case, the structure is rigid, and as
such, essentially provides a boundary condition to the compressible flow
calculation.  Furthermore, the geometry of the wedge is described via a
triangulated surface, rather than the geometric primitives usually used.
This allows the user to study flow around arbitrarily complex objects,
without the difficulty of generating a body fitted mesh around that object.

\subsection*{\underline{Simulation Specifics}}
\begin{description}
\item [Component used:] \hfill rmpmice (Rigid MPM-ICE)
\item [Input file name:] \hfill Mach2Wedge.ups
\item [Command used to run input file:]\hfill sus Mach2Wedge.ups
(Note: The files wedge40.pts and wedge40.tri must also be copied to
the same directory as sus.)

\item [Simulation Domain:]\hfill    0.25 x 0.0375 x 0.001 m

\item [Cell Spacing:]\hfill \\
.0005 x .0005 x .001 m (Level 0)

\item [Example Runtimes:] \hfill \\
 20 minutes   (1 processor, 3.16 GHz Xeon)\\

\item [Physical time simulated:] \hfill 0.3 milliseconds

\item [Associated visit session:] \hfill M2wedge.session

\end{description}

\section*{\underline{Results}}

Figure~\ref{figwedge} shows a snapshot of the simulation.  Contour
plot depicts pressure and reflects the presence of a leading shock
and an expansion fan.
\begin{figure}
  \center
  \includegraphics[scale=.4]{M2wedge.png}

  \caption{$20^o$ wedge moving at Mach 2.0 through initially stationary
air.  Contour plot depicts pressure.}
  \label{figwedge}
\end{figure}
\newpage
%
%__________________________________
\section*{\center Cylinder in a Crossflow}
\subsection*{\underline{Problem Description}}
In this example the domain is initially filled with air moving at a uniform velocity of $0.03m/s$  A ridgid cylinder $O.D. = 0.02m$ is placed $0.1m$ from the inlet and a passive scalar is injected into the domain through a $0.002m$ hole on in the inlet boundary of the domain.  A velocity perturbation is placed upstream of the cylinder to produce an instablity that will help trigger the onset of the K\'arm\'an vortex street.
%
\subsection*{\underline{Simulation Specifics}}
\begin{description}
\item [Component used:] \hfill rmpmice (Rigid MPM-ICE)
\item [Input file name:] \hfill \TT{cylinderCrossFlow.ups}
\item [Command used to run input file:]\hfill \\
\TT{mpirun -np 6  sus inputs/UintahRelease/MPMICE/cylinderCrossFlow.ups}

\item [Simulation Domain:]\hfill    0.3 x 0.15 x 0.001 m

\item [Cell Spacing:]\hfill \\
.00015 x .001 x .001 m (Level 0)

\item [Example Runtimes:] \hfill \\
 7ish hrs   (6 processor, 3.16 GHz Xeon)\\

\item [Physical time simulated:] \hfill 60 seconds

\item [Associated visit session:] \hfill cyl\_crossFlow.session

\end{description}

\section*{\underline{Results}}

Figure~\ref{fig:cylCrossFlow} shows a snapshot of the simulation at time $t=60sec$.  The contour
plot of the passive scalar shows the K\'arm\'an vortex street behind the cylinder at $Re=700$.
\begin{figure}
  \center
  \includegraphics[scale=.5]{cylCrossFlow.png}
  \caption{Flow over a stationary cylinder, $Re=700$, a passive scalar is used as a flow marker}
  \label{fig:cylCrossFlow}
\end{figure}
%
A movie of the results is located at
\begin{Verbatim}[fontsize=\footnotesize]
  movies/cyl_crossFlow.mpg
\end{Verbatim}
\newpage
%
%__________________________________
\subsection{References}
