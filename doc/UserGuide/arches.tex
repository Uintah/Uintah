%% LyX 1.6.2 created this file.  For more info, see http://www.lyx.org/.
%% Do not edit unless you really know what you are doing.
%\documentclass[english]{article}
%\usepackage[T1]{fontenc}
%\usepackage[latin9]{inputenc}

%\usepackage{babel}

%\begin{document}

\chapter{Arches}

\section{Introduction}
% please fix....
The ARCHES component was initially designed for predicting the potential hazard of an explosive device immersed in or near a pool fire of transportation fuel.  Since then, this component has been extended to solve many industrially relevant problems such as industrial flares, oxy-coal combustion processes, and fuel gasification.  

Given the wide range of length and time scales that are present in these examples, ARCHES utilizes models for bridging the molecular (micro) scales to the full, large (macro) scales.  ...something here about philosophy of our approach...

The ARCHES component solves the conservative, finite volume, low-mach formulation of the Navier-Stokes equation with a pressure projection that includes the effect of variable density, reaction, and many modes of heat transfer including radiation.  etc etc...

\section{Governing Equations}
%
The essential governing equations for the Arches component, written
in finite volume form, include the mass balance, momentum balance,
mixture fraction balance, and energy balance equations. Using a bold-face
symbol to represent a vector quantity, the equations are: 
\begin{enumerate}
\item The mass balance, \begin{equation}
\int_{V}\frac{\partial\rho}{\partial t}dV+\oint_{S}\rho\mathbf{u}\cdot d\mathbf{S}=0\;,\label{eqn:mass_balance}\end{equation}
 where $\rho$ is density and $\mathbf{u}$ is the velocity vector. 
\item The momentum balance, \begin{equation}
\int_{V}\frac{\partial\rho\mathbf{u}}{\partial t}dV+\oint_{S}\rho\mathbf{uu}\cdot d\mathbf{S}=\oint_{S}\tau\cdot d\mathbf{S}-\int_{V}\nabla pdV+\int_{V}\rho\mathbf{g}dV\;,\label{eqn:mom_balance}\end{equation}
 where $\tau$ is the deviatoric stress tensor defined as $\tau_{ij}=2\mu S_{ij}-\frac{2}{3}\mu\frac{\partial u_{k}}{\partial x_{k}}\delta_{ij}$,
the second isotropic term in $\tau_{ij}$ is absorbed into the pressure
projection for the current low-Mach scheme, and $S_{ij}=\frac{1}{2}\left(\frac{\partial u_{i}}{\partial x_{j}}+\frac{\partial u_{j}}{\partial x_{i}}\right)$.
Also in Equation \ref{eqn:mom_balance}, $\mathbf{g}$ is the gravitational
body force and $p$ is pressure. 
\item The mixture fraction balance, \begin{equation}
\int_{V}\frac{\partial\rho f}{\partial t}dV+\oint_{S}\rho\mathbf{u}f\cdot d\mathbf{S}=\oint_{S}D\nabla f\cdot d\mathbf{S}\;,\label{eqn:species_balance}\end{equation}
 where $f$ is the mixture fraction and a Fick's law form of the diffusion
term assuming equal diffusivities results in a single diffusion coefficient,
$D$. 
\item The thermal energy balance, \begin{equation}
\int_{V}\frac{\partial\rho h}{\partial t}dV+\oint_{S}\rho\mathbf{u}h\cdot d\mathbf{S}=\oint_{S}k\nabla h\cdot d\mathbf{S}-\oint_{S}q\cdot d\mathbf{S}\;,\label{eqn:heat_balance}\end{equation}
 where $h$ is the sum of the chemical plus sensible enthalpy, $q$
is the radiative flux, a Fourier's law form of the conduction term
is used with a diffusion coefficient, $k$, and the pressure term
is neglected. 
\end{enumerate}
These equations are solved in an LES context, meaning filters are
applied to the equations. Here, we use Favre filtering, defined as
\[
\overline{\phi}=\frac{\overline{\rho\phi}}{\overline{\rho}},\]
 to isolate the density in the filtered equations. The filtering operations
result in the classic turbulence closure problem and thus models are
required. 

Consider a control volume, $V$, with surface area $S$.  Because the equations will be  solved on a computational grid, one can safely assume that the the control volume has $N$ faces, where unique faces are identified with their index, $k$.  The discussion is further simplified by only considering cubic volumes with length $h$.  
Given the cubic control volume, a surface-filtered field for a variable $\phi$ is defined as $\overline{\phi}^{(j)}(\mathbf{x})$, where the variable is filtered on a plane in the $x_j$ orthogonal direction.  Then, for any surface, $k$, the field is sampled at the face centered location.  For example, if $j=1$, the surface-filtered quantity is
%
\begin{equation}
\overline{\phi}^{2d, (1)}(\mathbf x) = \frac{1}{h^2} \int_{x_2 - h/2}^{x_2 + h/2}  \int_{x_3 - h/2}^{x_3 + h/2} \phi(\mathbf x') dx_2' dx_3' \; .
\end{equation} 
%
The volume average follows as
%
\begin{equation}
\overline{\phi}^{3d} (\mathbf x) = \frac{1}{h^3} \int_{x_1 - h/2}^{x_1 + h/2} \int_{x_2 - h/2}^{x_2 + h/2}  \int_{x_3 - h/2}^{x_3 + h/2} \phi(\mathbf x') dx_1' dx_2' dx_3' \; .
\end{equation}
%
The bars over the variable, $\phi$, are labeled with `2d' and `3d' superscripts to distinguish between the two filters.  Pope \cite{Pope179} identifies the proceeding definitions as using the ``anisotropic box" filter kernel where the resultant variables are simply averages over the intervals $x_j - \frac{1}{2}h < x_j' < x_j + \frac{1}{2}h$. 

For convenience in isolating density in the filtered equations, a Favre-filtered quantity is defined for an arbitrary variable, $\varphi$, as 
%
\begin{equation}\label{eqn:favre_surf}
\widetilde{\varphi}^{2d} \equiv \frac{\overline{\rho \varphi}^{2d}}{\overline{\rho}^{2d}} \; ,
\end{equation}
%
and
%
\begin{equation}\label{eqn:favre_vol}
\widetilde{\varphi}^{3d} \equiv \frac{\overline{\rho \varphi}^{3d}}{\overline{\rho}^{3d}} \;.
\end{equation}
%
Because the 2d and 3d filters are explicitly defined, this convention is slightly different than what is normally observed in the literature.  Most literature, however, derives the filtered equations from the finite difference equations rather than the finite volume equations.  Thus, using $\overline{\rho}^{2d}$ and $\overline{\rho}^{3d}$ in Equations \ref{eqn:favre_surf} and \ref{eqn:favre_vol} to stress surface and volume filtered densities are appropriate for the present discussion.

The previous definitions are applied to the integral forms of the governing equations to obtain the Favre-filtered LES equations.  Nevertheless, there are terms in the Favre-filtered equations that cannot be solved.  These include the surface filtered convection of momentum, $\widetilde{u_i u}^{2d}_j$, the surface filtered convection of mixture fraction, $\widetilde{u_j f}^{2d}$, and the surface filtered convection of enthalpy, $\widetilde{u_j h}^{2d}$.  

For the filtered velocity product, $\overline{\rho}^{2d} \; \widetilde{ u_i u}^{2d}_j$, a subgrid stress tensor is defined as, 
%
\begin{equation}\label{eqn:tau_sgs}
\tau^{sgs}_{ij} = \widetilde{u_i u}^{2d}_j - \widetilde{u}^{2d}_i \widetilde{u}^{2d}_j \; .
\end{equation}
%
Similarly, subgrid diffusion terms are defined for mixture fraction and enthalpy, 
%
\begin{eqnarray}
\mathcal{J}^{f}  = \widetilde{u_j f}^{2d} - \widetilde{u}^{2d}_j \widetilde{f}^{2d} \;, \\
\mathcal{J}^{h} = \widetilde{u_j h}^{2d} - \widetilde{u}^{2d}_j \widetilde{h}^{2d} \; .\\
\end{eqnarray}

Using these definitions, the final form of the Favre-filtered equations is
%
\begin{enumerate}
\item The filtered mass balance, 
%
\begin{equation}\label{eqn:filtered_mass_balance}
\frac{d}{d t} \left(\widetilde{\rho}^{3d} \right)   + \frac{S_k}{V} n_{kj} \left( \overline{\rho}^{2d} \; \widetilde{u}^{2d}_j \right) = 0 \; .
\end{equation}
%
\item The filtered momentum balance, 
%
\begin{equation}\label{eqn:filtered_mom_balance}
\frac{d}{d t} \left( \overline{\rho}^{3d} \; \widetilde{u}^{3d}_i \right) = \frac{S_{k}}{V} n_{kj}\left( -\overline{\rho}^{2d} \; \widetilde{u}^{2d}_i\widetilde{u}^{2d}_j + \overline{\tau}^{2d}_{ij} + \tau^{sgs}_{ij} - \overline{p}^{2d} \delta_{ij} \right) + \overline{\rho}^{3d} \; g_i  \; .
\end{equation}
%
\item The filtered mixture fraction balance,
%
\begin{equation}\label{eqn:filtered_mixfrac_balance}
\frac{d}{d t} \left( \overline{\rho}^{3d} \; \widetilde{f}^{3d} \right) = \frac{S_k}{V} n_{kj} \left( -\overline{\rho}^{2d} \; \widetilde{u}^{2d}_j \widetilde{f}^{2d} + D \nabla \overline{f}^{2d} + \mathcal{J}^{f} \right) \; .
\end{equation}
%
%
\item The filtered thermal energy balance, 
%
\begin{equation}\label{eqn:filtered_heat_balance}
\frac{d}{d t} \left( \overline{\rho}^{3d} \; \widetilde{h}^{3d} \right) = \frac{S_{k}}{V} n_{kj} \left( -\overline{\rho}^{2d} \; \widetilde{u_j}^{2d}\widetilde{h}^{2d} + k \nabla \overline{h}^{2d}  - \overline{q}^{2d}  + \mathcal{J}^h \right) \; .
\end{equation}
%
\end{enumerate}
The subgrid momentum stress, $\tau_{ij}^{sgs}$, the subgrid mixture fraction dissipation, $\mathcal{J}^{f}$, and the subgrid heat dissipation, $\mathcal{J}^{h}$, contain the unresolved or subgrid action of the turbulence on the transported quantities.  Since these terms arise from definitions, models are introduced to include the subgrid effects that they represent.  These models are discussed next.

\subsection{Subgrid Turbulence Models}
%
The construction of both $\mathcal{J}^f$ and $\mathcal{J}^{h}$ is relatively straight forward. Invoking an ``eddy-viscosity" modeling concept, the subgrid transport due to turbulent advection is treated as an enhanced diffusion term for the unclosed terms listed above.   That is, the subgrid mixture fraction dissipation and subgrid enthalpy dissipation are respectively written as, 
%
\begin{equation}
\mathcal{J}^{f} = D_{t} \frac{\partial \overline{f}^{2d}}{\partial x_j} \; , 
\end{equation}
%
and 
%
\begin{equation}
\mathcal{J}^{h} = k_{t} \frac{\partial \overline{h}^{2d}}{\partial x_j} \; .
\end{equation}
%
To model $D_{t}$ and $k_{t}$, constant turbulent Schmidt ($Sc_t$),
%
\begin{equation}\label{eqn:subgrid_mixfrac}
Sc_{t} = \frac{1}{\rho} \frac{\mu_t}{D_t} \; ,
\end{equation}
%
and Prandlt ($Pr_t$), 
%
\begin{equation}\label{eqn:subgrid_enthalpy}
Pr_{t} = \frac{1}{\rho} \frac{\mu_t}{k_t} \; ,
\end{equation}
%
numbers are assumed with where $\mu_t$ is a turbulent viscosity.  Following Pitsch and Steiner \cite{pitsch2000}, the values of the turbulent Schmidt and Prandlt number are taken as $Sc_t = Pr_t = 0.4$, which is consistent with a unity Lewis number assumption.  

For the subgrid scale stress tensor, $\tau^{sgs}_{ij}$, two common LES turbulence closure models are the constant coefficient Smagorinsky model \cite{Smagorinsky178} and the dynamic coefficient Smagorinsky model \cite{Moin158}.  As with the scalar subgrid modeling terms, the eddy viscosity model is again invoked for $\tau^{sgs}_{ij}$.  Defining the deviatoric subgrid stress tensor as, 
%
\begin{equation}
\tau^{d, sgs}_{ij} = \tau^{sgs}_{ij} - \frac{1}{3}\tau^{sgs}_{kk} \delta_{ij}, 
\end{equation}
%
the subgrid stress is taken as,
%
\begin{equation}
\tau_{ij}^{d, sgs} \approx -2 \nu_t \overline{S}_{ij} = -2 (C_s \Delta)^2 |\overline{S}|\overline{S}_{ij} \; , 
\end{equation}
%
where $\Delta$ is the filter width, $\nu_t$ is the eddy viscosity and $|\overline{S}| \equiv (2\overline{S}_{ij}\overline{S}_{ij})^{1/2}$.  For the Smagorinsky model,  $C_s \approx 2$  depending on the filter type, numerical method, and flow configuration \cite{Pope179}.  

For the dynamic Smagorinsky model, $C_s$ is computed by taking a least squares approach to determining the length scale \cite{Lilly180},
%
\begin{equation} \label{eqn:cs_eqn}
(C_s \Delta)^2 = \frac{ \left< \mathcal{L}_{ij} M_{ij} \right>}{ \left< M_{ij}M_{ij} \right> } \; ,
\end{equation}
%
where
%
%
\begin{equation}
\mathcal{L}_{ij} = 2( C_s \Delta)^2 \widehat{ |\overline{S} | \overline{S} }_{ij} - 
   2( C_s \widehat{\Delta})^2 \widehat{ |\overline{S}} | \widehat{\overline{S}}_{ij} \; ,
\end{equation}
%   
and
\begin{equation}
M_{ij} \equiv 2 \left( \widehat{ | \overline{S} | \overline{S} }_{ij} - \alpha^2 |\widehat{\overline{S}}|\widehat{\overline{S}}_{ij} \right) \; .
\end{equation} 
%
The hat defines an explicit test filter and the angled brackets in Equation \ref{eqn:cs_eqn} conceptually represent an averaging over a homogeneous region of space that, experience has shown, is necessary for stability.  Experience has also shown that averaging over the test filter width is adequate and the filter width ratio, $\alpha = \widehat{\Delta}/\Delta$, is usually taken to be 2.

\subsection{Subgrid Momentum Dissipation}
%%
Addressing the momentum closure involves finding a suitable model for the subgrid scale stress tensor, $\tau^{sgs}_{ij}$.  For the intended use as shown in Figure \ref{fig:v_and_v_hierarchy}, two common LES turbulence closure models are examined: the constant coefficient Smagorinsky model and the dynamic coefficient Smagorinsky model.  
%%
In LES modeling, field variables are decomposed into a spatially filtered field and a residual component, $u = \overline{u} + u'$.  This decomposition is known as a Leonard decomposition.  While seemingly similar to a Reynolds decomposition used in Reynolds Averaged Navier-Stokes (RANS) models, the Leonard decomposition has the property that the filtered residual component is generally not equal to zero, $\overline{u'} \neq 0$.  As a result, the subgrid stress term contains several terms, 
%%%
\begin{eqnarray}
\tau_{ij}^{sgs} = \overline{ \left(\overline{u}_i + u_i' \right) \left(\overline{u}_i + u_i' \right)} - \overline{u}_i \overline{u}_j \; , \; \; \; \; \; \; \; \; \; \; \; \; \; \;  \; \; \nonumber \\
= \underbrace{\overline{\overline{u}_i \overline{u}_j} - \overline{u}_i \overline{u}_j}_{L_{ij}}  + 
   \underbrace{\overline{\overline{u}_i {u}_j'} + \overline{{u}_i' \overline{u}_j}}_{C_{ij}} + 
   \underbrace{ \overline{u_i' + u_j'} }_{R_{ij}} \; , 
\end{eqnarray}
%
referred to as the Leonard stress, the cross stresses, and the Reynolds stress respectively.  

It is useful to consider the physical interpretation of the various components of the stress. The Leonard term is responsible for filtering and projecting the nonlinear interactions of the resolved components back to the finite LES space.  This is a correction to the resolved advective term in accordance with the stated explicit filter used to derive the LES equations. It does not account for aliasing errors. The first cross term represents advection of the resolved field by turbulent fluctuations. The second cross term represents the advection of subgrid scales by the resolved field. The Reynolds stress is familiar from RANS and represents the advection of subgrid scales by turbulent fluctuations. 

As with the scalar subgrid modeling terms, the eddy viscosity model is again invoked for $\tau^{sgs}_{ij}$.  The most common eddy viscosity model in LES is the Smagorinsky model \cite{Smagorinsky178}.  Defining the deviatoric subgrid stress tensor as, 
%%
\begin{equation}
\tau^{d, sgs}_{ij} = \tau^{sgs}_{ij} - \frac{1}{3}\tau^{sgs}_{kk} \delta_{ij}, 
\end{equation}
%%
the subgrid stress is approximated by,
%%
\begin{equation}
\tau_{ij}^{d, sgs} \approx -2 \nu_t \overline{S}_{ij} = -2 (C_s \Delta)^2 |\overline{S}|\overline{S}_{ij} \; , 
\end{equation}
%%
where, $\Delta$ is the filter width, $\nu_t$ is the eddy viscosity, $|\overline{S}| \equiv (2\overline{S}_{ij}\overline{S}_{ij})^{1/2}$, and typically $C_s \approx 2$  depending on the filter type, numerical method, and flow configuration \cite{Pope179}. This model is basically identical to Prandtl's mixing length model with $l = C_s \Delta$.

The dynamic procedure \cite{Germano74, Moin158} eliminates the need to specify the model constant, $C_s$, a priori, with the basic assumption that the constant is the same for two different filter scales. The smaller scale is historically referred to as the ``grid scale" (though the filter width need not equal the grid spacing, $\Delta \geq h$)), and the larger scale is referred to as the ``test scale".  Implicit in this assumption is the requirement that both scales lie within the inertial subrange. 

Defining the deviatoric residual stress tensor as, 
\begin{equation}
T_{ij}^d = T_{ij} - \frac{1}{3} T_{kk}\delta_{ij} \;,
\end{equation}
the residual stress at the test scale is given by,
%%
\begin{equation}\label{eqn:res_stress}
T_{ij}^{d} \equiv \overline{\widehat {u_i u_j}} - \overline{\widehat{u}}_i \overline{\widehat{u}}_j \approx -2(C_s \widehat{\Delta})^2 |\widehat{\overline{S}}| \widehat{\overline{S}}_{ij} \; .
\end{equation}
%%
where $\widehat{\Delta}$ is the test filter width and the hat defines an explicit test filter. By test filtering Equation \ref{eqn:tau_sgs} and combining this with \ref{eqn:res_stress}, one can construct the Leonard term, $\mathcal{L}_{ij}$. This is also known as the ``Germano identity",
%%
\begin{equation}
\mathcal{L}_{ij} = T_{ij} - \widehat{ \tau^{sgs}}_{ij} =  \overline{\widehat {u_i u_j}} - \overline{\widehat{u}}_i \overline{\widehat{u}}_j \; .
\end{equation}
%%
Notice that the Leonard term is directly computable from resolved LES quantities. By restating the Smagorinsky model in terms of the Germano identity, one ends up with an over-determined system of equations for the unknown, $C_s$,
%%
\begin{equation}
\mathcal{L}_{ij} = 2( C_s \Delta)^2 \widehat{ |\overline{S} | \overline{S} }_{ij} - 
   2( C_s \widehat{\Delta})^2 \widehat{ |\overline{S}} | \widehat{\overline{S}}_{ij} \; .
\end{equation}
%%   
Although we have pulled $C_s$ out of the test filtering operation of the subgrid stress, this approximation yields acceptable results. In practice, one takes a least squares approach to determining the length scale \cite{Lilly180},
%%
\begin{equation} \label{eqn:cs_eqn}
(C_s \Delta)^2 = \frac{ \left< \mathcal{L}_{ij} M_{ij} \right>}{ \left< M_{ij}M_{ij} \right> } \; ,
\end{equation}
%%
where
%%
\begin{equation}
M_{ij} \equiv 2 \left( \widehat{ | \overline{S} | \overline{S} }_{ij} - \alpha^2 |\widehat{\overline{S}}|\widehat{\overline{S}}_{ij} \right) \; .
\end{equation} 
%%
The only model parameter, then, is the filter width ratio, $\alpha = \widehat{\Delta}/\Delta$, usually taken to be 2.

The angled brackets in Equation \ref{eqn:cs_eqn} conceptually represent averaging over a homogeneous region of space which, experience has shown, is necessary for stability. We have found that averaging over the test filter width is adequate.
%%
With these implementation practices, the dynamic model is generally robust. The implementation can be made more efficient by computing the constant roughly every 10 time steps (based on the advective CFL), and only for the first Runge-Kutta step.

\subsection{LES Algorithm}
%
The set of filtered equations (Equations \ref{eqn:filtered_mass_balance}-\ref{eqn:filtered_heat_balance}) are discretized in space and time and solved on a staggered, finite volume mesh.  The staggering scheme consists of four offset grids. One grid stores the scalar quantities and the remaining three grids store each component of the velocity vector. The velocity components are situated so that the center of their control volume is located on the face centers of the scalar grid in their respective direction.  Figure \ref{fig:staggered_grid} shows an example of a two-dimensional grid and the staggering arrangement.
%
%% Staggered Grid
\begin{figure}
 \begin{center}
    \scalebox{.85}{\includegraphics{staggered_grid.png}}
   \caption{Staggered grid arrangement in two dimensions with u and v velocity cell centers located on the face centers of the scalar cells.}\label{fig:staggered_grid}
   \end{center}
\end{figure}
%

The staggering arrangement is advantageous for computing low-Mach LES reacting flows.  First, since a pressure projection algorithm is used, the velocities are exactly projected without interpolation error because the location of the pressure gradient coincides directly with the location of the velocity storage location. Second, Morinishi et al. \cite{morinishi98} showed that kinetic energy is exactly conserved when using a central differencing scheme on the convection and diffusion terms without a subgrid model and in combination with a staggered grid.  Having a spatial scheme that conserves kinetic energy is advantageous because it limits artificial dissipation that arises from the differencing scheme.  These conservation properties make the staggered grid a prime choice for LES reacting flow simulation.

For the spatial discretization of the LES scalar equations, flux limiting and upwind schemes for the convection operator are used.  These schemes are advantageous for ensuring that scalar values remain bounded.  For the momentum equation, a central differencing scheme for the convection operator is used.  All diffusion terms are computed with a second order approximation of the gradient.  

When computing the 2d surface filtered field on the faces of the control volume, one is forced to use an interpolation from the 3d volume filtered field.  This approximation is tolerated because computing the 2d surface field is simply not possible with the given grid scheme. % This is a good example of how LES modeling issues and discrete numerics issue intertwine in applied LES modeling.     

%Jennifer edited
An explicit time stepping scheme is chosen.  A general, multistep explicit update for a variable, $\phi$, may be written as, 
%
%===== Explicit Update =====
\begin{eqnarray}\label{eqn:forward_euler}
\phi^0 = \phi^n \; , \nonumber \\
  \phi^{(i)} = V \sum_{k=0}^{m-1}
\left( \alpha_{i,k}  \phi^{(k)} +
\Delta t \beta_{i,k} L( \phi^{(k)}) \right) \; , \; \; \; \; i = 1, ..., m \\
\phi^{(m)} = \phi^{n+1} \; , \nonumber
\end{eqnarray}
%
where $n$ is the time level, $m$ is the substep between $n$ and $n+1$, $\alpha$ and $\beta$ are integration coefficients, and $L$ is a linearization operator on the the convective flux and source terms.
%
Letting $m=1$ and $\alpha = \beta = 1$ the forward-Euler time
integration scheme is determined,
%%
%%===== Forward Euler =====
\begin{equation}\label{eqn:forward_euler}
\left(  \phi \right)^{n+1} = \left(
\phi \right)^{n} + \Delta t (L( \phi)^n) \; .
\end{equation}
%%
A higher order, multistep method is derived by letting $m > 1$ and
choosing appropriate constants for $\alpha$ and $\beta$. For this
study, two step and three step, strong stability preserving (SSP)
coefficients were chosen from Gottlieb et al.
\cite{Gottlieb75}.

Using the coefficients given by Gottlieb et al., the SSP-RK 2 stepping scheme is
%%
%%===== second order time stepping =====
\begin{eqnarray}\label{eqn:rk_second_order}
( \phi)^{(1)} = ( \phi)^{n} + \Delta
t (L(\phi)^n) \\ \nonumber
( \phi)^{n+1} = \frac{1}{2}(
\phi)^{n} + \frac{1}{2}( \phi)^{1} +
\frac{1}{2}\Delta t (L( \phi)^{(1)}) \;.
\end{eqnarray}
%%
SSP-RK 3 time stepping scheme is,
%%
%%===== third order time stepping =====
\begin{eqnarray}\label{eqn:rk_second_order}
(\phi)^{(1)} = (\phi)^{n} + \Delta t (L(\phi)^n) \\ \nonumber
(\phi)^{(2)} = \frac{3}{4}(\phi)^{n} + \frac{1}{4}( \phi)^{(1)} +
\frac{1}{4}\Delta t (L( \phi)^{(1)}) \\ \nonumber ( \phi)^{(n+1)} =
\frac{1}{3}(\phi)^{n} + \frac{2}{3}(
\phi)^{(2)} + \frac{1}{4}\Delta t (L(
\phi)^{(2)}) \; .
\end{eqnarray}
%%

The time step is limited by
%
\begin{equation}
\Delta t \leq c \Delta t_{F.E.}
\end{equation}
%
where $\Delta t_{F.E.}$ is the forward-Euler time step limited by
the Courant-Friedrichs-Levy condition and $c$ is a constant less
than or equal to one.  

A higher order, multistep method is derived by letting $m > 1$ and choosing appropriate constants for $\alpha$ and $\beta$. For this study, two step and three step, strong stability preserving (SSP) coefficients were chosen from Gottlieb et al. \cite{Gottlieb75}.  The coefficients for SSP-RK 2 and SSP-RK 3 are optimal in the sense that the scheme is stable when $c=1$ if the forward-Euler time step is stable for hyperbolic problems.  In practice, for the Navier-Stokes equations, the value of $c$ is taken less than one.

Choosing an explicit time stepping scheme, rather than an implict one, creates a challenge for solving the set of equations.  The density at the $n+1$ timestep, which is required to determine the cardinal variables,  requires an estimation.  Taking the estimated density for  $\overline{\rho}^{n+1}$ to be $\overline{\rho}^*$, the estimation can be as simple as $\overline{\rho}^* = \overline{\rho}^n$.  Note that the 2d and 3d filter distinction is dropped for the remainder of this discussion for the sake of simplicity.  A slightly more complicated procedure involves a forward-Euler step of the continuity equation to obtain $\overline{\rho}^*$.  This is written as,  
%
\begin{equation}\label{eqn:rho_update}
\overline{\rho}^{ *} = \overline{\rho}^{ n} - \Delta t \frac{S_k}{V} n_{kj} \left(  \overline{\rho} \widetilde{u_j} \right) \; .
\end{equation}
%
Ideally, one would like to know $\overline{\rho}^{n+1}$ rather than an estimate.  While more details will be discussed in Section \ref{sec:combust_react_models}, one recognizes that $\rho$ is a function of the same variables that are being updated in time, namely, the mixture fraction, $f$, and enthalpy, $h$.  This quandary is a result of the explicit time stepping method will not be resolved for variable density flows without using a fully implicit method.  Explicit methods, however, do have advantages, especially for large scale parallel computations.  Specifically, explicit methods are easier to load balance because the amount of work required for each processor is readily determined a priori, which makes for an efficient parallel computation.  Explicit methods are also easier to code into a computer and to debug.  For these reasons, the current algorithm discussion is limited to explicit methods only. 

The explicit algorithm for solving the set of filtered equations is shown in Algorithm \ref{alg:LES_algorithm}.  
%
\begin{algorithm}[t]
\caption{Explicit LES algorithm.}\label{alg:LES_algorithm}
%
\begin{algorithmic}[] %add [1] for #'s
\FOR{$t = t_{min}...t_{max}$}
\FOR{$RK_{step} = 1$...$N$} 
\STATE 	Solve for scalars products $(\overline{\rho} \widetilde{f})^{n+1}$ and $(\overline{\rho} \widetilde{h})^{n+1}$.
\STATE	Estimate $\overline{\rho}^* = \overline{\rho}^{n+1}$ from Equation \ref{eqn:rho_update}
\IF{$\overline{\rho}^* < \overline{\rho}_{min}$ or $ \overline{\rho}^* > \overline{\rho}_{max}$}
\STATE $\overline{\rho}^* = \overline{\rho}^{n}$
\ENDIF
\STATE Compute $\widetilde{f}^{n+1} = (\overline{\rho}\widetilde{f})^{n+1}/{\overline{\rho}^*}$ and $\widetilde{h}^{n+1} = (\overline{\rho}\widetilde{h})^{n+1}/{\overline{\rho}^*}$ 
\STATE Compute $\overline{\rho}^{n+1} = f(\widetilde{f}^{n+1}, \widetilde{h}^{n+1})$
\STATE Compute $\widetilde{\mathbf{u}}^*$, the unprojected velocities
\STATE Perform RK averaging if needed
\STATE Compute correct pressure from pressure poisson equation
\STATE Project velocities with correct pressure to get $\widetilde{\mathbf{u}}^{n+1}$ 
\ENDFOR
\ENDFOR
\end{algorithmic}
\end{algorithm}

%FIXME
%\subsection{Uintah Specification}


%\subsubsection{Basic Inputs}


%\subsubsection{Turbulence}


%\subsubsection{Properties}


%\subsubsection{BoundaryConditions}


%\subsubsection{Physical Constants}


%\subsubsection{Solvers}


%\subsection{Examples}


%\subsection{References}

\section{References}
\bibliographystyle{plain}
\bibliography{arches}
