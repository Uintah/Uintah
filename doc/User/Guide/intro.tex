% -*-latex-*-
%
%  The contents of this file are subject to the University of Utah Public
%  License (the "License"); you may not use this file except in compliance
%  with the License.
%
%  Software distributed under the License is distributed on an "AS IS"
%  basis, WITHOUT WARRANTY OF ANY KIND, either express or implied. See the
%  License for the specific language governing rights and limitations under
%  the License.
%
%  The Original Source Code is SCIRun, released March 12, 2001.
%
%  The Original Source Code was developed by the University of Utah.
%  Portions created by UNIVERSITY are Copyright (C) 2001, 1994
%  University of Utah. All Rights Reserved.
%

% intro.tex
%

\section{Introduction}
\label{sec:intro}


This is the \etitle{\srug}.  It describes the purpose and use, of the
\sr{} problem solving environment (\pse).  It is for those users who
will be building and executing \dfn{networks} within the \sr{}
environment.

Those who will be installing \sr{} should read the
\htmladdnormallinkfoot{\srig}{\htmlurl{\latexhtml{\scisoftware/doc}{../../..}/Installation/Guide/ch.inst.html}}.


%\subsection{Conventions}
%\label{sec:conventions}

%\missing{Discussion of typographic conventions}

\subsection{Road Map}
\label{sec:roadmap}

This document is organized into the following main sections:

\begin{description}
  \descitem{\secref{Introduction}{sec:intro}} This introduction.
  
  \descitem{\secref{Concepts}{sec:concepts}} Introduces the concept of
  an integrated problem solving environment and describes how \SR{}
  embodies these ideas.
  
  \descitem{\secref{Starting \sr}{sec:startingup}} Describes

  \descitem{\secref{Working with Networks}{sec:workwithnets}}
  Discusses tasks involved in building, editing, and executing
  networks.

  \descitem{\secref{Visualization with the \viewer{}}{sec:viewer}}
  Describes the purpose and use of the \viewer visualization module.
  The \viewer{} is \sr{}'s most commonly used module.

  \descitem{\secref{Packages}{sec:packages}} Gives an overview of the 
  of the \sr{} and \biopse{} packages.
  
  \descitem{\secref{Importing data into \sr{}}{sec:import}} Describes
  ways to import/export ``foreign'' data into/out of \SR{}.
\end{description}

\subsection{Getting Help}
\label{sec:help}

Help is available from the following sources.

\subsubsection{In the Documentation Distribution}

\sr{} documentation is distributed separately from its source code.
\sr{}'s documentation distribution can be download from
\htmladdnormallinkfoot{\sr{}'s software download
  page}{\scisoftwarearchiveurl}.  See the \sr{}
\htmladdnormallinkfoot{\srig}{\htmlurl{\latexhtml{\scisoftware/doc}{../../..}/Installation/Guide/ch.inst.html}} for
instructions on downloading and installing the documentation
distribution.

After installing the documentation point your browser at the
\filename{index.html} file which is located in the distribution's top
level \directory{doc} directory (\ie{} \ab{top of documentation
  distribution}/doc/index.html).

\subsubsection{On the Web}

This and other documents related to \sr{} can be found 
\htmladdnormallinkfoot{online}{\scidocurl{}}.

Be sure to visit the \sci{} web site for lots of other
information related to \sr{} and the \scii{}.

\subsubsection{From the Mailing Lists}

The \sr{} \emph{users} mail list is a forum for discussing \sr{}
related issues.  To subscribe send mail to:

\mailto{Majordomo@sci.utah.edu}

with the following command in the body of your message:

\keyboard{subscribe scirun-users}

After subscribing you may send questions to
\mailto{scirun-users@sci.utah.edu}.

The \sr{} \emph{developers} list is a forum for network and module
developers.  To subscribe send mail to:

\mailto{Majordomo@sci.utah.edu}

with the following command in the body of your message:

\keyboard{subscribe scirun-develop}

After subscribing you may send questions to
\mailto{scirun-develop@sci.utah.edu}.

\subsection{Reporting Bugs}
\label{sec:bugs}

Please report bugs!  To report a bug visit \sr{}'s
\htmladdnormallinkfoot{bug database}{\bugsurl} web page.

Reporting bugs this way (rather than by way of the mailing list) ensures
that they will be fixed in timely manner.

%%% Local Variables: 
%%% mode: latex
%%% TeX-master: "usersguide"
%%% End: 
