%
%  For more information, please see: http://software.sci.utah.edu
% 
%  The MIT License
% 
%  Copyright (c) 2004 Scientific Computing and Imaging Institute,
%  University of Utah.
% 
%  License for the specific language governing rights and limitations under
%  Permission is hereby granted, free of charge, to any person obtaining a
%  copy of this software and associated documentation files (the "Software"),
%  to deal in the Software without restriction, including without limitation
%  the rights to use, copy, modify, merge, publish, distribute, sublicense,
%  and/or sell copies of the Software, and to permit persons to whom the
%  Software is furnished to do so, subject to the following conditions:
% 
%  The above copyright notice and this permission notice shall be included
%  in all copies or substantial portions of the Software.
% 
%  THE SOFTWARE IS PROVIDED "AS IS", WITHOUT WARRANTY OF ANY KIND, EXPRESS
%  OR IMPLIED, INCLUDING BUT NOT LIMITED TO THE WARRANTIES OF MERCHANTABILITY,
%  FITNESS FOR A PARTICULAR PURPOSE AND NONINFRINGEMENT. IN NO EVENT SHALL
%  THE AUTHORS OR COPYRIGHT HOLDERS BE LIABLE FOR ANY CLAIM, DAMAGES OR OTHER
%  LIABILITY, WHETHER IN AN ACTION OF CONTRACT, TORT OR OTHERWISE, ARISING
%  FROM, OUT OF OR IN CONNECTION WITH THE SOFTWARE OR THE USE OR OTHER
%  DEALINGS IN THE SOFTWARE.
%


\chapter{Importing and Exporting \sr{} Data}
\label{ch:import_export} 
\index{importing}
\index{exporting}

This chapter describes the format of text-based data files and the use
of \sr{}'s reader and writer modules to import and export \sr{} data
objects to text files. 

\sr{}'s reader and writer modules support two classes of data formats:
\sr{}'s binary (also known as persistent) data files and text-based
data files.  Text-based data are used primarily to import data
generated by other software into \sr{} and to export \sr{} generated
data to other software.  Note that binary formats load faster, support
better numerical precision, and are usually smaller than their
text-based counterparts.  \sr{}'s binary data files are portable
between machine architectures.

\sr{}'s fields, matrices, and color maps can be stored in either type
of file although fields with regular meshes can only be stored in
\sr{}'s binary format.  Fields with regular meshes can be
imported/exported using the \package{Teem} package.  See
\htmladdnormallinkfoot{\package{Teem} file format
  documentation}{http://www.cs.utah.edu/~gk/teem/nrrd/format.html} and
\htmladdnormallinkfoot{\package{Teem} module
  descriptions}{\htmlurl{\latexhtml{\scisoftware{}/doc}{../../..}/Developer/Modules/Teem.html}}
for information.

Note that 


\section{Text-based File Formats}

The following text-based files are supported:

\begin{itemize}
\item \dfn{node coordinate data}
\item \dfn{node connectivity data}
\item \dfn{structured mesh data}
\item \dfn{column matrix data}
\item \dfn{dense matrix data}
\item \dfn{sparse row matrix data}
\item \dfn{color map data}
\end{itemize}

High-level data types are stored as one or more text files.  For
example, an unstructured field is spread across three files: a node
coordinate data file, a node connectivity data file, and a column matrix data
file.  The following table shows the files needed by each importable \sr{} data
type.

\begin{tabular}{|l|l|l|l|l|l|l|}
&Coordinate&Connectivity&Structure Mesh&Column Matrix&Dense
Matrix&Sparse Row Matrix&Color Map\\ \hline
Curve Field&X&X&&X&&&\\ \hline
Hex Vol Field&X&X&&X&&&\\ \hline
Point Cloud Field&X&&&X&&&\\ \hline
Quad Surface Field&X&X&&X&&&\\ \hline
Tet Vol Field&X&X&&X&&&\\ \hline
Tri Surface Field&X&X&&X&&&\\ \hline
Structured Curved Field&X&&&&&&\\ \hline
Structured Quad Surface Field&X&&&&&&\\ \hline
Structured Hex Vol Field&X&&&&&&\\ \hline
Column Matrix&&&&X&&&\\ \hline
Dense Matrix&&&&&X&&\\ \hline
Sparse Row Matrix&&&&&&X&\\ \hline
Color Map&&&&&&&X\\ \hline
\end{tabular}



\subsection{Node Coordinate File}
\label{sec:node_loc_fmt}

Node coordinate files are called \dfn{pts} files (the file extension is
\filename{.pts}).  A \filename{pts} file defines the coordinate of
each node in a mesh.

The format is as follows:

\begin{verbatim}
N
x0 y0 z0
x1 y1 z1
    .
    .
    .
xN yN zN
\end{verbatim}

\verb|N| is the number of nodes in the file.  \verb|N| is optional.

On export, option \option{-noPtsHeader} tells an export converter to
omit \verb|N| from the output file.

If \verb|N| is omitted from a file on import, option
\option{-noPtsHeader} must be passed to the import converter.


\subsection{Connectivity Files}
\label{sec:node_conn_fmt}

Five types of node connectivity files exist corresponding to the
unstructured mesh types: \datatype{CurveMesh}, \datatype{TriSurfMesh},
\datatype{QuadSurfMesh}, \datatype{TetVolMesh}, and
\datatype{HexVolMesh}.  The five file formats define
edge, triangular, quadrilateral, tetrahedral, and hexahedral,
elements respectively.  Connectivity files are text files.

Files defining edge elements have the extension \filename{.edge} and
are called \dfn{edge} files.  \dfn{Tri} files define triangular
elements and have the extension \filename{.tri}.  Likewise for
\dfn{Quad}, \dfn{tet}, and \dfn{hex} files.

The unstructured field converters import/export \filename{edge},
\filename{tri}, \filename{quad}, \filename{tet}, and hex connectivity
files.

Connectivity files are formatted as follows:

\begin{verbatim}
N
Node indicies of element 0
Node indicies of element 1
            .
            .
            .
Node indicies of element N
\end{verbatim}

\verb|N| specifies the number of elements in a file.  \verb|N| is
optional.  Converter command line option \option{-noElementsCount} is
used to omit \verb|N| from a connectivity file on export.  The same
command line option is used to tell a converter that \verb|N| is
missing from a text file on import.

Each of the remaining lines specify node indices of one element.
Node indices are assumed to be zero-based unless option
\option{-oneBasedIndexing} is used.  This option can be used with
import and export converters to read ones-based indices on
import or to write ones-based indices on export.

Edge files have two node indices per line, tri files have three, quad
and tet files have four, and hex files have eight node indices per
line.  For example, a tri file looks like the following:

\begin{verbatim}
N
i0 j0 k0
i1 j1 k0
    .
    .
    .
iN jN kN
\end{verbatim}

or like this (with \verb|N|) omitted:

\begin{verbatim}
i0 j0 k0
i1 j1 k0
    .
    .
    .
iN jN kN
\end{verbatim}

%% Needs work!
\subsection{Structured Meshes}

Structured meshes are imported/exported by the converters as node
coordinate (\filename{pts}) files.  Node coordinates are specified explicitly in a
\filename{pts} file. Connectivities are implicit because node coordinates are
stored in \dfn{scan-line} order.

For example, in a structured hexahedral mesh, the list of nodes comprising
element \(e_{i,j,k}\) is \(\{n_{i,j,k}, n_{i,j,k+1}, n_{i,j+1,k+1}, n_{i,j+1,k}, n_{i+1,j,k}, n_{i+1,j,k+1}, n_{i+1,j+1,k+1}, n_{i+1,j+1,k}\}\)

% Verify the following paragraph.  See converter source code.
\filename{Pts} files for structured meshes differ slightly
from \filename{pts} files described in \secref{Node Coordinate File
  Format}{sec:node_loc_fmt}.  \filename{Pts} files for
structured meshes specify the number of indices in each dimension of the
mesh. 

A structured curve field (StructCurveField) \filename{pts} file looks
like this:

\begin{verbatim}
NI
x0 y0 z0
x1 y1 z1
    .
    .
    .
xN yN zN
\end{verbatim}

A structured quadrilateral surface field
(StructQuadSurfField) \filename{pts} file looks like this:

\begin{verbatim}
NI NJ
x0 y0 z0
x1 y1 z1
    .
    .
    .
xN yN zN
\end{verbatim}

A structured hexahedral volume field (StructHexVolField)
\filename{pts} file looks like this:

\begin{verbatim}
NI NJ NK
x0 y0 z0
x1 y1 z1
    .
    .
    .
xN yN zN
\end{verbatim}

On export, option \option{-noHeader} tells a converter
(\command{StructCurveFieldToText}, \command{StructQuadSurfFieldToText}, or
\command{StructHexVolFieldToText}) to omit mesh index counts
from the output file.

If node index count values are missing from a file on import, option
\option{-noHeader} must be used.  Option \option{-noHeader} tells the
converter (\command{TextToStructCurveField},
\command{TextToStructQuadSurfField}, or
\command{TextToStructHexVolField}) the mesh index count(s).  Option
\option{-noHeader} is followed by the appropriate mesh index count
values.

\subsection{Column Matrix}
\label{sec:colmat}

The column matrix file format is:

\begin{verbatim}
N
v0 
v1
.
.
.
vN
\end{verbatim}

The converters \command{ColumnMatrixToText} and \command{TextToColumnMatrix} 
export/import column matrices.

\verb|N| is optional.  It specifies the number of matrix rows.  

On export, option \option{-noHeader} tells
\command{ColumnMatrixToText} to omit \verb|N| from the output file.

If \verb|N| is omitted from a file on import, option
\option{-noHeader} must be passed to \command{TextToColumnMatrix}.
Option \option{-noHeader} tells the converter the number of data
values in the matrix.

\verb|N| is followed by a list of data values.

\subsection{Dense Matrix}
\label{sec:dense_matrix}

The dense matrix file format is:

\begin{verbatim}
N M
v(0,0) v(0,1)...v(0,M)
v(1,0) v(1,1)...v(1,M)
        .
        .
        .
v(N,0) v(N,1)...v(N,M)
\end{verbatim}

The programs \command{DenseMatrixToText} and \command{TextToDenseMatrix} 
convert dense matrices to and from text based forms.

\verb|N| and \verb|M| are optional.  \verb|N|, \verb|M| are the number
of matrix rows and columns respectively .  

On export, option \option{-noHeader} tells
\command{DenseRowMatrixToText} to omit \verb|N| and \verb|M| from the
output file..

If \verb|N| and \verb|M| are omitted from a file on import, option
\option{-noHeader N M} must be passed to
\command{TextToDenseMatrix}.

Following \verb|N| and \verb|M| is a list of data values given in row
major order.


\subsection{Sparse Row Matrix}
\label{sparse_row_matrix}

The sparse row matrix file format is:

\begin{verbatim}
NR NC NE
r0 c0 v0
r1 c1 v1
   .
   .
   .
rNE cNE vNE
\end{verbatim}

The programs \command{SparseRowMatrixToText} and
\command{TextToSparseRowMatrix} export/import sparse row matrices.

\verb|NR|, \verb|NC|, and \verb|NE| are optional.  If present, they
specify the number of rows, number of columns, and number of matrix
entries respectively.

On export, option \option{-noHeader} tells
\command{SparseRowMatrixToText} to omit \verb|NR|, \verb|NC|, and
\verb|NE| from the output file.

If \verb|NR|, \verb|NC|, and \verb|NE| are omitted from a file on
import, option \option{-noHeader \ptext{nrows} \ptext{ncols}
  \ptext{nentries}} must be passed to \command{TextToSparseRowMatrix}.

Matrix entries must have ascending row indices. Column indices must be
in ascending order for rows with multiple entries.

\subsection{Color Map}
\label{sec:colormap_fmt}

The color map file format is:

\begin{verbatim}
N
r1 g1 b1 a1 v1
r2 g2 b2 a2 v2
      .
      .
      .
rN gN bN aN vN
\end{verbatim}

The programs \command{ColorMapToText} and \command{TextToColorMap}
export/import color maps.

\verb|N| is optional.  It specifies the number of color map entries in
the file.

On export option \option{-noHeader} tells \command{ColorMapToText} to
omit \verb|N| from the output file.

If \verb|N| is omitted from a file on import, option
\option{-noHeader} must be passed to \command{TextToColorMap}.

Each line following \verb|N| is a color map entry consisting of an RGB
color entry, an alpha value, and a data value.  All RGB, alpha, and
data values are in the range 0.0 to 1.0 inclusive.


\section{Using Readers}
\label{sec:using_readers}


\section{Using Writers}
\label{sec:using_writers}

