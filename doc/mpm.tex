
\section{MPM}

\subsection{Introduction}

The Material Point Method (MPM) as described by Sulsky, et al.
\cite{sulskycmame,sulskycpc} is a particle method for structural
mechanics simulations.  Solid objects are represented by a
collection of particles, or ``material points."  Each
of these particles carries with it information for that part of the
solid object that it represents.  This includes the position, mass, volume,
velocity, stress and state of deformation of that material.  MPM differs from
other so called ``mesh-free" particle methods in that, while each object
is primarily represented by a collection of particles, a computational mesh
is also an important part of the calculation.  Particles do not interact
with each other directly, rather the particle information is accumulated
to the grid, where the equations of motion are integrated forward in time.
This time advanced solution is then used to update the particle state.

The method usually uses a regular structured grid as a computational mesh.
While this grid, in principle, deforms as the material that it is representing
deforms, at the end of each timestep, it is reset to its original undeformed
position, in effect providing a new computational grid for each timestep.
The use of a regular structured grid for each time step has a number of
computational advantages.  Computation of spatial gradients is simplified.
Mesh entanglement, which can plague fully Lagrangian techniques, such as
the Finite Element Method (FEM), is avoided.  MPM has also been successful
in solving problems involving contact between colliding objects, having an
advantage over FEM in that the use of the regular grid eliminates the
need for doing costly searches for contact surfaces\cite{bard}.

In addition to the advantages that MPM brings, as with any numerical technique, it has its own set of shortcomings.  It is computationally more
expensive than a comparable FEM code.  Accuracy for MPM is typically lower
than FEM, and errors associated with particles moving around the computational
grid can introduce non-physical oscillations into the solution.  Finally,
numerical difficulties can still arise in simulations involving large
deformation that will prematurely terminate the simulation.  The severity of
all of these issues (except for the expense) has been significantly reduced
with the introduction of the Generalized Interpolation Material Point Method,
or GIMP\cite{bardgimp}.  The basic concepts associated with GIMP will be
described in the subsequent section.  Throughout this document, MPM (which
ends up being a special case of GIMP) will frequently be referred to
interchangably with GIMP.

In addition, MPM can be incorporated with a multi-material CFD algorithm
as the structural component in a fluid-structure interaction formulation.
This capability was first demonstrated in the CFDLIB codes from
Los Alamos by Bryan Kashiwa and co-workers\cite{kashiwa2000}.  There, as
in the Uintah-MPMICE component,
MPM serves as the Lagrangian description of the solid
material in a multimaterial CFD code.  Certain elements of the
solution procedure are based in the Eulerian CFD algorithm, including
intermaterial heat and momentum transfer as well as satisfaction
of a multimaterial equation of state.  The use of a Lagrangian method
such as MPM to advance the solution of the solid material eliminates
the diffusion typically associated with Eulerian methods.  The Uintah-MPM
component will be described in later chapter of this manual.

Subsequent sections of this chapter will first give a relatively brief
description of the MPM and GIMP algorithms.  This will, of course, be
focused mainly on describing the capabilities of the Uintah-MPM component.
Following that is a description of the information that goes into an input
file.  Finally, a number of examples are provided, along with representative
results.

\subsection{Theory - Algorithm Description}

\subsection{Uintah Specification}

\subsubsection{Basic Inputs}
\subsubsection{Physical Constants}
\subsubsection{Material Properties}
\subsubsection{Constitutive Models}
\subsubsection{Contact}
\subsubsection{BoundaryConditions}
\subsubsection{Physical Boundary Conditions}

\subsection{Examples}

\subsection{References}
\bibliography{mpm}
