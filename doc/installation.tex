\section{Downloading the Software} \label{Sec:download}

The Uintah problem solving environment provides a framework for solving PDEs on structured grids

Uintah can be obtained either from a tarball (http://www.uintah.utah.edu) or by using svn to download the latest source from the following:

\begin{verbatim}
svn co https://code.sci.utah.edu/svn/SCIRun/trunk Uintah
\end{verbatim}

The above command checks out the Uintah source tree and installs it
into a directory called Uintah in the users home directory.

The Thirdparty library can similarly be obtained via:
\begin{verbatim}
svn co https://code.sci.utah.edu/svn/Thirdparty/1.25.4 Thirdparty
\end{verbatim}

The Thirdparty library source code is downloaded into a directory
called Thirdparty.

\subsection{Installing Thirdparty, SCIRun, and Uintah}

\subsubsection{Thirdparty Install}


Please read the README.txt found in \emph{~/Thirdparty}.

Thirdparty should be installed in \emph{/usr/local/Thirdparty}.  As
root, create this directory:

\begin{verbatim}

     \# mkdir /usr/local/Thirdparty

\end{verbatim}

Chang to the Thirdparty directory you checked out, i.e. cd ~/Thirdparty

After reading the README.txt file type the follow as the root user:

\textbf{32bit OS:}

\begin{verbatim}

      \# ./install.sh /usr/local/Thirdparty/ 32

\end{verbatim}

\textbf{64bit OS:}


\begin{verbatim}

      \# ./install.sh /usr/local/Thirdparty/ 64

\end{verbatim}


\subsubsection{Configuring Uintah}

cd to \emph{~/SCIRun} and create the following directories: dbg and opt

cd to dbg and type the following to configure for a debug build:

\begin{verbatim}
./src/configure --enable-debug --enable-sci-malloc 
--enable-package=Uintah 
--with-thirdparty=/usr/local/Thirdparty/1.25.5/Linux/gcc-4.3.1-2-32bit/
\end{verbatim}

Then build the software by typing \texttt{make} at the command
line. Once the debug build has finished which can take roughly an hour
on a single processor Pentium IV computer, cd to the opt/ and type the
following to configure for an optimized build:

\begin{verbatim}
./src/configure '--enable-optimze=-march=pentium4 -msse -msse2 
-mfpmath=sse -03' --disable-sci-malloc --enable-assertion-level=0 
--enable-package=Uintah 
--with-thirdparty=/usr/local/Thirdparty/1.25.4/Linux/gcc-4.3.1-2-32bit/
\end{verbatim}

Then build the software by typing \texttt{make} at the command line


