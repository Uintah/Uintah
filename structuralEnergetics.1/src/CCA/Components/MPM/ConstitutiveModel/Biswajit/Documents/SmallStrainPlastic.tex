\documentclass[twoside,10pt,a4paper]{article}
%\usepackage[T1]{fontenc}
%\usepackage[urw-garamond]{mathdesign}
\usepackage{helvet}
\usepackage[pdftex]{graphicx}
\usepackage[usenames]{color}
\usepackage{cancel}
%\usepackage{mathtools}
%\newcommand\hmmax{-1} % default 3
%\newcommand\bmmax{4} % default 4
\usepackage{bm}
\usepackage{amsmath}
%\usepackage{amssymb}
\usepackage{amstext}
%\usepackage{amsfonts}
\usepackage{lastpage}
\usepackage[sf,bf,compact,topmarks,calcwidth,pagestyles]{titlesec}
\usepackage{titletoc}
\usepackage{fancyhdr}
\usepackage{datetime}
\usepackage[margin={1.5cm,1.5cm},font=normalsize,format=hang,labelfont=bf,textfont=it,labelsep=endash]{caption}
\usepackage{setspace}
\usepackage{boxedminipage}
%\usepackage{stmaryrd}
\usepackage{pxfonts}
\usepackage[normalem]{ulem}
\usepackage{algorithm}
\usepackage{algorithmic}

% The page layout plan.
\setlength{\parskip}       {8pt}
\setlength{\oddsidemargin} {-10mm}
\setlength{\evensidemargin}{-10mm}
\setlength{\topmargin}     {-15mm}
\setlength{\headheight}    {10mm}
\setlength{\headsep}       {5mm}
\setlength{\textheight}    {250mm}
\setlength{\textwidth}     {180mm}
\setlength{\topskip}       {12pt}
\setlength{\footskip}      {10mm}
\setlength{\marginparsep}  {0mm}
\setlength{\marginparwidth}{0mm}
\setlength{\marginparpush} {0mm}
\raggedright

\definecolor{red}{rgb}{1,0,0}
\definecolor{green}{rgb}{0,1,0}
\definecolor{blue}{rgb}{0,0,1}
\definecolor{Red}{rgb}{0.8666,0.03137,0.02352}
\definecolor{Blue}{rgb}{0.00784,0.67059,0.91764}
\definecolor{Darkgreen}{rgb}{0,0.68235,0}
\definecolor{Green}{rgb}{0,0.8,0}
\definecolor{Bl}{rgb}{0,0.2,0.91764}
\definecolor{Royalblue}{rgb}{0,0.2,0.91764}
\definecolor{Brickred}{rgb}{0.644541,0.164065,0.164065}
\definecolor{Brown}{rgb}{0.6,0.4,0.4}
\definecolor{Orange}{rgb}{1,0.647059,0}
\definecolor{Indigo}{rgb}{0.746105,0,0.996109}
\definecolor{Violet}{rgb}{0.308598,0.183597,0.308598}
\definecolor{Lightgrey}{rgb}{0.762951,0.762951,0.762951}
\definecolor{Darkgrey}{rgb}{0.503548,0.503548,0.503548}
\definecolor{Pink}{rgb}{1,0.6,0.6}
\definecolor{MyLightMagenta}{cmyk}{0.1,0.8,0,0.1}
\definecolor{MyDarkBlue}{rgb}{0,0.08,0.45}

\pagestyle{myheadings}

\newcommand{\Red}{\color{Brickred}}
\newcommand{\Blue}{\color{Royalblue}}
\newcommand{\Green}{\color{Darkgreen}}
\newcommand{\Violet}{\color{Violet}}
\newcommand{\Jump}[1]{\ensuremath{\llbracket#1\rrbracket}}
\newcommand{\Blimitx}[1]{\ensuremath{\left[#1\right]_{x_a}^{x_b}}}
\newcommand{\Deriv}[2]{\ensuremath{\cfrac{d#1}{d#2}}}
\newcommand{\MDeriv}[2]{\ensuremath{\cfrac{D#1}{D#2}}}
\newcommand{\DDeriv}[2]{\ensuremath{\cfrac{d^2#1}{d#2^2}}}
\newcommand{\DDDeriv}[2]{\ensuremath{\cfrac{d^3#1}{d#2^3}}}
\newcommand{\DDDDeriv}[2]{\ensuremath{\cfrac{d^4#1}{d#2^4}}}
\newcommand{\Intx}{\ensuremath{\int_{x_a}^{x_b}}}
\newcommand{\IntX}{\ensuremath{\int_{X_a}^{X_b}}}
\newcommand{\Intiso}{\ensuremath{\int_{-1}^{1}}}
\newcommand{\IntOmegaA}{\ensuremath{\int_{\Omega_0}}}
\newcommand{\IntOmega}{\ensuremath{\int_{\Omega}}}
\newcommand{\IntDOmega}{\ensuremath{\int_{\partial\Omega}}}
\newcommand{\IntOmegap}{\ensuremath{\int_{\Omega'}}}
\newcommand{\Norm}[2]{\ensuremath{\left\lVert#1\right\rVert_{#2}}}
\newcommand{\Var}[1]{\ensuremath{\delta #1}}
\newcommand{\DelT}{\ensuremath{\Delta t}}
\newcommand{\CalA}{\ensuremath{\mathcal{A}}}
\newcommand{\CalB}{\ensuremath{\mathcal{B}}}
\newcommand{\CalD}{\ensuremath{\mathcal{D}}}
\newcommand{\BCalD}{\ensuremath{\boldsymbol{\CalD}}}
\newcommand{\CalF}{\ensuremath{\mathcal{F}}}
\newcommand{\CalL}{\ensuremath{\mathcal{L}}}
\newcommand{\CalM}{\ensuremath{\mathcal{M}}}
\newcommand{\BCalM}{\ensuremath{\boldsymbol{\CalM}}}
\newcommand{\CalN}{\ensuremath{\mathcal{N}}}
\newcommand{\CalP}{\ensuremath{\mathcal{P}}}
\newcommand{\CalS}{\ensuremath{\mathcal{S}}}
\newcommand{\BCalS}{\ensuremath{\boldsymbol{\CalS}}}
\newcommand{\CalT}{\ensuremath{\mathcal{T}}}
\newcommand{\CalV}{\ensuremath{\mathcal{V}}}
\newcommand{\CalW}{\ensuremath{\mathcal{W}}}
\newcommand{\CalX}{\ensuremath{\mathcal{X}}}
\newcommand{\Comp}[2]{\ensuremath{#1 \circ #2}}
\newcommand{\Map}[3]{\ensuremath{#1 : #2 \rightarrow #3}}
\newcommand{\MapTo}[3]{\ensuremath{#1 : #2 \mapsto #3}}
\newcommand{\Real}[1]{\ensuremath{\mathbb{R}^{#1}}}
\newcommand{\Ve}{\ensuremath{\varepsilon}}
\newcommand{\BHat}[1]{\ensuremath{\widehat{\boldsymbol{#1}}}}
\newcommand{\BTx}{\ensuremath{\tilde{\boldsymbol{x}}}}
\newcommand{\Beh}{\ensuremath{\hat{\boldsymbol{e}}}}
\newcommand{\BHex}{\ensuremath{\hat{\boldsymbol{e}}_1}}
\newcommand{\BHey}{\ensuremath{\hat{\boldsymbol{e}}_2}}
\newcommand{\BHez}{\ensuremath{\hat{\boldsymbol{e}}_3}}
\newcommand{\BHn}[1]{\ensuremath{\hat{\boldsymbol{n}}_{#1}}}
\newcommand{\BHe}[1]{\ensuremath{\hat{\boldsymbol{e}}_{#1}}}
\newcommand{\BHg}[1]{\ensuremath{\hat{\boldsymbol{g}}_{#1}}}
\newcommand{\BHG}[1]{\ensuremath{\hat{\boldsymbol{G}}_{#1}}}
\newcommand{\Hn}{\ensuremath{\hat{\boldsymbol{n}}}}
\newcommand{\Mba}{\ensuremath{\mathbf{a}}}
\newcommand{\Mbatilde}{\ensuremath{\widetilde{\mathbf{a}}}}
\newcommand{\Mbb}{\ensuremath{\mathbf{b}}}
\newcommand{\Mbd}{\ensuremath{\mathbf{d}}}
\newcommand{\Mbf}{\ensuremath{\mathbf{f}}}
\newcommand{\Mbn}{\ensuremath{\mathbf{n}}}
\newcommand{\Mbntilde}{\ensuremath{\widetilde{\mathbf{n}}}}
\newcommand{\Mbr}{\ensuremath{\mathbf{r}}}
\newcommand{\Mbu}{\ensuremath{\mathbf{u}}}
\newcommand{\Mbv}{\ensuremath{\mathbf{v}}}
\newcommand{\Mbx}{\ensuremath{\mathbf{x}}}
\newcommand{\MbA}{\ensuremath{\mathbf{A}}}
\newcommand{\MbB}{\ensuremath{\mathbf{B}}}
\newcommand{\MbC}{\ensuremath{\mathbf{C}}}
\newcommand{\MbD}{\ensuremath{\mathbf{D}}}
\newcommand{\MbH}{\ensuremath{\mathbf{H}}}
\newcommand{\MbHbar}{\ensuremath{\mathbf{\overline{H}}}}
\newcommand{\MbI}{\ensuremath{\mathbf{I}}}
\newcommand{\MbK}{\ensuremath{\mathbf{K}}}
\newcommand{\MbKbar}{\ensuremath{\overline{\mathbf{K}}}}
\newcommand{\MbKtilde}{\ensuremath{\widetilde{\mathbf{K}}}}
\newcommand{\MbM}{\ensuremath{\mathbf{M}}}
\newcommand{\MbN}{\ensuremath{\mathbf{N}}}
\newcommand{\MbP}{\ensuremath{\mathbf{P}}}
\newcommand{\MbPbar}{\ensuremath{\overline{\mathbf{P}}}}
\newcommand{\MbR}{\ensuremath{\mathbf{R}}}
\newcommand{\MbT}{\ensuremath{\mathbf{T}}}
\newcommand{\MbV}{\ensuremath{\mathbf{V}}}
\newcommand{\MbX}{\ensuremath{\mathbf{X}}}
\newcommand{\MbSig}{\ensuremath{\boldsymbol{\sigma}}}
\newcommand{\Mbone}{\ensuremath{\mathbf{1}}}
\newcommand{\Mbzero}{\ensuremath{\mathbf{0}}}
%\newcommand{\Mb}{\ensuremath{\left[\mathsf{b}\right]}}
%\newcommand{\Mu}{\ensuremath{\left[\mathsf{u}\right]}}
%\newcommand{\Mv}{\ensuremath{\left[\mathsf{v}\right]}}
%\newcommand{\Mw}{\ensuremath{\left[\mathsf{w}\right]}}
%\newcommand{\Mx}{\ensuremath{\left[\mathsf{x}\right]}}
\newcommand{\MA}{\ensuremath{\left[\mathsf{A}\right]}}
\newcommand{\MB}{\ensuremath{\left[\mathsf{B}\right]}}
\newcommand{\MC}{\ensuremath{\left[\mathsf{C}\right]}}
\newcommand{\MD}{\ensuremath{\left[\mathsf{D}\right]}}
\newcommand{\MH}{\ensuremath{\left[\mathsf{H}\right]}}
\newcommand{\MI}{\ensuremath{\left[\mathsf{I}\right]}}
\newcommand{\ML}{\ensuremath{\left[\mathsf{L}\right]}}
\newcommand{\MM}{\ensuremath{\left[\mathsf{M}\right]}}
\newcommand{\MN}{\ensuremath{\left[\mathsf{N}\right]}}
\newcommand{\MP}{\ensuremath{\left[\mathsf{P}\right]}}
\newcommand{\MR}{\ensuremath{\left[\mathsf{R}\right]}}
\newcommand{\MT}{\ensuremath{\left[\mathsf{T}\right]}}
\newcommand{\MV}{\ensuremath{\left[\mathsf{V}\right]}}
%\newcommand{\Mone}{\ensuremath{\left[\mathsf{1}\right]}}
%\newcommand{\Mzero}{\ensuremath{\left[\mathsf{0}\right]}}
\newcommand{\SfA}{\ensuremath{\boldsymbol{\mathsf{A}}}}
\newcommand{\Sfc}{\ensuremath{\boldsymbol{\mathsf{c}}}}
\newcommand{\SfC}{\ensuremath{\boldsymbol{\mathsf{C}}}}
\newcommand{\SfD}{\ensuremath{\boldsymbol{\mathsf{D}}}}
\newcommand{\SfI}{\ensuremath{\boldsymbol{\mathsf{I}}}}
\newcommand{\SfL}{\ensuremath{\boldsymbol{\mathsf{L}}}}
\newcommand{\Sfp}{\ensuremath{\boldsymbol{\mathsf{p}}}}
\newcommand{\SfP}{\ensuremath{\boldsymbol{\mathsf{P}}}}
\newcommand{\SfS}{\ensuremath{\boldsymbol{\mathsf{S}}}}
\newcommand{\SfT}{\ensuremath{\boldsymbol{\mathsf{T}}}}
\newcommand{\Msig}{\ensuremath{\left[\boldsymbol{\sigma}\right]}}
\newcommand{\Meps}{\ensuremath{\left[\boldsymbol{\varepsilon}\right]}}
\newcommand{\Ex}{\ensuremath{\boldsymbol{e}_1}}
\newcommand{\Ey}{\ensuremath{\boldsymbol{e}_2}}
\newcommand{\Ez}{\ensuremath{\boldsymbol{e}_3}}
\newcommand{\Exp}{\ensuremath{\boldsymbol{e}^{'}_1}}
\newcommand{\Eyp}{\ensuremath{\boldsymbol{e}^{'}_2}}
\newcommand{\Ezp}{\ensuremath{\boldsymbol{e}^{'}_3}}
\newcommand{\Ep}{\ensuremath{\varepsilon_p}}
\newcommand{\Epi}{\ensuremath{\varepsilon_{pi}}}
\newcommand{\Epdot}[1]{\ensuremath{\dot{\varepsilon}_{p#1}}}
\newcommand{\Epsxx}{\ensuremath{\varepsilon_{11}}}
\newcommand{\Epsyy}{\ensuremath{\varepsilon_{22}}}
\newcommand{\Epszz}{\ensuremath{\varepsilon_{33}}}
\newcommand{\Epsyz}{\ensuremath{\varepsilon_{23}}}
\newcommand{\Epszx}{\ensuremath{\varepsilon_{31}}}
\newcommand{\Epsxy}{\ensuremath{\varepsilon_{12}}}
\newcommand{\Sigxx}{\ensuremath{\sigma_{11}}}
\newcommand{\Sigyy}{\ensuremath{\sigma_{22}}}
\newcommand{\Sigzz}{\ensuremath{\sigma_{33}}}
\newcommand{\Sigyz}{\ensuremath{\sigma_{23}}}
\newcommand{\Sigzx}{\ensuremath{\sigma_{31}}}
\newcommand{\Sigxy}{\ensuremath{\sigma_{12}}}
\newcommand{\Eps}[1]{\ensuremath{\varepsilon_{#1}}}
\newcommand{\Sig}[1]{\ensuremath{\sigma_{#1}}}
\newcommand{\X}{\ensuremath{X_1}}
\newcommand{\Y}{\ensuremath{X_2}}
\newcommand{\Z}{\ensuremath{X_3}}
\newcommand{\Balpha}{\ensuremath{\boldsymbol{\alpha}}}
\newcommand{\Balphahat}{\ensuremath{\widehat{\boldsymbol{\alpha}}}}
\newcommand{\Bchi}{\ensuremath{\boldsymbol{\chi}}}
\newcommand{\Beta}{\ensuremath{\boldsymbol{\eta}}}
\newcommand{\Bveps}{\ensuremath{\boldsymbol{\varepsilon}}}
\newcommand{\BGamma}{\ensuremath{\boldsymbol{\mathit{\Gamma}}}}
\newcommand{\BGammahat}{\ensuremath{\boldsymbol{\mathit{\widehat{\Gamma}}}}}
%\newcommand{\BGammahat}{\ensuremath{\widehat{\BGamma}}}
\newcommand{\Bkappa}{\ensuremath{\boldsymbol{\kappa}}}
\newcommand{\Bbeps}{\ensuremath{\bar{\boldsymbol{\varepsilon}}}}
\newcommand{\Bnabla}{\ensuremath{\boldsymbol{\nabla}}}
\newcommand{\Bomega}{\ensuremath{\boldsymbol{\omega}}}
\newcommand{\Bsig}{\ensuremath{\boldsymbol{\sigma}}}
\newcommand{\Btau}{\ensuremath{\boldsymbol{\tau}}}
\newcommand{\Bpi}{\ensuremath{\boldsymbol{\pi}}}
\newcommand{\Brho}{\ensuremath{\boldsymbol{\rho}}}
\newcommand{\Bvarphi}{\ensuremath{\boldsymbol{\varphi}}}
\newcommand{\Blambda}{\ensuremath{\boldsymbol{\lambda}}}
\newcommand{\Btheta}{\ensuremath{\boldsymbol{\theta}}}
\newcommand{\Bmu}{\ensuremath{\boldsymbol{\mu}}}
\newcommand{\Bxi}{\ensuremath{\boldsymbol{\xi}}}
\newcommand{\BPi}{\ensuremath{\boldsymbol{\Pi}}}
\newcommand{\Bone}{\ensuremath{\boldsymbol{\mathit{1}}}}
\newcommand{\Bonev}{\ensuremath{\boldsymbol{1}}}
\newcommand{\Bzero}{\ensuremath{\boldsymbol{0}}}
\newcommand{\BzeroT}{\ensuremath{\boldsymbol{\mathit{0}}}}
\newcommand{\Ba}{\ensuremath{\mathbf{a}}}
\newcommand{\Bb}{\ensuremath{\mathbf{b}}}
\newcommand{\BbT}{\ensuremath{\boldsymbol{b}}}
\newcommand{\Bc}{\ensuremath{\mathbf{c}}}
\newcommand{\Bd}{\ensuremath{\mathbf{d}}}
\newcommand{\BdT}{\ensuremath{\boldsymbol{d}}}
\newcommand{\Be}{\ensuremath{\mathbf{e}}}
\newcommand{\BeT}{\ensuremath{\boldsymbol{e}}}
\newcommand{\Bf}{\ensuremath{\mathbf{f}}}
\newcommand{\Bg}{\ensuremath{\mathbf{g}}}
\newcommand{\BTg}{\ensuremath{\boldsymbol{g}}}
\newcommand{\Bh}{\ensuremath{\mathbf{h}}}
\newcommand{\Bk}{\ensuremath{\mathbf{k}}}
\newcommand{\Bl}{\ensuremath{\mathbf{l}}}
\newcommand{\Bm}{\ensuremath{\mathbf{m}}}
\newcommand{\Bn}{\ensuremath{\mathbf{n}}}
\newcommand{\BnT}{\ensuremath{\boldsymbol{n}}}
\newcommand{\Bo}{\ensuremath{\mathbf{o}}}
\newcommand{\Bp}{\ensuremath{\mathbf{p}}}
\newcommand{\Bq}{\ensuremath{\mathbf{q}}}
\newcommand{\Br}{\ensuremath{\mathbf{r}}}
\newcommand{\Bs}{\ensuremath{\mathbf{s}}}
\newcommand{\BsT}{\ensuremath{\boldsymbol{s}}}
\newcommand{\Bt}{\ensuremath{\mathbf{t}}}
\newcommand{\Bu}{\ensuremath{\mathbf{u}}}
\newcommand{\Bv}{\ensuremath{\mathbf{v}}}
\newcommand{\Bw}{\ensuremath{\mathbf{w}}}
\newcommand{\Bx}{\ensuremath{\mathbf{x}}}
\newcommand{\By}{\ensuremath{\mathbf{y}}}
\newcommand{\Bz}{\ensuremath{\mathbf{z}}}
\newcommand{\BA}{\ensuremath{\boldsymbol{A}}}
\newcommand{\BB}{\ensuremath{\boldsymbol{B}}}
\newcommand{\BBbar}{\ensuremath{\bar{\boldsymbol{B}}}}
\newcommand{\BC}{\ensuremath{\boldsymbol{C}}}
\newcommand{\BCbar}{\ensuremath{\bar{\boldsymbol{C}}}}
\newcommand{\Cbar}{\ensuremath{\bar{C}}}
\newcommand{\BD}{\ensuremath{\boldsymbol{D}}}
\newcommand{\BE}{\ensuremath{\boldsymbol{E}}}
\newcommand{\BF}{\ensuremath{\boldsymbol{F}}}
\newcommand{\BFbar}{\ensuremath{\bar{\boldsymbol{F}}}}
\newcommand{\Fbar}{\ensuremath{\bar{F}}}
\newcommand{\BG}{\ensuremath{\boldsymbol{G}}}
\newcommand{\BH}{\ensuremath{\boldsymbol{H}}}
\newcommand{\BI}{\ensuremath{\boldsymbol{I}}}
\newcommand{\BJ}{\ensuremath{\boldsymbol{J}}}
\newcommand{\BK}{\ensuremath{\boldsymbol{K}}}
\newcommand{\BL}{\ensuremath{\boldsymbol{L}}}
\newcommand{\BM}{\ensuremath{\boldsymbol{M}}}
\newcommand{\BNv}{\ensuremath{\mathbf{N}}}
\newcommand{\BN}{\ensuremath{\boldsymbol{N}}}
\newcommand{\BP}{\ensuremath{\boldsymbol{P}}}
\newcommand{\BQ}{\ensuremath{\boldsymbol{Q}}}
\newcommand{\BR}{\ensuremath{\boldsymbol{R}}}
\newcommand{\BS}{\ensuremath{\boldsymbol{S}}}
\newcommand{\BT}{\ensuremath{\boldsymbol{T}}}
\newcommand{\BTv}{\ensuremath{\mathbf{T}}}
\newcommand{\BW}{\ensuremath{\boldsymbol{W}}}
\newcommand{\BX}{\ensuremath{\mathbf{X}}}
\newcommand{\BXT}{\ensuremath{\boldsymbol{X}}}
\newcommand{\BY}{\ensuremath{\boldsymbol{Y}}}
\newcommand{\Trial}{\ensuremath{\text{trial}}}
\newcommand{\Tint}{\ensuremath{\text{int}}}
\newcommand{\Text}{\ensuremath{\text{ext}}}
\newcommand{\Tkin}{\ensuremath{\text{kin}}}
\newcommand{\Tbody}{\ensuremath{\text{body}}}
\newcommand{\Tmin}{\ensuremath{\text{min}}}
\newcommand{\Tmax}{\ensuremath{\text{max}}}
\newcommand{\Tor}{\ensuremath{\text{or}}}
\newcommand{\Tr}{\ensuremath{\text{tr}}}
\newcommand{\Tdev}{\ensuremath{\text{dev}}}
\newcommand{\Tvol}{\ensuremath{\text{vol}}}
\newcommand{\Dev}{\ensuremath{\text{dev}}}
\newcommand{\Half}{\ensuremath{\tfrac{1}{2}}}
\newcommand{\SThr}{\ensuremath{\sqrt{3}}}
\newcommand{\STT}{\ensuremath{\frac{\sqrt{3}}{2}}}
\newcommand{\Third}{\ensuremath{\tfrac{1}{3}}}
\newcommand{\TwoThird}{\ensuremath{\tfrac{2}{3}}}
\newcommand{\Inner}[2]{\ensuremath{\langle#1,~#2\rangle}}
\newcommand{\Bcross}[2]{\ensuremath{#1\boldsymbol{\times}#2}}
\newcommand{\Bdot}[2]{\ensuremath{#1\cdot#2}}
\newcommand{\Dyad}[2]{\ensuremath{#1\boldsymbol{\otimes}#2}}
\newcommand{\Grad}[1]{\ensuremath{\Bnabla #1}}
\newcommand{\Gradp}[1]{\ensuremath{\Bnabla' #1}}
\newcommand{\Grads}[1]{\ensuremath{\Bnabla_s #1}}
\newcommand{\Grady}[1]{\ensuremath{\Bnabla_y #1}}
\newcommand{\Lap}[1]{\ensuremath{\nabla^2 #1}}
\newcommand{\Biharm}[1]{\ensuremath{\nabla^4 #1}}
\newcommand{\Div}[1]{\ensuremath{\Bdot{\Bnabla}{#1}}}
\newcommand{\Divp}[1]{\ensuremath{\Bdot{\Bnabla'}{#1}}}
\newcommand{\Divy}[1]{\ensuremath{\Bdot{\Bnabla_y}{#1}}}
\newcommand{\Curl}[1]{\ensuremath{\Bcross{\Bnabla}{#1}}}
\newcommand{\Curlp}[1]{\ensuremath{\Bcross{\Bnabla'}{#1}}}
\newcommand{\Curls}[1]{\ensuremath{\Bcross{\Bnabla_s}{#1}}}
\newcommand{\Curly}[1]{\ensuremath{\Bcross{\Bnabla_y}{#1}}}
\newcommand{\Gradu}{\ensuremath{\Grad{\Bu}}}
\newcommand{\Divu}{\ensuremath{\Div{\Bu}}}
\newcommand{\Curlu}{\ensuremath{\Curl{\Bu}}}
\newcommand{\Gradv}{\ensuremath{\Grad{\Bv}}}
\newcommand{\Divv}{\ensuremath{\Div{\Bv}}}
\newcommand{\Curlv}{\ensuremath{\Curl{\Bv}}}
\newcommand{\Dualn}{\ensuremath{\Bdual{\Bn}{\Bn}}}
\newcommand{\Over}[1]{\ensuremath{\frac{1}{#1}}}
\newcommand{\Diff}[2]{\ensuremath{\frac{d #1}{d #2}}}
\newcommand{\Partial}[2]{\ensuremath{\frac{\displaystyle\partial #1}{\displaystyle\partial #2}}}
\newcommand{\PPartial}[2]{\ensuremath{\frac{\partial^2 #1}{\partial #2^2}}}
\newcommand{\PPartialA}[3]{\ensuremath{\frac{\partial^2 #1}{\partial #2\partial#3}}}
\newcommand{\FPartial}[2]{\ensuremath{\frac{\partial^4 #1}{\partial #2^4}}}
\newcommand{\FPartialA}[3]{\ensuremath{\frac{\partial^4 #1}{\partial #2^2
         \partial #3^2}}}
\newcommand{\DotMbT}{\ensuremath{\dot{\MbT}}}
\newcommand{\TildeMbT}{\ensuremath{\widetilde{\MbT}}}
\newcommand{\BarT}{\ensuremath{\overline{T}}}
\newcommand{\Barq}{\ensuremath{\overline{q}}}
\newcommand{\Domega}{\ensuremath{\partial{\Omega}}}
\newcommand{\Av}[1]{\ensuremath{\left\langle#1\right\rangle}}
\newcommand{\AvSig}{\ensuremath{\langle\Bsig\rangle}}
\newcommand{\AvTau}{\ensuremath{\langle\Btau\rangle}}
\newcommand{\AvP}{\ensuremath{\langle\BP\rangle}}
\newcommand{\AvEps}{\ensuremath{\langle\Beps\rangle}}
\newcommand{\AvEpsdot}{\ensuremath{\langle\dot{\Beps}\rangle}}
\newcommand{\AvDisp}{\ensuremath{\langle\Bu\rangle}}
\newcommand{\AvF}{\ensuremath{\langle\BF\rangle}}
\newcommand{\AvFdot}{\ensuremath{\langle\dot{\BF}\rangle}}
\newcommand{\Avl}{\ensuremath{\overline{\Bl}}}
\newcommand{\AvSigBar}{\ensuremath{\overline{\Bsig}}}
\newcommand{\AvTauBar}{\ensuremath{\overline{\Btau}}}
\newcommand{\AvOmega}{\ensuremath{\langle\Bomega\rangle}}
\newcommand{\AvGradu}{\ensuremath{\langle\Gradu\rangle}}
\newcommand{\AvGradudot}{\ensuremath{\langle\Grad{\dot{\Bu}}\rangle}}
\newcommand{\AvGradv}{\ensuremath{\langle\Gradv\rangle}}
\newcommand{\AvPower}{\ensuremath{\langle\Bsig:\Gradv\rangle}}
\newcommand{\AvPowerInf}{\ensuremath{\langle\Bsig:\dot{\Beps}\rangle}}
\newcommand{\AvWorkInf}{\ensuremath{\langle\Bsig:\Beps\rangle}}
\newcommand{\AvPowerPF}{\ensuremath{\langle\BP^T:\dot{\BF}\rangle}}
\newcommand{\DA}{\ensuremath{\text{dA}}}
\newcommand{\DAvec}{\ensuremath{\text{d}\mathbf{A}}}
\newcommand{\Da}{\ensuremath{\text{da}}}
\newcommand{\Davec}{\ensuremath{\text{d}\mathbf{a}}}
\newcommand{\DV}{\ensuremath{\text{dV}}}
\newcommand{\BCe}{\ensuremath{\mathcal{E}}}
\newcommand{\GradX}[1]{\ensuremath{\Bnabla_0~#1}}
\newcommand{\DivX}[1]{\ensuremath{\Bdot{\Bnabla_0}{#1}}}
\newcommand{\Bxdot}{\ensuremath{\dot{\Bx}}}
\newcommand{\BFdot}{\ensuremath{\dot{\BF}}}
\newcommand{\BAv}{\ensuremath{\mathbf{A}}}
\newcommand{\BBv}{\ensuremath{\mathbf{B}}}
\newcommand{\BDv}{\ensuremath{\mathbf{D}}}
\newcommand{\BEv}{\ensuremath{\mathbf{E}}}
\newcommand{\BFv}{\ensuremath{\mathbf{F}}}
\newcommand{\BHv}{\ensuremath{\mathbf{H}}}
\newcommand{\BJv}{\ensuremath{\mathbf{J}}}
\newcommand{\BMv}{\ensuremath{\mathbf{M}}}
\newcommand{\BPv}{\ensuremath{\mathbf{P}}}
\newcommand{\BRv}{\ensuremath{\mathbf{R}}}
\newcommand{\BVv}{\ensuremath{\mathbf{V}}}
\newcommand{\Bdelta}{\ensuremath{\boldsymbol{\delta}}}
\newcommand{\Beps}{\ensuremath{\boldsymbol{\epsilon}}}
\newcommand{\BVeps}{\ensuremath{\boldsymbol{\varepsilon}}}
\newcommand{\BDtildev}{\ensuremath{\mathbf{\widetilde{D}}}}
\newcommand{\IntInfT}{\ensuremath{\int_{-\infty}^t}}
\newcommand{\IntInfInf}{\ensuremath{\int_{-\infty}^{\infty}}}
\newcommand{\IntInfZero}{\ensuremath{\int_{-\infty}^{0}}}
\newcommand{\IntZeroInf}{\ensuremath{\int_{0}^{\infty}}}
\newcommand{\IIntInfInf}{\ensuremath{\int_{-\infty}^{\infty}\int_{-\infty}^{\infty}}}
\newcommand{\IIIntInfInf}{\ensuremath{\int_{-\infty}^{\infty}\int_{-\infty}^{\infty}\int_{-\infty}^{\infty}}}
\newcommand{\Dtau}{\ensuremath{\text{d}\tau}}
\newcommand{\domega}{\ensuremath{\text{d}\omega}}
\newcommand{\dOmega}{\ensuremath{\text{d}\Omega}}
\newcommand{\dGamma}{\ensuremath{\text{d}\Gamma}}
\newcommand{\dzeta}{\ensuremath{\text{d}\zeta}}
\newcommand{\Ds}{\ensuremath{\text{d}s}}
\newcommand{\Dt}{\ensuremath{\text{d}t}}
\newcommand{\Dx}{\ensuremath{\text{d}\Bx}}
\newcommand{\dr}{\ensuremath{\text{d}r}}
\newcommand{\dx}{\ensuremath{\text{d}x}}
\newcommand{\dy}{\ensuremath{\text{d}y}}
\newcommand{\dz}{\ensuremath{\text{d}z}}
\newcommand{\dk}{\ensuremath{\text{d}k}}
\newcommand{\dBx}{\ensuremath{\text{d}\Bx}}
\newcommand{\dBk}{\ensuremath{\text{d}\Bk}}
\newcommand{\dBr}{\ensuremath{\text{d}\Br}}
\newcommand{\BKbar}{\ensuremath{\boldsymbol{\bar{K}}}}
\newcommand{\That}{\ensuremath{\widehat{T}}}
\newcommand{\Bahat}{\ensuremath{\widehat{\Ba}}}
\newcommand{\Bbhat}{\ensuremath{\widehat{\Bb}}}
\newcommand{\BAhat}{\ensuremath{\widehat{\BA}}}
\newcommand{\BBhat}{\ensuremath{\widehat{\BB}}}
\newcommand{\BBhatv}{\ensuremath{\widehat{\mathbf{B}}}}
\newcommand{\BDhatv}{\ensuremath{\widehat{\mathbf{D}}}}
\newcommand{\BEhatv}{\ensuremath{\widehat{\mathbf{E}}}}
\newcommand{\BFhatv}{\ensuremath{\widehat{\mathbf{F}}}}
\newcommand{\BHhatv}{\ensuremath{\widehat{\mathbf{H}}}}
\newcommand{\BPhatv}{\ensuremath{\widehat{\mathbf{P}}}}
\newcommand{\BVhatv}{\ensuremath{\widehat{\mathbf{V}}}}
\newcommand{\Rea}{\ensuremath{\text{Re}}}
\newcommand{\Img}{\ensuremath{\text{Im}}}
\newcommand{\Teff}{\ensuremath{\text{eff}}}
\newcommand{\Tand}{\ensuremath{\text{and}}}
\newcommand{\CalE}{\ensuremath{\mathcal{E}}}
\newcommand{\CalH}{\ensuremath{\mathcal{H}}}
\newcommand{\CalJ}{\ensuremath{\mathcal{J}}}
\newcommand{\CalU}{\ensuremath{\mathcal{U}}}
\newcommand{\Dhat}{\ensuremath{\widehat{D}}}
\newcommand{\Ehat}{\ensuremath{\widehat{E}}}
\newcommand{\Fhat}{\ensuremath{\widehat{F}}}
\newcommand{\Phat}{\ensuremath{\widehat{P}}}
\newcommand{\Uhat}{\ensuremath{\widehat{U}}}
\newcommand{\Vhat}{\ensuremath{\widehat{V}}}
\newcommand{\fhat}{\ensuremath{\widehat{f}}}
\newcommand{\ghat}{\ensuremath{\widehat{g}}}
\newcommand{\phat}{\ensuremath{\widehat{p}}}
\newcommand{\uhat}{\ensuremath{\widehat{u}}}
\newcommand{\vhat}{\ensuremath{\widehat{v}}}
\newcommand{\xhat}{\ensuremath{\widehat{x}}}
\newcommand{\yhat}{\ensuremath{\widehat{y}}}
\newcommand{\Beq}{\begin{equation}}
\newcommand{\Eeq}{\end{equation}}
\newcommand{\Bal}{\begin{aligned}}
\newcommand{\Eal}{\end{aligned}}
\newcommand{\Ibar}{\ensuremath{\bar{I}}}
\newcommand{\Tbar}{\ensuremath{\text{bar}}}
\newcommand{\Tball}{\ensuremath{\text{ball}}}
\newcommand{\Buhat}{\ensuremath{\widehat{\Bu}}}
\newcommand{\BHhat}{\ensuremath{\widehat{\BH}}}
\newcommand{\Bsighat}{\ensuremath{\widehat{\Bsig}}}
\newcommand{\Bepshat}{\ensuremath{\widehat{\Beps}}}
\newcommand{\sighat}{\ensuremath{\widehat{\sigma}}}
\newcommand{\epshat}{\ensuremath{\widehat{\epsilon}}}
\newcommand{\rhohat}{\ensuremath{\widehat{\rho}}}
\newcommand{\phihat}{\ensuremath{\widehat{\varphi}}}
\newcommand{\kappahat}{\ensuremath{\widehat{\kappa}}}
\newcommand{\SfK}{\ensuremath{\boldsymbol{\mathsf{K}}}}
\newcommand{\SfZero}{\ensuremath{\boldsymbol{\mathsf{0}}}}
\newcommand{\Gradbar}[1]{\ensuremath{\overline{\Bnabla} #1}}
\newcommand{\Divbar}[1]{\ensuremath{\Bdot{\overline{\Bnabla}}{#1}}}
\newcommand{\ktilde}{\ensuremath{\widetilde{k}}}
\newcommand{\Etilde}{\ensuremath{\widetilde{E}}}
\newcommand{\Htilde}{\ensuremath{\widetilde{H}}}
\newcommand{\Ktilde}{\ensuremath{\widetilde{K}}}
\newcommand{\Rtilde}{\ensuremath{\widetilde{R}}}
\newcommand{\Ttilde}{\ensuremath{\widetilde{T}}}
\newcommand{\TAi}{\ensuremath{\text{Ai}}}
\newcommand{\TBi}{\ensuremath{\text{Bi}}}
\newcommand{\sgn}{\ensuremath{\text{sgn}}}
\newcommand{\DelTwo}{\ensuremath{\Delta/2}}
\newcommand{\Bepseff}{\ensuremath{\Beps_\Teff}}
\newcommand{\Bmueff}{\ensuremath{\Bmu_\Teff}}
\newcommand{\epseff}{\ensuremath{\epsilon_\Teff}}
\newcommand{\mueff}{\ensuremath{\mu_\Teff}}
\newcommand{\BAeff}{\ensuremath{\BA_\Teff}}
\newcommand{\Conj}[1]{\ensuremath{\overline{#1}}}
\newcommand{\kappatilde}{\ensuremath{\widetilde{\kappa}}}
\newcommand{\lambdatilde}{\ensuremath{\widetilde{\lambda}}}

\newcommand{\MAmat}{\ensuremath{\uuline{\mathsf{A}}}}
\newcommand{\MBmat}{\ensuremath{\uuline{\mathsf{B}}}}
\newcommand{\MCmat}{\ensuremath{\uuline{\mathsf{C}}}}
\newcommand{\Ma}{\ensuremath{\uuline{\mathsf{a}}}}
\newcommand{\Matilde}{\ensuremath{\widetilde{\Ma}}}
\newcommand{\Mb}{\ensuremath{\uuline{\mathsf{b}}}}
\newcommand{\Mf}{\ensuremath{\uuline{\mathsf{f}}}}
\newcommand{\Mg}{\ensuremath{\uuline{\mathsf{g}}}}
\newcommand{\Mn}{\ensuremath{\uuline{\mathsf{n}}}}
\newcommand{\Mntilde}{\ensuremath{\widetilde{\Mn}}}
\newcommand{\Mp}{\ensuremath{\uuline{\mathsf{p}}}}
\newcommand{\Mq}{\ensuremath{\uuline{\mathsf{q}}}}
\newcommand{\Mr}{\ensuremath{\uuline{\mathsf{r}}}}
\newcommand{\Ms}{\ensuremath{\uuline{\mathsf{s}}}}
\newcommand{\Mnhat}{\ensuremath{\uuline{\widehat{\mathsf{n}}}}}
\newcommand{\Mqhat}{\ensuremath{\uuline{\widehat{\mathsf{q}}}}}
\newcommand{\Mshat}{\ensuremath{\uuline{\widehat{\mathsf{s}}}}}
\newcommand{\Mu}{\ensuremath{\uuline{\mathsf{u}}}}
\newcommand{\Mv}{\ensuremath{\uuline{\mathsf{v}}}}
\newcommand{\Mw}{\ensuremath{\uuline{\mathsf{w}}}}
\newcommand{\Mx}{\ensuremath{\uuline{\mathsf{x}}}}
\newcommand{\Mone}{\ensuremath{\uuline{\mathsf{1}}}}
\newcommand{\Mzero}{\ensuremath{\uuline{\mathsf{0}}}}

\newcommand{\erf}{\text{erf}}
\newcommand{\Xidot}{\ensuremath{\dot{\xi}}}
\newcommand{\BU}{\ensuremath{\boldsymbol{U}}}
\newcommand{\Bbeta}{\ensuremath{\boldsymbol{\beta}}}
\newcommand{\SfB}{\ensuremath{\boldsymbol{\mathsf{B}}}}



% For adding "DRAFT" to each page uncomment the following
% --------
\usepackage{eso-pic}
\makeatletter
\AddToShipoutPicture{%
  \setlength{\@tempdimb}{.5\paperwidth}%
  \setlength{\@tempdimc}{.5\paperheight}%
  \setlength{\unitlength}{1pt}%
  \put(\strip@pt\@tempdimb,\strip@pt\@tempdimc){%
    \makebox(0,0){\rotatebox{45}{\textcolor[gray]{0.9}%
      {\fontsize{6cm}{6cm}\selectfont{DRAFT}}}}%
  }%
}
\makeatother
% --------

\begin{document}
\DeclareGraphicsExtensions{.jpg,.pdf}

\title{Small strain elastic-plastic model}
\author{Biswajit Banerjee}

\maketitle
\tableofcontents
\newpage

\section{Preamble}
Let $\boldsymbol{F}$ be the deformation gradient, $\boldsymbol{\sigma}$ be the 
Cauchy stress, and $\boldsymbol{d}$ be the rate of deformation tensor.  We
first decompose the deformation gradient into a stretch and a rotations using
$\BF = \BR\cdot\BU$.  The rotation $\BR$ is then used to rotate the stress and
the rate of deformation into the material configuration to give us
\Beq
  \widehat{\Bsig} = \BR^T\cdot\Bsig\cdot\BR ~;~~
  \widehat{\Bd} = \BR^T\cdot\Bd\cdot\BR 
\Eeq
This is equivalent to using a Green-Naghdi objective stress rate.  In the 
following all equations are with respect to the hatted quantities and we 
drop the hats for convenience.

\section{Elastic relation}
Let us split the Cauchy stress into a volumetric and a deviatoric part
\Beq
  \Bsig = p~\Bone + \Bs ~;~~ p = \Third~\Tr(\Bsig) ~.
\Eeq
Taking the time derivative gives us
\Beq
  \dot{\Bsig} = \dot{p}~\Bone + \dot{\Bs} ~.
\Eeq
We assume that the elastic response of the material is isotropic.  The constitutive 
relation for a hypoelastic material of grade 0 can be expressed as
\Beq
  \dot{\Bsig} = \left[\lambda~\Tr(\Bd^e) - 3~\kappa~\alpha~\Deriv{}{t}(T-T_0)\right]~\Bone 
    + 2~\mu~\Bd^e  ~;~~ \Bd = \Bd^e + \Bd^p
\Eeq
where $\Bd^e, \Bd^p$ are the elastic and plastic parts of the rate of deformation 
tensor, $\lambda,\mu$ are the Lame constants, $\kappa$ is the bulk modulus, $\alpha$
is the coefficient of thermal expansion, $T_0$ is the reference temperature, and $T$ is the
current temperature.  If we split $\Bd^e$ into volumetric and deviatoric parts as
\Beq
  \Bd^e = \Third~\Tr(\Bd^e)~\Bone + \Beta^e
\Eeq
we can write
\Beq
  \dot{\Bsig} = \left[\left(\lambda + \cfrac{2}{3}~\mu\right)~\Tr(\Bd^e) 
     - 3~\kappa~\alpha~\Deriv{}{t}(T-T_0)\right]~\Bone + 2~\mu~\Beta^e  =
  \kappa~\left[\Tr(\Bd^e) - 3~\alpha~\Deriv{}{t}(T-T_0)\right]~\Bone + 2~\mu~\Beta^e  
\Eeq
Therefore, we have
\Beq
  \dot{\Bs} = 2~\mu~\Beta^e ~.
\Eeq
and
\Beq
  \dot{p} = \kappa~\left[\Tr(\Bd^e) - 3~\alpha~\Deriv{}{t}(T-T_0)\right] ~.
\Eeq
We will use a standard elastic-plastic stress update algorithm to integrate the rate 
equation for the deviatoric stress.  However, we will assume that the volumetric part 
of the Cauchy stress can be computed using an equation of state.  Then the final
Cauchy stress will be given by
\Beq
  \Bsig = \left[p(J) - 3~J~\Deriv{p(J)}{J}~\alpha~(T-T_0)\right]~\Bone + \Bs ~;~~ 
  J = \det(\BF)~.
\Eeq
(Note that we assume that the plastic part of the deformation is volume preserving.
This is not true for Gurson type models and will lead to a small error in the 
computed value of $\Bsig$.)

\section{Flow rule}
We assume that the flow rule is given by
\Beq
  \Bd^p = \dot{\gamma}~\Br
\Eeq
We can split $\Bd^p$ into a trace part and a trace free part, i.e.,
\Beq
  \Bd^p = \Third~\Tr(\Bd^p)~\Bone + \Beta^p 
\Eeq
Then, using the flow rule, we have
\Beq
  \Bd^p = \Third~\dot{\gamma}~\Tr(\Br)~\Bone + \Beta^p  ~.
\Eeq
Therefore we can write the flow rule as
\Beq
  \Beta^p = \dot{\gamma}~\left(-\Third~\Tr(\Br)~\Bone + \Br\right) ~.
\Eeq
Note that 
\Beq
  \Bd = \Bd^e + \Bd^p \quad \implies \quad
  \Tr(\Bd) = \Tr(\Bd^e) + \Tr(\Bd^p) ~.
\Eeq
Also,
\Beq
  \Bd = \Bd^e + \Bd^p \quad \implies \quad
  \Third~\Tr(\Bd)~\Bone + \Beta = 
  \Third~\Tr(\Bd^e)~\Bone + \Beta^e  + 
  \Third~\Tr(\Bd^p)~\Bone + \Beta^p ~.
\Eeq
Therefore,
\Beq
  \Beta = \Beta^e + \Beta^p ~.
\Eeq

\section{Isotropic and Kinematic hardening and porosity evolution rules}
We assume that the strain rate, temperature, and porosity can be fixed at the
beginning of a timestep and consider only the evolution of plastic strain and
the back stress while calculating the current stress.

We assume that the plastic strain evolves according to the relation
\Beq
  \dot{\Ve^p} = \dot{\gamma}~h^{\alpha}
\Eeq

We also assume that the back stress evolves according to the relation
\Beq
  \dot{\widehat{\Bbeta}} = \dot{\gamma}~\Bh^{\beta}
\Eeq
where $\hat{\Bbeta}$ is the back stress.  If $\Bbeta$ is the deviatoric part of
$\widehat{\Bbeta}$, then we can write
\Beq
  \dot{\Bbeta} = \dot{\gamma}~\Dev(\Bh^{\beta}) ~.
\Eeq

The porosity $\phi$ is assumed to evolve according to the relation
\Beq
  \dot{\phi} = \dot{\gamma}~h^{\phi} ~.
\Eeq

\section{Yield condition}
The yield condition is assumed to be of the form
\Beq
  f(\Bs, \Bbeta, \Ve^p, \phi, \dot{\Ve}, T, \dots) 
    = f(\Bxi, \Ve^p, \phi, \dot{\Ve}, T, \dots) = 0
\Eeq
where $\Bxi = \Bs - \Bbeta$ and $\Bbeta$ is the deviatoric part of $\widehat{\Bbeta}$.  
The Kuhn-Tucker loading-unloading conditions are
\Beq
  \dot{\gamma} \ge 0 ~;~~  f \le 0 ~;~~ \dot{\gamma}~f = 0
\Eeq
and the consistency condition is $\dot{f} = 0$.

\section{Temperature increase due to plastic dissipation}
The temperature increase due to plastic dissipation is assume to be
given by the rate equation
\Beq
  \dot{T} = \cfrac{\chi}{\rho~C_p}~\sigma_y~\dot{\Ve^p} ~.
\Eeq
The temperature is updated using
\Beq
  T_{n+1} = T_n + 
   \cfrac{\chi_{n+1}~\Delta t}{\rho_{n+1}~C_p}~\sigma^{n+1}_y~\dot{\Ve^p}_{n+1} ~.
\Eeq

\section{Continuum elastic-plastic tangent modulus}
To determine whether the material has undergone a loss of stability we need to compute
the acosutic tensor which needs the computation of the continuum elastic-plastic tangent
modulus.

To do that recall that
\Beq
  \Bsig = p~\Bone + \Bs \quad \implies \quad \dot{\Bsig} = \dot{p}~\Bone ~.
\Eeq
We assume that 
\Beq
  \dot{p} = J~\Partial{p}{J}~\Tr(\Bd) \qquad \Tand \quad
  \dot{\Bs} = 2~\mu~\Beta^e ~.
\Eeq
Now, the consistency condition requires that
\Beq
  \dot{f}(\Bs, \Bbeta, \Ve^p, \phi, \dot{\Ve}, T, \dots) = 0 ~.
\Eeq
Keeping $\dot{\Ve}$ and $T$ fixed over the time interval, we can use the chain rule
to get
\Beq
  \dot{f} = \Partial{f}{\Bs}:\dot{\Bs} + \Partial{f}{\Bbeta}:\dot{\Bbeta} + 
    \Partial{f}{\Ve^p}~\dot{\Ve^p} + \Partial{f}{\phi}~\dot{\phi} = 0~.
\Eeq
The needed rate equations are
\Beq
  \Bal
    \dot{\Bs} & = 2~\mu~\Beta^e = 2~\mu~(\Beta - \Beta^p) = 
      2~\mu~[\Beta - \dot{\gamma}~\Dev(\Br)] \\
    \dot{\Bbeta} & = \dot{\gamma}~\Dev(\Bh^{\beta}) \\
    \dot{\Ve^p} & = \dot{\gamma}~h^{\alpha} \\
    \dot{\phi} & = \dot{\gamma}~h^{\phi} 
  \Eal
\Eeq
Plugging these into the expression for $\dot{f}$ gives
\Beq
  2~\mu~\Partial{f}{\Bs}:[\Beta - \dot{\gamma}~\Dev(\Br)] 
    + \dot{\gamma}~\Partial{f}{\Bbeta}:\Dev(\Bh^{\beta}) 
    + \dot{\gamma}~\Partial{f}{\Ve^p}~h^{\alpha} 
    + \dot{\gamma}~\Partial{f}{\phi}~h^{\phi}  = 0
\Eeq
or,
\Beq
  \dot{\gamma} = \cfrac{2~\mu~\Partial{f}{\Bs}:\Beta}
    {2~\mu~\Partial{f}{\Bs}:\Dev(\Br) - \Partial{f}{\Bbeta}:\Dev(\Bh^{\beta})
     - \Partial{f}{\Ve^p}~h^{\alpha} - \Partial{f}{\phi}~h^{\phi}} ~.
\Eeq
Plugging this expression for $\dot{\gamma}$ into the equation for $\dot{\Bs}$,
we get
\Beq
  \dot{\Bs} = 2~\mu~\left[\Beta - \left( 
    \cfrac{2~\mu~\Partial{f}{\Bs}:\Beta}
    {2~\mu~\Partial{f}{\Bs}:\Dev(\Br) - \Partial{f}{\Bbeta}:\Dev(\Bh^{\beta})
     - \Partial{f}{\Ve^p}~h^{\alpha} - \Partial{f}{\phi}~h^{\phi}}\right)~\Dev(\Br)\right]~.
\Eeq
At this stage, note that a symmetric $\Bsig$ implies a symmetric $\Bs$ and hence a 
symmetric $\Beta$.  Also we assume that $\Br$ is symmetric (and hence $\Dev(\Br)$), which
is true if the flow rule is associated.  Then we can write
\Beq
  \Beta = \SfI^{4s}:\Beta \qquad \Tand \qquad \Dev(\Br) = \SfI^{4s}:\Dev(\Br)
\Eeq
where $\SfI^{4s}$ is the fourth-order symmetric identity tensor.  Also note that
if $\BA$, $\BC$, $\BD$ are second order tensors and $\SfB$ is a fourth order tensor,
then
\Beq
  (\BA:\SfB:\BC)~(\SfB:\BD) \equiv A_{ij}~B_{ijkl}~C_{kl}~B_{mnpq}~D_{pq}
     = (B_{mnpq}~D_{pq})~(A_{ij}~B_{ijkl})~C_{kl} \equiv [(\SfB:\BD)\otimes(\BA:\SfB)]:\BC ~.
\Eeq
Therefore we have
\Beq
  \dot{\Bs} = 2~\mu~\left[\SfI^{4s}:\Beta - \left( 
    \cfrac{2~\mu~[\SfI^{4s}:\Dev(\Br)]\otimes[\Partial{f}{\Bs}:\SfI^{4s}]}
    {2~\mu~\Partial{f}{\Bs}:\Dev(\Br) - \Partial{f}{\Bbeta}:\Dev(\Bh^{\beta})
     - \Partial{f}{\Ve^p}~h^{\alpha} - \Partial{f}{\phi}~h^{\phi}}\right):\Beta\right]~.
\Eeq
Also,
\Beq
  \SfI^{4s}:\Dev(\Br) = \Dev(\Br) \qquad \Tand \qquad
  \Partial{f}{\Bs}:\SfI^{4s} = \Partial{f}{\Bs} ~.
\Eeq
Hence we can write
\Beq
  \dot{\Bs} = 2~\mu~\left[\SfI^{4s} - \left( 
    \cfrac{2~\mu~\Dev(\Br)\otimes\Partial{f}{\Bs}}
    {2~\mu~\Partial{f}{\Bs}:\Dev(\Br) - \Partial{f}{\Bbeta}:\Dev(\Bh^{\beta})
     - \Partial{f}{\Ve^p}~h^{\alpha} - \Partial{f}{\phi}~h^{\phi}}\right)\right]:\Beta
\Eeq
or,
\Beq
  \dot{\Bs} = \SfB^{ep}:\Beta = \SfB^{ep}:\left[\Bd - \Third~\Tr(\Bd)~\Bone\right]
\Eeq
where
\Beq
  \SfB^{ep} := 
    2~\mu~\left[\SfI^{4s} - \left( 
    \cfrac{2~\mu~\Dev(\Br)\otimes\Partial{f}{\Bs}}
    {2~\mu~\Partial{f}{\Bs}:\Dev(\Br) - \Partial{f}{\Bbeta}:\Dev(\Bh^{\beta})
     - \Partial{f}{\Ve^p}~h^{\alpha} - \Partial{f}{\phi}~h^{\phi}}\right)\right] ~.
\Eeq
Adding in the volumetric component gives
\Beq
  \Bal
  \dot{\Bsig} & = \dot{p}~\Bone + \dot{\Bs} \\
   & = J~\Partial{p}{J}~\Tr(\Bd)~\Bone +  
       \SfB^{ep}:\left[\Bd - \Third~\Tr(\Bd)~\Bone\right] \\
   & = \left[3~J~\Partial{p}{J}~\Bone - \SfB^{ep}:\Bone\right]~\cfrac{\Bd:\Bone}{3}
       + \SfB^{ep}:\Bd \\
   & = J~\Partial{p}{J}~(\Bone\otimes\Bone):\Bd - \Third~[\SfB^{ep}:(\Bone\otimes\Bone)]:\Bd
       + \SfB^{ep}:\Bd ~.
  \Eal
\Eeq
Therefore,
\Beq
  \dot{\Bsig} = \left[J~\Partial{p}{J}~(\Bone\otimes\Bone) - 
      \Third~[\SfB^{ep}:(\Bone\otimes\Bone)] + \SfB^{ep}\right]:\Bd 
   = \SfC^{ep}:\Bd ~.
\Eeq
The quantity $\SfC^{ep}$ is the continuum elastic-plastic tangent modulus.  We also use the 
continuum elastic-plastic tangent modulus in the implicit version of the code.  However,
for improved accuracy and faster convergence, an algorithmically  consistent tangent modulus 
should be used instead.  That tangent modulus can be calculated in the usual manner and 
is left for development and implementation as an additional feature in the future.

\section{Stress update}
A standard return algorithm is used to compute the updated Cauchy stress.  Recall
that the rate equation for the deviatoric stress is given by
\Beq
  \dot{\Bs} = 2~\mu~\Beta^e ~.
\Eeq
Integrating the rate equation using a Backward Euler scheme gives
\Beq
  \Bs_{n+1} - \Bs_n = 2~\mu~\Delta~t~\Beta^e_{n+1}
    = 2~\mu~\Delta~t~(\Beta_{n+1} - \Beta^p_{n+1})
\Eeq
Now, from the flow rule, we have
\Beq
  \Beta^p = \dot{\gamma}~\left(\Br -\Third~\Tr(\Br)~\Bone\right) ~.
\Eeq
Define the deviatoric part of $\Br$ as
\Beq
  \Dev(\Br) := \Br -\Third~\Tr(\Br)~\Bone ~.
\Eeq
Therefore, 
\Beq
  \Bs_{n+1} - \Bs_n 
    = 2~\mu~\Delta~t~\Beta_{n+1} - 2~\mu~\Delta\gamma_{n+1}~\Dev(\Br_{n+1}) ~.
\Eeq
where $\Delta\gamma := \dot{\gamma}~\Delta t$.  Define the trial stress
\Beq
  \Bs^{\Trial} := \Bs_n + 2~\mu~\Delta~t~\Beta_{n+1} ~.
\Eeq
Then
\Beq
  \Bs_{n+1} = \Bs^{\Trial} - 2~\mu~\Delta\gamma_{n+1}~\Dev(\Br_{n+1}) ~.
\Eeq
Also recall that the back stress is given by
\Beq
  \dot{\Bbeta} = \dot{\gamma}~\Dev{\Bh}^{\beta}
\Eeq
The evolution equation for the back stress can be integrated to get
\Beq
  \Bbeta_{n+1} - \Bbeta_n = \Delta\gamma_{n+1}~\Dev(\Bh)^{\beta}_{n+1} ~.
\Eeq
Now,
\Beq
  \Bxi_{n+1} = \Bs_{n+1} - \Bbeta_{n+1} ~.
\Eeq
Plugging in the expressions for $\Bs_{n+1}$ and $\Bbeta_{n+1}$, we get
\Beq
  \Bxi_{n+1} = \Bs^{\Trial} - 2~\mu~\Delta\gamma_{n+1}~\Dev(\Br_{n+1}) 
     - \Bbeta_n - \Delta\gamma_{n+1}~\Dev(\Bh)^{\beta}_{n+1} ~.
\Eeq
Define
\Beq
  \Bxi^{\Trial} := \Bs^{\Trial} - \Bbeta_n ~.
\Eeq
Then
\Beq
  \Bxi_{n+1} = \Bxi^{\Trial} - \Delta\gamma_{n+1}(2~\mu~\Dev(\Br_{n+1}) + \Dev(\Bh)^{\beta}_{n+1}) ~.
\Eeq
Similarly, the evolution of the plastic strain is given by
\Beq
  \Ve^p_{n+1} = \Ve^p_{n} + \Delta\gamma_{n+1}~h^{\alpha}_{n+1}
\Eeq
and the porosity evolves as
\Beq
  \phi_{n+1} = \phi_n + \Delta\gamma_{n+1}~h^{\phi}_{n+1} ~.
\Eeq
The yield condition is discretized as
\Beq
  f(\Bs_{n+1}, \Bbeta_{n+1}, \Ve^p_{n+1}, \phi_{n+1}, \dot{\Ve}_{n+1}, T_{n+1}, \dots) = 
  f(\Bxi_{n+1}, \Ve^p_{n+1}, \phi_{n+1}, \dot{\Ve}_{n+1}, T_{n+1}, \dots) = 0 ~.
\Eeq
{\bf Important:} We assume that the derivatives with respect to $\dot{\Ve}$ and $T$ are
small enough to be neglected.

\subsection{Newton iterations}
We now have the following equations that have to be solved for $\Delta\gamma_{n+1}$:
\Beq
  \Bal
  \Bxi_{n+1} & = \Bxi^{\Trial} - \Delta\gamma_{n+1}(2~\mu~\Dev(\Br_{n+1}) + \Dev(\Bh)^{\beta}_{n+1})\\
  \Ve^p_{n+1} & = \Ve^p_{n} + \Delta\gamma_{n+1}~h^{\alpha}_{n+1} \\
  \phi_{n+1} & = \phi_n + \Delta\gamma_{n+1}~h^{\phi}_{n+1}  \\
  f(\Bxi_{n+1}, \Ve^p_{n+1}, \phi_{n+1}, \dot{\Ve}_{n+1}, T_{n+1}, \dots) & = 0 ~.
  \Eal
\Eeq

Recall that if $g(\Delta\gamma) = 0$ is a nonlinear equation that we have to solve
for $\Delta\gamma$, an iterative Newton method can be expressed as
\Beq
  \Delta\gamma^{(k+1)} = \Delta\gamma^{(k)} - \left[\Deriv{g}{\Delta\gamma}\right]^{-1}_{(k)}~g^{(k)} ~.
\Eeq
Define 
\Beq
  \delta\gamma := \Delta\gamma^{(k+1)} - \Delta\gamma^{(k)} ~.
\Eeq
Then, the iterative scheme can be written as
\Beq
  g^{(k)} + \left[\Deriv{g}{\Delta\gamma}\right]^{(k)}~\delta\gamma  = 0~.
\Eeq
In our case we have
\Beq
  \Bal
  \Ba(\Delta\gamma) = 0 &= -\Bxi + \Bxi^{\Trial} - \Delta\gamma(2~\mu~\Dev(\Br) + \Dev(\Bh)^{\beta})\\
  b(\Delta\gamma) = 0 &= -\Ve^p + \Ve^p_{n} + \Delta\gamma~h^{\alpha} \\
  c(\Delta\gamma) = 0 & = -\phi + \phi_n + \Delta\gamma~h^{\phi}  \\
  f(\Delta\gamma) = 0 &= f(\Bxi, \Ve^p, \phi, \dot{\Ve}, T, \dots) 
  \Eal
\Eeq
Therefore,
\Beq
  \Bal
  \Deriv{\Ba}{\Delta\gamma} & = 
   -\Partial{\Bxi}{\Delta\gamma}  - (2~\mu~\Dev(\Br) + \Dev(\Bh)^{\beta})
   - \Delta\gamma~\left(2~\mu~\Partial{\Dev(\Br)}{\Delta\gamma} + 
        \Partial{\Dev(\Bh)^{\beta}}{\Delta\gamma}\right) \\
   & =
   -\Partial{\Bxi}{\Delta\gamma}  - (2~\mu~\Dev(\Br) + \Dev(\Bh)^{\beta})
   - \Delta\gamma~\left(
      2~\mu~\Partial{\Dev(\Br)}{\Bxi}:\Partial{\Bxi}{\Delta\gamma} + 
      2~\mu~\Partial{\Dev(\Br)}{\Ve^p}~\Partial{\Ve^p}{\Delta\gamma} + 
      2~\mu~\Partial{\Dev(\Br)}{\phi}~\Partial{\phi}{\Delta\gamma} + 
      \right. \\
   & \qquad \qquad
      \left.
      \Partial{\Dev(\Bh)^{\beta}}{\Bxi}:\Partial{\Bxi}{\Delta\gamma} + 
      \Partial{\Dev(\Bh)^{\beta}}{\Ve^p}~\Partial{\Ve^p}{\Delta\gamma} +
      \Partial{\Dev(\Bh)^{\beta}}{\phi}~\Partial{\phi}{\Delta\gamma} 
      \right) \\
  \Deriv{b}{\Delta\gamma} & = -\Partial{\Ve^p}{\Delta\gamma} +  h^{\alpha} 
    + \Delta\gamma~\left(\Partial{h^{\alpha}}{\Bxi}:\Partial{\Bxi}{\Delta\gamma} + 
                        \Partial{h^{\alpha}}{\Ve^p}~\Partial{\Ve^p}{\Delta\gamma} + 
                        \Partial{h^{\alpha}}{\phi}~\Partial{\phi}{\Delta\gamma}\right) \\
  \Deriv{c}{\Delta\gamma} & = -\Partial{\phi}{\Delta\gamma} +  h^{\phi} 
    + \Delta\gamma~\left(\Partial{h^{\phi}}{\Bxi}:\Partial{\Bxi}{\Delta\gamma} + 
                        \Partial{h^{\phi}}{\Ve^p}~\Partial{\Ve^p}{\Delta\gamma} + 
                        \Partial{h^{\phi}}{\phi}~\Partial{\phi}{\Delta\gamma}\right) \\
  \Deriv{f}{\Delta\gamma} & 
     = \Partial{f}{\Bxi}:\Partial{\Bxi}{\Delta\gamma} + 
          \Partial{f}{\Ve^p}~\Partial{\Ve^p}{\Delta\gamma} +
          \Partial{f}{\phi}~\Partial{\phi}{\Delta\gamma} ~.
  \Eal
\Eeq
Now, define
\Beq
   \Delta\Bxi := \Partial{\Bxi}{\Delta\gamma}~\delta\gamma ~;~~
   \Delta\Ve^p := \Partial{\Ve^p}{\Delta\gamma}~\delta\gamma ~;~~
   \Delta\phi := \Partial{\phi}{\Delta\gamma}~\delta\gamma ~.
\Eeq
Then
\Beq
  \Bal
  \Ba^{(k)} & - \Delta\Bxi - [2~\mu~\Dev(\Br^{(k)}) + \Dev(\Bh)^{\beta (k)}]~\delta\gamma \\
   & \qquad \qquad
    - 2~\mu~\Delta\gamma~\left(
      \Partial{\Dev(\Br^{(k)})}{\Bxi}:\Delta\Bxi + 
      \Partial{\Dev(\Br^{(k)})}{\Ve^p}~\Delta\Ve^p + 
      \Partial{\Dev(\Br^{(k)})}{\phi}~\Delta\phi 
      \right) \\
   & \qquad \qquad
    - \Delta\gamma~\left(
      \Partial{\Dev(\Bh)^{\beta (k)}}{\Bxi}:\Delta\Bxi + 
      \Partial{\Dev(\Bh)^{\beta (k)}}{\Ve^p}~\Delta\Ve^p +
      \Partial{\Dev(\Bh)^{\beta (k)}}{\phi}~\Delta\phi
      \right)  = 0\\
  b^{(k)} & - \Delta\Ve^p + h^{\alpha}~\delta\gamma 
    + \Delta\gamma~\left(\Partial{h^{\alpha (k)}}{\Bxi}:\Delta\Bxi + 
                        \Partial{h^{\alpha (k)}}{\Ve^p}~\Delta\Ve^p +
                        \Partial{h^{\alpha (k)}}{\phi}~\Delta\phi\right)
     = 0 \\
  c^{(k)} & - \Delta\phi + h^{\phi}~\delta\gamma 
    + \Delta\gamma~\left(\Partial{h^{\phi (k)}}{\Bxi}:\Delta\Bxi + 
                        \Partial{h^{\phi (k)}}{\Ve^p}~\Delta\Ve^p +
                        \Partial{h^{\phi (k)}}{\phi}~\Delta\phi\right)
     = 0 \\
  f^{(k)} & + \Partial{f^{(k)}}{\Bxi}:\Delta\Bxi + 
          \Partial{f^{(k)}}{\Ve^p}~\Delta\Ve^p +
          \Partial{f^{(k)}}{\phi}~\Delta\phi 
      = 0
  \Eal
\Eeq
Because the derivatives of $\Br^{(k)}, \Bh^{\alpha (k)}, \Bh^{\beta (k)}, \Bh^{\phi (k)}$ with respect 
to $\Bxi, \Ve^p, \phi$ may be difficult to calculate, we instead use a semi-implicit scheme in our 
implementation where the quantities $\Br$, $h^{\alpha}$, $\Bh^{\beta}$, and $h^{\phi}$ are evaluated 
at $t_n$.  Then the problematic derivatives disappear and we are left with
\Beq
  \Bal
  \Ba^{(k)} - \Delta\Bxi - [2~\mu~\Dev(\Br_n) + \Dev(\Bh)^{\beta}_n]~\delta\gamma & = 0\\
  b^{(k)} - \Delta\Ve^p + h^{\alpha}_n~\delta\gamma & = 0 \\
  c^{(k)} - \Delta\phi + h^{\phi}_n~\delta\gamma & = 0 \\
  f^{(k)} + \Partial{f^{(k)}}{\Bxi}:\Delta\Bxi + 
          \Partial{f^{(k)}}{\Ve^p}~\Delta\Ve^p +
          \Partial{f^{(k)}}{\phi}~\Delta\phi & = 0
  \Eal
\Eeq
We now force $\Ba^{(k)}$, $b^{(k)}$, and $c^{(k)}$ to be zero at all times, leading
to the expressions
\Beq
  \Bal
  \Delta\Bxi & = - [2~\mu~\Dev(\Br_n) + \Dev(\Bh)^{\beta}_n]~\delta\gamma \\
  \Delta\Ve^p & =  h^{\alpha}_n~\delta\gamma \\
  \Delta\phi & =  h^{\phi}_n~\delta\gamma \\
  f^{(k)} + \Partial{f^{(k)}}{\Bxi}:\Delta\Bxi + 
          \Partial{f^{(k)}}{\Ve^p}~\Delta\Ve^p +
          \Partial{f^{(k)}}{\phi}~\Delta\phi & = 0
  \Eal
\Eeq
Plugging the expressions for $\Delta\Bxi$, $\Delta\Ve^p$, $\Delta\phi$ from the 
first three equations into the fourth gives us
\Beq
  f^{(k)} -\Partial{f^{(k)}}{\Bxi}:[2~\mu~\Dev(\Br_n) + \Dev(\Bh)^{\beta}_n]~\delta\gamma +
          h^{\alpha}_n~\Partial{f^{(k)}}{\Ve^p}~\delta\gamma  + 
          h^{\phi}_n~\Partial{f^{(k)}}{\phi}~\delta\gamma  = 0 
\Eeq
or
\Beq
  \Delta\gamma^{(k+1)} - \Delta\gamma^{(k)} = \delta\gamma = 
   \cfrac{f^{(k)}}
   {\Partial{f^{(k)}}{\Bxi}:[2~\mu~\Dev(\Br_n) + \Dev(\Bh^{\beta}_n)] - 
   h^{\alpha}_n~\Partial{f^{(k)}}{\Ve^p} -
   h^{\phi}_n~\Partial{f^{(k)}}{\phi} } ~.
\Eeq

\subsection{Algorithm}
The following stress update algorithm is used for each (plastic) time step:
\begin{enumerate}
  \item Initialize:
  \Beq
    k = 0 ~;~~ (\Ve^p)^{(k)} = \Ve^p_n ~;~~ \phi^{(k)} = \phi_n ~;~~
    \Bbeta^{(k)} = \Bbeta_n ~;~~ \Delta\gamma^{(k)} = 0 ~;~~
    \Bxi^{(k)} = \Bxi^{\Trial}~.
  \Eeq
  \item Check yield condition:
  \Beq
    f^{(k)} := f(\Bxi^{(k)}, (\Ve^p)^{(k)}, \phi^{(k)}, \dot{\Ve}_n, T_n, \dots)
  \Eeq
  If $f^{(k)} < \text{tolerance}$ then 
  go to step 5 else go to step 3.
  \item Compute updated $\delta\gamma^{(k)}$ using
  \Beq
    \delta\gamma^{(k)} = 
     \cfrac{f^{(k)}}
     {\Partial{f^{(k)}}{\Bxi}:[2~\mu~\Dev(\Br_n) + \Dev(\Bh^{\beta}_n)] - 
     h^{\alpha}_n~\Partial{f^{(k)}}{\Ve^p} - 
     h^{\phi}_n~\Partial{f^{(k)}}{\phi}} ~.
  \Eeq
  Compute
  \Beq
  \Bal
    \Delta\Bxi^{(k)} & = -[2~\mu~\Dev(\Br_n) + \Dev(\Bh^{\beta}_n)]~\delta\gamma^{(k)} \\
    (\Delta\Ve^p)^{(k)} & =  h^{\alpha}_n~\delta\gamma^{(k)}   \\
    \Delta\phi^{(k)} & =  h^{\phi}_n~\delta\gamma^{(k)}   
  \Eal
  \Eeq
  \item Update variables:
  \Beq
    \Bal
    (\Ve^p)^{(k+1)} & = (\Ve^p)^{(k)} + (\Delta\Ve^p)^{(k)} \\
    \phi^{(k+1)} & = \phi^{(k)} + \Delta\phi^{(k)} \\
    \Bxi^{(k+1)} & = \Bxi^{(k)} + \Delta\Bxi^{(k)} \\
    \Delta\gamma^{(k+1)} & = \Delta\gamma^{(k)} + \delta\gamma^{(k)}
    \Eal
  \Eeq
  Set $k \leftarrow k+1$ and go to step 2.
  \item Update and calculate back stress and the deviatoric part of Cauchy stress:
  \Beq
    \Ve^p_{n+1} = (\Ve^p)^{(k)} ~;~~
    \phi_{n+1} = \phi^{(k)} ~;~~
    \Bxi_{n+1} = \Bxi^{(k)} ~;~~
    \Delta\gamma_{n+1} = \Delta\gamma^{(k)}
  \Eeq
  and
  \Beq
    \Bal
    \widehat{\Bbeta}_{n+1} & = \widehat{\Bbeta}_n + \Delta\gamma_{n+1}~\Bh^{\beta}(\Bxi_{n+1}, \Ve^p_{n+1}, \phi_{n+1}) \\
    \Bbeta_{n+1} & = \widehat{\Bbeta}_{n+1} - \Third~\Tr(\widehat{\Bbeta}_{n+1})~\Bone \\
    \Bs_{n+1} & = \Bxi_{n+1} + \Bbeta_{n+1}
    \Eal
  \Eeq
  \item Update the temperature and the Cauchy stress
  \Beq
    \Bal
    T_{n+1} & = T_n + 
     \cfrac{\chi_{n+1}~\Delta t}{\rho_{n+1}~C_p}~\sigma^{n+1}_y~\dot{\Ve^p}_{n+1} 
     = T_n + 
     \cfrac{\chi_{n+1}~\Delta\gamma_{n+1}}{\rho_{n+1}~C_p}~\sigma^{n+1}_y~h^{\alpha}_{n+1} \\
    p_{n+1} & = p(J_{n+1}) \\ 
    \kappa_{n+1} & = J_{n+1}~\left[\Deriv{p(J)}{J}\right]_{n+1} \\
    \Bsig_{n+1} & = \left[p_{n+1} - 3~\kappa_{n+1}~\alpha~(T_{n+1}-T_0)\right]~\Bone + \Bs_{n+1}
    \Eal
  \Eeq
\end{enumerate}

\section{Examples}
Let us now look at a few examples.
\subsection{Example 1}
Consider the case of $J_2$ plasticity with the yield condition
\Beq
  f := \sqrt{\frac{3}{2}} \Norm{\Bs-\Bbeta}{} - \sigma_y(\Ve^p, \dot{\Ve}, T, \dots) = 
       \sqrt{\frac{3}{2}} \Norm{\Bxi}{} - \sigma_y(\Ve^p, \dot{\Ve}, T, \dots) \le 0 
\Eeq
where $\Norm{\Bxi} = \sqrt{\Bxi:\Bxi}$. Assume the associated flow rule
\Beq
  \Bd^p = \dot{\gamma}~\Br = \dot{\gamma}~\Partial{f}{\Bsig} = \dot{\gamma}~\Partial{f}{\Bxi} ~.
\Eeq
Then
\Beq
  \Br = \Partial{f}{\Bxi} = \sqrt{\frac{3}{2}}~\cfrac{\Bxi}{\Norm{\Bxi}{}} 
\Eeq
and
\Beq
  \Bd^p = \sqrt{\frac{3}{2}}~\dot{\gamma}~\cfrac{\Bxi}{\Norm{\Bxi}{}} ~;~~
  \Norm{\Bd^p}{} = \sqrt{\frac{3}{2}}~\dot\gamma ~.
\Eeq
The evolution of the equivalent plastic strain is given by
\Beq
  \dot{\Ve^p} = \dot{\gamma}~h^{\alpha} = \sqrt{\cfrac{2}{3}}~\Norm{\Bd^p}{} = \dot{\gamma}~.
\Eeq
This definition is consistent with the definition of equivalent plastic strain
\Beq
  \Ve^p = \int_0^t \dot{\Ve}^p~d\tau = 
   \int_0^t \sqrt{\cfrac{2}{3}}~\Norm{\Bd^p}{}~d\tau ~.
\Eeq
The evolution of porosity is given by (there is no evolution of porosity)
\Beq
  \dot{\phi} = \dot{\gamma}~h^{\phi} = 0
\Eeq
The evolution of the back stress is given by the Prager kinematic hardening rule
\Beq
  \dot{\widehat{\Bbeta}} = \dot{\gamma}~\Bh^{\beta} = \frac{2}{3}~H'~\Bd^p 
\Eeq
where $\widehat{\Bbeta}$ is the back stress and
$H'$ is a constant hardening modulus.  Also, the trace of $\Bd^p$ is 
\Beq
  \Tr(\Bd^p) = \sqrt{\frac{3}{2}}~\dot{\gamma}~\cfrac{\Tr(\Bxi)}{\Norm{\Bxi}{}}~.
\Eeq
Since $\Bxi$ is deviatoric, $\Tr(\Bxi) = 0$ and hence $\Bd^p = \Beta^p$.
Hence, $\widehat{\Bbeta} = \Bbeta$ (where $\Bbeta$ is the deviatoric part of $\widehat{\Bbeta}$), and
\Beq
  \dot{\Bbeta} = \sqrt{\frac{2}{3}}~H'~\dot{\gamma}~\cfrac{\Bxi}{\Norm{\Bxi}{}} ~.
\Eeq

These relation imply that
\Beq
  \boxed{
  \Bal
    \Br & = \sqrt{\frac{3}{2}}~\cfrac{\Bxi}{\Norm{\Bxi}{}} \\
     h^{\alpha} & = 1 \\
     h^{\phi} & = 0 \\
    \Bh^{\beta} & = \sqrt{\frac{2}{3}}~H'~\cfrac{\Bxi}{\Norm{\Bxi}{}} ~.
  \Eal
  }
\Eeq
We also need some derivatives of the yield function.  These are
\Beq
  \Bal
  \Partial{f}{\Bxi} & = \Br \\
  \Partial{f}{\Ve^p} & = -\Partial{\sigma_y}{\Ve^p} \\
  \Partial{f}{\phi} & = 0 ~.
  \Eal
\Eeq

Let us change the kinematic hardening model and use the Armstrong-Frederick
model instead, i.e.,
\Beq
  \dot{\Bbeta} = \dot{\gamma}~\Bh^{\beta} = \frac{2}{3}~H_1~\Bd^p - H_2~\Bbeta~\Norm{\Bd^p}{} ~.
\Eeq
Since
\Beq
  \Bd^p = \sqrt{\frac{3}{2}}~\dot{\gamma}~\cfrac{\Bxi}{\Norm{\Bxi}{}}
\Eeq
we have
\Beq
  \Norm{\Bd^p}{} = 
   \sqrt{\frac{3}{2}}~\dot{\gamma}~\cfrac{\Norm{\Bxi}{}}{\Norm{\Bxi}{}} = 
   \sqrt{\frac{3}{2}}~\dot{\gamma} ~.
\Eeq
Therefore,
\Beq
  \dot{\Bbeta} = \sqrt{\frac{2}{3}}~H_1~\dot{\gamma}~\cfrac{\Bxi}{\Norm{\Bxi}{}} 
    - \sqrt{\frac{3}{2}}~H_2~\dot{\gamma}~\Bbeta ~.
\Eeq
Hence we have
\Beq
  \boxed{
  \Bh^{\beta} = \sqrt{\frac{2}{3}}~H_1~\cfrac{\Bxi}{\Norm{\Bxi}{}} 
    - \sqrt{\frac{3}{2}}~H_2~\Bbeta ~.
   }
\Eeq

\subsection{Example 2}
Let us now consider a Gurson type yield condition with kinematic hardening.  In this
case the yield condition can be written as
\Beq
  f := \cfrac{3~\Bxi:\Bxi}{2~\sigma_y^2} + 
     2~q_1~\phi^{*}~\cosh\left(\cfrac{q_2~\Tr(\Bsig)}{2~\sigma_y}\right)
     - [1 + q_3~(\phi^*)^2]
\Eeq
where $\phi$ is the porosity and
\Beq
  \phi^* = \begin{cases}
             \phi & \text{for}~ \phi \le \phi_c \\
             \phi_c - \cfrac{\phi_u^* - \phi_c}{\phi_f - \phi_c}~(\phi - \phi_c) & 
              \text{for}~ \phi > \phi_c
           \end{cases}
\Eeq
Final fracture occurs for $\phi = \phi_f$ or when $\phi_u^* = 1/q_1$.  

Let us use an associated flow rule
\Beq
  \Bd^p = \dot{\gamma}~\Br = \dot{\gamma}~\Partial{f}{\Bsig} ~.
\Eeq
Then
\Beq
  \Br = \Partial{f}{\Bsig} = \cfrac{3~\Bxi}{\sigma_y^2} + \cfrac{q_1~q_2~\phi^{*}}{\sigma_y}~
   \sinh\left(\cfrac{q_2~\Tr(\Bsig)}{2~\sigma_y}\right)~\Bone ~.
\Eeq
In this case
\Beq
  \Tr(\Br) = \cfrac{3~q_1~q_2~\phi^{*}}{\sigma_y}~\sinh\left(\cfrac{q_2~\Tr(\Bsig)}{2~\sigma_y}\right)
  \ne 0 
\Eeq
Therefore,
\Beq
  \Bd^p \ne \Beta^p ~.
\Eeq

For the evolution equation for the plastic strain we use
\Beq
  (\Bsig-\widehat{\Bbeta}):\Bd^p = (1 - \phi)~\sigma_y~\dot{\Ve}^p
\Eeq
where $\dot{\Ve}^p$ is the effective plastic strain rate in the matrix material.  Hence,
\Beq
  \dot{\Ve}^p = \dot{\gamma}~h^{\alpha}
    = \dot{\gamma}~\cfrac{(\Bsig - \widehat{\Bbeta}):\Br}{(1 - \phi)~\sigma_y} ~.
\Eeq

The evolution equation for the porosity is given by
\Beq
  \dot{\phi} = (1 - \phi)~\Tr(\Bd^p) + A~\dot{\Ve^p}
\Eeq
where
\Beq
A = \cfrac{f_n}{s_n \sqrt{2\pi}} \exp [-1/2 (\Ve^p - \Ve_n)^2/s_n^2]
\Eeq
and $ f_n $ is the volume fraction of void nucleating particles, 
$ \Ve_n $ is the mean of the normal distribution of nucleation strains, and 
$ s_n $ is the standard deviation of the distribution.

Therefore,
\Beq
  \dot{\phi} = \dot{\gamma}~h^{\phi} =
    \dot{\gamma}~\left[(1 - \phi)~\Tr(\Br) + A~
    \cfrac{(\Bsig - \widehat{\Bbeta}):\Br}{(1 - \phi)~\sigma_y}\right] ~.
\Eeq

If the evolution of the back stress is given by the Prager kinematic hardening rule
\Beq
  \dot{\widehat{\Bbeta}} = \dot{\gamma}~\Bh^{\beta} = \frac{2}{3}~H'~\Bd^p 
\Eeq
where $\widehat{\Bbeta}$ is the back stress, then
\Beq
  \dot{\widehat{\Bbeta}} = \frac{2}{3}~H'~\dot{\gamma}~\Br ~.
\Eeq
Alternatively, if we use the Armstrong-Frederick model, then
\Beq
  \dot{\widehat{\Bbeta}} = \dot{\gamma}~\Bh^{\beta} = 
   \frac{2}{3}~H_1~\Bd^p - H_2~\widehat{\Bbeta}~\Norm{\Bd^p}{} ~.
\Eeq
Plugging in the expression for $\Bd^p$, we have
\Beq
  \dot{\widehat{\Bbeta}} = \dot{\gamma}~
  \left[\frac{2}{3}~H_1~\Br - H_2~\widehat{\Bbeta}~\Norm{\Br}{}\right] ~.
\Eeq
Therefore, for this model,
\Beq
  \boxed{
  \Bal
  \Br & = \cfrac{3~\Bxi}{\sigma_y^2} + \cfrac{q_1~q_2~\phi^{*}}{\sigma_y}~
   \sinh\left(\cfrac{q_2~\Tr(\Bsig)}{2~\sigma_y}\right)~\Bone  \\
  h^{\alpha} &  
    = \cfrac{(\Bsig - \Bbeta):\Br}{(1 - \phi)~\sigma_y} \\
  h^{\phi} & = 
    (1 - \phi)~\Tr(\Br) + A~
    \cfrac{(\Bsig - \widehat{\Bbeta}):\Br}{(1 - \phi)~\sigma_y}  \\
  \Bh^{\beta} & = 
   \frac{2}{3}~H_1~\Br - H_2~\widehat{\Bbeta}~\Norm{\Br}{}
  \Eal
  }
\Eeq
The other derivatives of the yield function that we need are
\Beq
  \Bal
  \Partial{f}{\Bxi} & = \cfrac{3~\Bxi}{\sigma_y^2} \\
  \Partial{f}{\Ve^p} & = \Partial{f}{\sigma_y}~\Partial{\sigma_y}{\Ve^p} 
   = -\left[\cfrac{3~\Bxi:\Bxi}{\sigma_y^3} +
     \cfrac{q_1~q_2~\phi^*~\Tr(\Bsig)}{\sigma_y^2}~
     \sinh\left(\cfrac{q_2~\Tr(\Bsig)}{2~\sigma_y}\right)\right]~
     \Partial{\sigma_y}{\Ve^p}\\
  \Partial{f}{\phi} & = 2~q_1~\Deriv{\phi^*}{\phi}~
    \cosh\left(\cfrac{q_2~\Tr(\sigma)}{2~\sigma_y}\right) 
    - 2~q_3~\phi^*~\Deriv{\phi^*}{\phi} ~.
  \Eal
\Eeq



\end{document}







